\documentclass[11pt,a4paper]{article}
\usepackage[utf8]{inputenc}
\usepackage[spanish]{babel}

\title{\textbf{Propuesta de tesina}}
\date{}

\begin{document}

\maketitle


\section{Situación del postulante}

Al día 10/12/2022 tengo aprobadas 33 materias de la carrera y sólo
tengo pendiente realizar la tesina para finalizar la carrera de
Licenciatura en Ciencias de la Computación. Me encuentro trabajando
como docente en la Facultad de Ciencias Exactas, Ingeniería y
Agrimensura (UNR) 20 horas semanales (2 cargos simples).

\section{Título}

??

\section{Motivación y objetivo general}

La ley de Moore postulaba que el número de transistores que se pueden
poner un chip de computadora dobla (de manera aproximada) cada un par
de años. De manera práctica, esto significaba que para obtener
mejorías en la velocidad de ejecución de un programa simplemente había
que esperar: después de un par de años, los programas que escribiríamos
hoy ejecutarían más rápido entonces.

Sin embargo, esta ley ya no se verifica en la práctica. Por razones
físicas, es cada vez más costoso aumentar el número de transistores de
un chip, y este tipo de progreso ya no puede ser garantizado. Los
programadores se ven obligados a encontrar otras maneras de mejorar el
desempeño de sus programas, y una de las maneras más naturales que
surgió fue pensar que podemos resolver varias tareas al mismo tiempo,
y de esa manera evitar puntos ociosos donde solo estamos esperando el
entorno (por ej. I/O bloqueante). En los últimos años, con la
masificación de los microprocesadores con múltiples core, se hace aún
más evidente la necesidad de proveer al programador con la capacidad
de concurrencia.

Si bien los sistemas operativos modernos nos dan primitivas para
manejar la concurrencia, no siempre los lenguajes de programación han
incorporado estas características de manera natural. Los lenguajes de
programación imperativos en general se basan en la idea de un modelo
de ejecución secuencial, donde la computación se desarrolla siguiendo
una única serie de pasos. Es así que lenguajes como C o Java pueden
manejar concurrencia, pero sin proponer cambios radicales en la
concepción del lenguaje: simplemente librerías que capturan esta
capacidad de manera externa.

La misma situación se refleja en los lenguajes de programación
funcionales, sobre todo cuando se intenta escribir programas con
efectos. Los programas con efectos suelen representarse utilizando
mónadas, de manera que los programas funcionales toman una apariencia
fundamentalmente imperativa. Al igual que lo que sucede en la
programación imperativa, al considerar la concurrencia de los efectos, la
manera usual de propuesta es mediante llamadas a funciones ad-hoc, y
no usando primitivas bien fundadas que deriven de alguna estructura
matemática, así como lo hacen las operaciones de las mónadas.

Las mónadas concurrentes fueron recientemente definidas buscando
obtener primitivas bien fundadas para la concurrencia, extendiendo las
mónadas con nuevos operadores. Si bien esta estructura es
matemáticamente bien fundada, basada en monoides concurrentes, es
considerablemente complicado encontrar y probar modelos válidos de
esta estructura.

Una hipótesis particular es que la mónada de Delay, utilizada para
modelizar el efecto de no terminación, puede ser dotada de estructura
de mónada concurrente. En esta tesina desarrollaremos el concepto de
mónada concurrente en el ámbito del lenguaje de pruebas Agda, y con el
objetivo principal de probar o refutar la hipótesis de que a la mónada
Delay se le puede dar una estructura de mónada concurrente.

\section{Fundamentos y estado de conocimiento sobre el tema}

El uso de mónadas para estructuras semánticas de lenguajes con efectos
fue desarrollado originalmente por E. Moggi~\cite{Mog:CLM,
  Mog:NCM}. Poco más tarde, P. Wadler~\cite{Wad:EFP} adaptó el
concepto de manera interna en los lenguajes de programación funcional,
dando origen a la programación con mónadas. Existe una gran variedad
de efectos que pueden ser capturados usando mónadas internas, por
ejejmplo, estado, excepciones, entornos, continuaciones, etc.

La mónada Delay fue introducida por Capretta~\cite{Cap:GRCT} con el
objetivo de capturar el efecto de no terminación de manera explícita y
uniforme. Su estructura fue estudiada en varios artículos, entre los
que podemos mencionar la tesis de N. Veltri.

Las mónadas concurrentes fueron recientemente introducidas en una
pre-impresión por M. Jaskelioff y E. Rivas~\cite{RJ:MM}. Esta
definición surge de categorificar la noción de monoide concurrente,
definido por C. A. R. Hoare et al.~\cite{HHMOHPS:concmonoid} hace unos
años.

Agda es un asistente de pruebas basado en teoría de tipos, con soporte
induction-recursion, y que ya ha sido utilizado para estudiar la
mónada Delay.

\section{Metodología y plan de trabajo}

\begin{enumerate}

\item Estudio de la estructura de mónada concurrente.

\item Introducción de la clase de mónada concurrente en Agda.

\item Estudio de la mónada Delay y su implementación en Agda.

\item Adaptación de las operaciones de la mónada Delay en el contexto de
2, y las pruebas de estructura de mónada y functor monoidal.

\item Prueba del axioma de intercambio de mónada concurrente en Agda
para la mónada Delay.

\end{enumerate}

\bibliographystyle{plain}
\bibliography{referencias.bib}

\end{document}

