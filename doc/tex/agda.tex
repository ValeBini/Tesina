\chapter{Introducci\'on a Agda} \label{chapter:agda}

Agda es un lenguaje de programación funcional desarrollado inicialmente por Ulf Norell en la Universidad de Chalmers como parte de su tesis doctoral \cite{norell:thesis} que se caracteriza por tener tipos dependientes. A diferencia de otros lenguajes donde hay una separación clara entre el mundo de los tipos y el de los valores, en un lenguaje con tipos dependientes estos universos están más entremezclados. Los tipos pueden contener valores arbitrarios (lo que los hace \textit{depender} de ellos) y pueden aparecer también como argumentos o resultados de funciones.

El hecho de que los tipos puedan contener valores, permite que se puedan escribir propiedades de ciertos valores como tipos. Los elementos de estos tipos son pruebas de que la propiedad que representan es verdadera. Esto hace que los lenguajes con tipos dependientes puedan ser utilizados como una lógica. Esta fue la idea principal de la teoría de tipos desarrollada por Martin Löf, en la cual está basado el desarrollo de Agda. Una característica importante de esta teoría es su enfoque constructivista, en el cual para demostrar la existencia de un objeto debemos construirlo.

Para poder utilizar a Agda como una lógica se necesita que sea consistente, y es por eso que se requiere que todos los programas sean totales, es decir que no tienen permitido fallar o no terminar. En consecuencia, Agda incluye mecanismos que comprueban la terminación de los programas.

El objetivo de esta sección es presentar una introducción a Agda, haciendo énfasis en las características necesarias para exponer la temática de esta tesina. 

\section{Tipos de datos y \textit{Pattern Matching}}\label{agda:tipos}

Un concepto clave en Agda es el \textit{pattern matching} sobre tipos de datos algebraicos. Al agregar los tipos dependientes el \textit{pattern matching} se hace aún más poderoso. Se verá este tema más en detalle en la sección \ref{agda:dependent}. Para comenzar, en esta sección se describirán las funciones y tipos de datos con tipos simples. 

Los tipos de datos se definen utilizando una declaración \AgdaKeyword{data}, en la que se especifica el nombre y el tipo del tipo de dato a definir, así como los constructores y sus respectivos tipos. En el siguiente código se puede ver una forma de definir el tipo de los booleanos:

\ExecuteMetaData[latex/Agda.tex]{booleans}

El tipo de \AgdaDatatype{Bool} es \AgdaPrimitiveType{Set}, el tipo de los tipos pequeños. Las funciones sobre \AgdaDatatype{Bool} pueden definirse por \textit{pattern matching}:

\ExecuteMetaData[latex/Agda.tex]{not}

Las funciones en Agda no tienen permitido fallar, por lo que una función debe cubrir todos los casos posibles. Esto será constatado por el \textit{type checker}, el cual lanzará un error si hay casos no definidos. 

Otro tipo de dato que puede ser útil son los números naturales. 

\ExecuteMetaData[latex/Agda.tex]{naturals}

La suma sobre números naturales puede ser definida como una función recursiva (también utilizando \textit{pattern matching}). 

\ExecuteMetaData[latex/Agda.tex]{add}

Para garantizar la terminación de la función, las llamadas recursivas deben ser aplicadas a argumentos más pequeños que los originales. En este caso, \AgdaFunction{\_+\_} pasa el chequeo de terminación ya que el primer argumento se hace más pequeño en la llamada recursiva. 

Si el nombre de una función contiene guiones bajos (\AgdaSymbol{\_}), entonces puede ser utilizado como un operador en el cual los argumentos se posicionan donde están los guiones bajos. En consecuencia, la función \AgdaFunction{\_+\_} puede ser utilizada como un operador infijo escribiendo \AgdaArgument{n} \AgdaFunction{+} \AgdaArgument{m} en lugar de \AgdaFunction{\_+\_} \AgdaArgument{n m}. 

La precedencia y asociatividad de un operador se define utilizando una declaración \AgdaKeyword{infix}. Para mostrar esto se agregará, además de la suma, una función producto (la cual tiene más precedencia que la suma). La precedencia y asociatividad de ambas funciones podrían escribirse de la siguiente manera:

\ExecuteMetaData[latex/Agda.tex]{precedence}

La palabra clave \AgdaKeyword{infixl} indica que se asocia a izquierda (de igual manera existe \AgdaKeyword{infixr} para asociar a derecha o \AgdaKeyword{infix} si no se asocia hacia ningún lado) y el número que sigue indica la precedencia, operadores con mayor número tendrán más precedencia que operadores con menor número.

\subsection{Tipos de datos parametrizados}\label{tipos:parametrized}

Los tipos de datos pueden estar parametrizados por otros tipos de datos. El tipo de las listas de elementos de tipo arbitrario se define de la siguiente manera:

\ExecuteMetaData[latex/Agda.tex]{lists}

En este ejemplo el tipo \AgdaDatatype{List} está parametrizado por el tipo \AgdaArgument{A}, el cual define el tipo de dato que tendrán los elementos de las listas. \AgdaDatatype{List} \AgdaArgument{$\mathbb{N}$} es el tipo de las listas de números naturales. 


\subsection{Patrones con puntos}\label{tipos:puntos}

En algunos casos, al definir una función por \textit{pattern matching}, ciertos patrones de un argumento fuerzan que otro argumento tenga un único valor posible que tipe correctamente. Para indicar que el valor de un argumento fue deducido por chequeo de tipos y no observado por \textit{pattern matching},  se le agrega delante un punto (\AgdaSymbol{.}). Para mostrar un ejemplo de uso de un patrón con punto, se considerará el siguiente tipo de dato \AgdaDatatype{Square} definido como sigue:

\ExecuteMetaData[latex/Agda.tex]{square}

El tipo \AgdaDatatype{Square} \AgdaArgument{n} representa una propiedad sobre el número \AgdaArgument{n}, la cual dice
dicho número es un cuadrado perfecto. Un habitante de tal tipo es una prueba de que el número \AgdaArgument{n} efectivamente es un cuadrado perfecto. Si se quisiera definir entonces una función \AgdaFunction{root} que tome un natural y una prueba de que dicho natural es un cuadrado perfecto, y devuelva su raíz cuadrada, podría realizarse de la siguiente manera:

\ExecuteMetaData[latex/Agda.tex]{root}

Se puede observar que al \textit{matchear} el argumento de tipo \AgdaDatatype{Square} \AgdaArgument{n} con el constructor \AgdaFunction{sq} aplicado a un natural \AgdaArgument{m}, \AgdaArgument{n} se ve forzado a ser igual a \AgdaArgument{m} \AgdaFunction{*} \AgdaArgument{m}.


\subsection{Patrones absurdos}\label{tipos:absurdos}

Otro tipo de patrón especial es el patrón absurdo. Usar un patrón absurdo al definir una función significa que no es necesario dar una definición para ese caso ya que no hay manera alguna en la que alguien pudiera dar un argumento para la función. El tipo de dato definido a continuación será de utilidad para ver un ejemplo de este tipo de patrones:

\ExecuteMetaData[latex/Agda.tex]{even}

El tipo \AgdaDatatype{Even} \AgdaArgument{n} representa, al igual que \AgdaDatatype{Square} \AgdaArgument{n}, una propiedad sobre \AgdaArgument{n}. En este caso la propiedad afirma que \AgdaArgument{n} es un número par. Un habitante de este tipo es una prueba de que dicha proposición se cumple. 

Si se quisiera definir una función que, dado un número y una prueba de que es par, devuelva el resultado de dividirlo por dos, podría realizarse de la siguiente manera:

\ExecuteMetaData[latex/Agda.tex]{half}

Se puede ver que en el caso del \AgdaNumber{1}, no existe una prueba de que ese número sea par, y por lo tanto no debemos dar una definición para ese caso. Requerir la pureba de paridad nos asegura que no hay riesgo de intentar dividir por dos un número impar. 

\subsection{El constructor \texttt{with}}\label{tipos:with}

A veces no alcanza con hacer \textit{pattern matching} sobre los argumentos de una función, sino que se necesita analizar por casos el resultado de alguna computación intermedia. Para esto se utiliza el constructor \AgdaKeyword{with}. 

Si se tiene una expresión \AgdaArgument{e} en la definición de una función \AgdaFunction{f}, se puede abstraer \AgdaFunction{f} sobre el valor de \AgdaArgument{e}. Al hacer esto se agrega a \AgdaFunction{f} un argumento extra, sobre el cual se puede hacer \textit{pattern matching} al igual que con cualquier otro argumento. 

Para proveer un ejemplo de uso del constructor \AgdaKeyword{with}, se definirá a continuación la relación de orden \AgdaFunction{\_<\_} sobre los números naturales.

\ExecuteMetaData[latex/Agda.tex]{less}

Si se quisiera definir entonces, utilizando esta función, una función \AgdaFunction{min} que calcule el mínimo entre dos números naturales \AgdaArgument{x} e \AgdaArgument{y}, se debería analizar cuál es el resultado de calcular \AgdaArgument{x} \AgdaFunction{<} \AgdaArgument{y}. Esto se escribe haciendo uso del constructor \AgdaKeyword{with} como sigue:

\ExecuteMetaData[latex/Agda.tex]{min}

El argumento extra que se agrega está separado por una barra vertical y corresponde al valor de la expresión \AgdaArgument{x} \AgdaFunction{<} \AgdaArgument{y}. Se puede realizar esta abstracción sobre varias expresiones a la vez, separándolas entre ellas mediante barras verticales. Las abstracciones \AgdaKeyword{with} también pueden anidarse. En el lado izquierdo de las ecuaciones, los argumentos abstraídos con \AgdaKeyword{with} deben estar separados también con barras verticales. 

En este caso, el valor que tome \AgdaArgument{x} \AgdaFunction{<} \AgdaArgument{y} no cambia nada la información que se tiene sobre los argumentos \AgdaArgument{x} e \AgdaArgument{y}, por lo que volver a escribirlos no es necesario, puede reemplazarse la parte izquierda por tres puntos como se muestra a continuación:

\ExecuteMetaData[latex/Agda.tex]{min2} 

\section{Funciones dependientes}\label{agda:dependent}

Como se mencionó anteriormente, una de las principales características de Agda es que tiene tipos dependientes. El tipo dependiente más básico de todos son las funciones dependientes, en las cuales el tipo del resultado depende del valor del argumento. En Agda se escribe \AgdaSymbol{(}\AgdaArgument{x} \AgdaSymbol{:} \AgdaDatatype{A}\AgdaSymbol{)} \AgdaSymbol{$\rightarrow$} \AgdaDatatype{B} para indicar el tipo de una función que toma un argumento \AgdaArgument{x} de tipo \AgdaDatatype{A} y devuelve un resultado de tipo \AgdaDatatype{B}, donde \AgdaArgument{x} puede aparecer en \AgdaDatatype{B}. Un caso especial de esto es cuando \AgdaArgument{x} es un tipo en sí mismo. Se podría definir, por ejemplo:

\ExecuteMetaData[latex/Agda.tex]{id}

\AgdaFunction{identity} es una función dependiente que toma como argumento un tipo \AgdaDatatype{A} y un elemento de \AgdaDatatype{A} y retorna dicho elemento. De esta manera se codifican las funciones polimórficas en Agda. 

A continuación se muestra un ejemplo de una función dependiente menos trivial, la cual toma una función dependiente y la aplica a cierto argumento:

\ExecuteMetaData[latex/Agda.tex]{apply}

\subsection{Argumentos Implícitos}\label{dependent:implicit}

Los tipos dependientes sirven para definir funciones polimórficas. Pero en los ejemplos provistos en la sección anterior se da de forma explícita el tipo al cual cierta función polimórfica se debe aplicar. Usualmente esto es diferente. En general se espera que el tipo sobre el cual se va a aplicar una función polimórfica sea inferido por el \textit{type checker}. Para solucionar este problema, Agda utiliza un mecanismo de \textit{argumentos implícitos}. 

Para declarar un argumento implícito de una función, se utilizan llaves en lugar de paréntesis. \AgdaSymbol{\{}\AgdaArgument{x} \AgdaSymbol{:} \AgdaDatatype{A}\AgdaSymbol{\}} \AgdaSymbol{$\rightarrow$} \AgdaDatatype{B} significa lo mismo que \AgdaSymbol{(}\AgdaArgument{x} \AgdaSymbol{:} \AgdaDatatype{A}\AgdaSymbol{)} \AgdaSymbol{$\rightarrow$} \AgdaDatatype{B}, excepto que cuando se utiliza una función de este tipo el verificador de tipos intenta inferir el argumento por su cuenta. 

Con esta nueva sintaxis puede definirse una nueva versión de la función identidad, donde no es necesario explicitar el tipo argumento:

\ExecuteMetaData[latex/Agda.tex]{id2} 

Se puede observar que el tipo argumento es implícito tanto cuando la función se aplica como cuando es definida. No hay restricciones sobre cuáles o cuántos argumentos pueden ser implícitos, así como tampoco hay garantías de que estos puedan ser efectivamente inferidos por el \textit{type checker}. 

Para dar explícitamente un argumento implícito se usan también llaves. \AgdaFunction{f} \AgdaSymbol{\{}\AgdaArgument{v}\AgdaSymbol{\}} asigna \AgdaArgument{v} al primer argumento implícito de \AgdaFunction{f}. Si se requiere explicitar un argumento que no es el primero, escribe \AgdaFunction{f} \AgdaSymbol{\{}\AgdaArgument{x} \AgdaSymbol{=} \AgdaArgument{v}\AgdaSymbol{\}}, lo cual asigna \AgdaArgument{v} al argumento implícito llamado \AgdaArgument{x}. El nombre de un argumento implícito se obtiene de la declaración del tipo de la función. 

Si se desea, por el contrario, que el verificador de tipos infiera un término que debería darse explícitamente, se puede reemplazar por un guión bajo. Por ejemplo:

\ExecuteMetaData[latex/Agda.tex]{one'}

\section{Familias de Tipos de datos}\label{agda:family}

Se definió en el apartado \ref{tipos:parametrized} el tipo de las listas de tipo arbitrario parametrizado por \AgdaArgument{A}. Estas listas pueden tener cualquier largo, tanto una lista vacía como una lista con un millón de elementos son de tipo \AgdaDatatype{List} \AgdaArgument{A}. En ciertos casos es útil que el tipo restrinja el largo que tiene la lista, y es así como surgen las listas de largo definido, llamadas comúnmente vectores, que se definen como sigue:

\ExecuteMetaData[latex/Agda.tex]{vec}

El tipo de \AgdaDatatype{Vec} \AgdaArgument{A} es \AgdaDatatype{$\mathbb{N}$} \AgdaSymbol{$\rightarrow$} \AgdaPrimitiveType{Set}. Esto significa que \AgdaDatatype{Vec} \AgdaArgument{A} es una familia de tipos indexada por los números naturales. Por lo tanto, para cada \AgdaArgument{n} natural, \AgdaDatatype{Vec} \AgdaArgument{A n} es un tipo. Los constructores pueden construir elementos de cualquier tipo dentro de la familia. Hay una diferencia sustancial entre parámetros e índices de un tipo de dato. Se dice que \AgdaDatatype{Vec} está parametrizado por un tipo \AgdaArgument{A} e indexado sobre los números naturales. 

En el tipo del constructor \AgdaFunction{\_::\_} se puede observar un ejemplo de una función dependiente. El primer argumento del constructor es un número natural \AgdaArgument{n} implícito, el cual es el largo de la cola. Es seguro poner \AgdaArgument{n} como argumento implícito ya que el verificador de tipos siempre podrá inferirlo en base al tipo del tercer argumento. 

Lo que tienen de interesante las familias de tipos es lo que sucede cuando se usa \textit{pattern matching} sobre sus elementos. Si se quisiera definir una función que devuelva la cabeza de una lista no vacía, el tipo \AgdaDatatype{Vec} permite expresar el tipo de las listas no vacías, lo cual hace posible definir la función \AgdaFunction{head} de manera segura como se muestra a continuación: 

\ExecuteMetaData[latex/Agda.tex]{head}

La definición es aceptada por el verificador de tipos ya que, aunque no se da un caso para la lista vacía, es exhaustiva. Esto es gracias a que un elemento del tipo \mbox{\AgdaDatatype{Vec} \AgdaArgument{A} \AgdaSymbol{(}\AgdaFunction{suc} \AgdaArgument{n}\AgdaSymbol{)}} sólo puede ser construído por el constructor \AgdaFunction{\_::\_}, y resulta útil ya que la función \AgdaFunction{head} no está correctamente definida para el caso de la lista vacía.  

\section{Sistema de Módulos}\label{agda:modules}

El objetivo del sistema de módulos de Agda es manejar el espacio de nombres. Un programa se estructura en diversos archivos, cada uno de los cuales tiene un módulo \textit{top-level}, dentro del cual van todas las definiciones. El nombre del módulo principal de un archivo debe coincidir con el nombre de dicho archivo. Si se tiene, por ejemplo, un archivo llamado \texttt{Agda.agda}, al comienzo del archivo se debería encontrar la siguiente línea:

\ExecuteMetaData[latex/Agda.tex]{module} 

Dentro del módulo principal se pueden definir otros módulos. Esto se hace de la misma manera que se define el módulo \textit{top-level}. Por ejemplo:

\ExecuteMetaData[latex/Agda.tex]{moduleNumbers}

Para acceder a entidades definidas en otro módulo hay que anteponer al nombre de la entidad el nombre del módulo en el cual está definida. Para hacer referencia a \AgdaDatatype{Nat} desde fuera del módulo \AgdaModule{Numbers} se debe escribir \AgdaModule{Numbers.Nat}:


\ExecuteMetaData[latex/Agda.tex]{one}


La extensión de los módulos (excepto el módulo principal) se determina por indentación. Si se quiere hacer referencia a las definiciones de un módulo sin anteponer el nombre del módulo constantemente se puede utilizar la sentencia \AgdaKeyword{open}, tanto localmente como en \textit{top-level}:


\ExecuteMetaData[latex/Agda.tex]{two}


\ExecuteMetaData[latex/Agda.tex]{two2}


Al abrir un módulo, se puede controlar qué definiciones se muestran y cuáles no, así como también cambiar el nombre de algunas de ellas. Para esto se utilizan las palabras clave \AgdaKeyword{using} (para restringir cuáles definiciones traer), \AgdaKeyword{hiding} (para esconder ciertas definiciones) y \AgdaKeyword{renaming} (para cambiarles el nombre). Si se quisiera abrir el módulo \AgdaModule{Numbers} ocultando la función \AgdaFunction{suc₂} y cambiando los nombres del tipo y los constructores, se debería escribir:

\ExecuteMetaData[latex/Agda.tex]{hidingrenaming}

\subsection{Módulos Parametrizados}\label{modules:parametrized}

Los módulos pueden ser parametrizados por cualquier tipo de dato. En caso de que se quiera definir un módulo para ordenar listas, por ejemplo, puede ser conveniente asumir que las listas son de tipo $A$ y que tenemos una relación de orden sobre $A$. A continuación se presenta dicho ejemplo: 


\ExecuteMetaData[latex/Agda.tex]{sort}


Cuando se mira desde afuera una función definida dentro de un módulo parametrizado, la función toma como argumentos, además de los propios, los parámetros del módulo. De esta manera se podría definir:


\ExecuteMetaData[latex/Agda.tex]{sort1}


También pueden aplicarse todas las funciones de un módulo parametrizado a los parámetros del módulo de una vez instanciando el módulo de la siguiente manera: 


\ExecuteMetaData[latex/Agda.tex]{sortnat}


Esto crea el módulo \AgdaModule{SortNat} que contiene las funciones \AgdaFunction{insert} y \AgdaFunction{sort}, las cuales ya no tienen como argumentos los parámetros del módulo \AgdaModule{Sort}, sino que directamente trabajan con naturales y la relación sobre naturales \AgdaFunction{<}. 


\ExecuteMetaData[latex/Agda.tex]{sort2}


También se puede instanciar el módulo y abrir directamente el módulo resultante sin darle un nuevo nombre, lo cual se escribe de forma simplificada como sigue:


\ExecuteMetaData[latex/Agda.tex]{opensortnat}


\subsection{Importando módulos desde otros archivos}\label{modules:files}

Se describió hasta ahora la forma de utilizar diferentes módulos dentro de un archivo, el cual tiene siempre un módulo principal. Muchas veces, sin embargo, los programas se dividen en diversos archivos y uno se ve en la necesidad de utilizar un módulo definido en un archivo diferente al actual. Cuando esto sucede, se debe \textit{importar} el módulo correspondiente.  

Los módulos se importan por nombre. Si se tiene un módulo \AgdaModule{A.B.C} en un archivo \texttt{/alguna/direccion/local/A/B/C.agda}, este se importa con la sentencia \AgdaKeyword{import} \AgdaModule{A.B.C}. Para que el sistema pueda encontrar el archivo, \texttt{/alguna/direccion/local} debe estar en el \textit{path} de búsqueda de Agda. 

Al importar módulos se pueden utilizar las mismas palabras claves de control de espacio de nombres que al abrir un módulo (\AgdaKeyword{using}, \AgdaKeyword{hiding} y \AgdaKeyword{renaming}). Importar un módulo, sin embargo, no lo abre automáticamente. Se puede abrir de forma separada con una sentencia \AgdaKeyword{open} o usar la forma corta \AgdaKeyword{open import} \AgdaModule{A.B.C}.

\section{Records}\label{agda:records}

Un tipo \textit{record} se define de forma similar a un tipo \textit{data}, donde en lugar de constructores se tienen campos, los cuales son provistos por la palabra clave \AgdaKeyword{field}. Por ejemplo: 

\ExecuteMetaData[latex/Agda.tex]{point}

Esto declara el registro \AgdaRecord{Point} con dos campos naturales \AgdaArgument{x} e \AgdaArgument{y}. Para construir un elemento de \AgdaRecord{Point} se escribe:

\ExecuteMetaData[latex/Agda.tex]{mkpoint}

Si antes de la palabra clave \AgdaKeyword{field} se agrega la palabra clave \AgdaKeyword{constructor}, se puede dar un constructor específico para el registro, el cual permite construir de manera simplificada un elemento del mismo. 

\ExecuteMetaData[latex/Agda.tex]{point2}

Para poder extraer los campos de un \textit{record}, cada tipo \textit{record} viene con un módulo con el mismo nombre. Este módulo está parametrizado por un elemento del tipo y contiene funciones de proyección para cada uno de los campos. En el ejemplo de \AgdaRecord{Point} se obtiene el siguiente módulo:

\ExecuteMetaData[latex/Agda.tex]{modpoint}

Este módulo puede utilizarse como viene o puede instanciarse a un registro en particular. 

\ExecuteMetaData[latex/Agda.tex]{get}

Es posible agregar funciones al módulo de un \textit{record} incluyéndolas en la declaración del mismo luego de los campos. 

\ExecuteMetaData[latex/Agda.tex]{monad}

Como se puede ver en este ejemplo, los \textit{records} pueden ser, al igual que los \textit{datatype}, parametrizados. En este caso, el \textit{record} \AgdaRecord{Monad} está parametrizado por \AgdaArgument{M}. Cuando un \textit{record} está parametrizado, el módulo generado por él tiene los parámetros del \textit{record} como parámetros implícitos.

\section{Características adicionales}\label{agda:additional}

Agda tiene soporte para definiciones mutuamente recursivas. Estas deben ser declaradas dentro de un bloque \AgdaKeyword{mutual}:

\ExecuteMetaData[latex/Agda.tex]{mutual}

\section{Coinducción}\label{agda:coinduction}

