\chapter{M\'onadas Concurrentes}\label{chapter:monconc}

En este capítulo se hará una introducción teórica sobre mónadas concurrentes. Se definirán al comienzo algunos conceptos previos de la teoría de categorías sobre la cual se definen las mónadas. En la segunda sección se hará una introducción a las mónadas en sí mismas, en la cual se darán varias definiciones y ejemplos. Por último, en la tercera, se introducirán los conceptos específicos necesarios para definir una mónada concurrente. 

\section{Teoría de Categorías}\label{monconc:cat}

La teoría de categorías fue desarrollada por Eilenberg y MacLane \cite{eilenberg:1945} en 1945. Esta teoría busca axiomatizar de forma abstracta diversas estructuras matemáticas como una sola, tales como los grupos y los espacios topológicos, mediante el uso de objetos y morfismos. A su vez, esta axiomatización se realiza de una manera nueva sin incluir las nociones de elemento o pertenencia, es decir, sin utilizar conjuntos. Con el concepto de categoría se pretende capturar la esencia de una clase de objetos matemáticos que se relacionan entre sí mediante aplicaciones, poniendo énfasis en la relación entre los objetos y no en la pertenencia como en la teoría de conjuntos.

\begin{definition}[Categoría]
Una \textbf{categoría} $\mathscr{C}$ consiste de:
\begin{itemize}[noitemsep,label=$\blacktriangleright$]
	\item una clase de \textbf{objetos}: $\mathbf{ob} \ \mathscr{C}$;
	\item una clase de \textbf{morfismos} o \textbf{flechas}: $\mathbf{mor} \ \mathscr{C}$;
	\item dos funciones de clase:
	\begin{itemize}[noitemsep,label=$\bullet$]
		\item $dom : \mathbf{mor} \ \mathscr{C} \rightarrow \mathbf{ob} \ \mathscr{C}$ (dominio),
		\item $codom : \mathbf{mor} \ \mathscr{C} \rightarrow \mathbf{ob} \ \mathscr{C}$ (codominio).
	\end{itemize}
	Para cada par de objetos $A, B$ en $\mathbf{ob} \ \mathscr{C}$ se denomina $Hom(A,B)$ al conjunto de flechas o morfismos de $A$ a $B$, es decir:
	\begin{equation*}
		Hom(A,B) := \{f \in \mathbf{mor} \ \mathscr{C} : dom(f) = A, codom(f) = B\}
	\end{equation*}
	\item Y para cada $A, B, C \in \mathbf{ob} \ \mathscr{C}$ una operación  
	\begin{equation*}
		\circ : Hom(A,B) \times Hom(B,C) \rightarrow Hom(A,C)
	\end{equation*}
	llamada \textbf{composición} con las siguientes propiedades: 
	\begin{itemize}[noitemsep,label=$\bullet$]
		\item Se denota $\circ(f,g) = g \circ f$.
		\item \textbf{Asociatividad}: para cada $A,B,C,D \in \mathbf{ob} \ \mathscr{C}$ y $f,g,h \in \mathbf{mor} \ \mathscr{C}$ tales que $f \in Hom(A,B)$, $g \in Hom(B,C)$ y $h \in Hom(C,D)$, \ \ $h \circ (g \circ f) = (h \circ g) \circ f$.
		\item Para cada $A \in \mathbf{ob} \ \mathscr{C}$ existe un \textbf{morfismo identidad} $id_A \in Hom(A,A)$ tal que
		\begin{itemize}[noitemsep,label=$\star$]
			\item $\forall B, \ \forall f \in Hom(A,B), \ f \circ id_A = f$,
			\item $\forall C, \ \forall g \in Hom(C,A), \ id_A \circ g = g$.
		\end{itemize}
	\end{itemize}
	
\end{itemize}
\end{definition}

A continuación se presentan algunos ejemplos de categorías que pueden ser de utilidad para comprender mejor el concepto.

\begin{ejemplo}[Categoría \textbf{Set}]
La categoría \textbf{Set} es aquella tal que:
\begin{itemize}[noitemsep,label=$\blacktriangleright$]
	\item $\text{\bf ob Set} = \text{conjuntos}$
	\item $\text{\bf mor Set} = \text{funciones}$. 
\end{itemize}
\end{ejemplo}
\begin{ejemplo}[Categor\'ia $\mathds{1}$]
La categoría $\mathds{1}$ es aquella tal que:
\begin{itemize}[noitemsep,label=$\blacktriangleright$]
	\item $\text{\bf ob} \ \mathds{1} = \{\star\}$
	\item $\text{\bf mor} \ \mathds{1} = \{\text{id}_{\star}\}$. 
\end{itemize}
\end{ejemplo}

Dentro de los objetos de una categoría, hay dos clases especiales de objetos: iniciales y terminales. Estos se definen como sigue:

\begin{definition}[Objetos iniciales y terminales]
Un objeto $\mathbf{0} \in \text{\bf ob} \ \mathscr{C}$ se dice \textbf{inicial} si $\forall A \in \text{\bf ob} \ \mathscr{C}$, $\exists! \mathbf{0} \rightarrow A$. 
Un objeto $\mathbf{1} \in \text{\bf ob} \ \mathscr{C}$ se dice \textbf{terminal} si $\forall A \in \text{\bf ob} \ \mathscr{C}$, $\exists! A \rightarrow \mathbf{1}$.
\end{definition}

\begin{ejemplo} 
En \textbf{Set}, $\emptyset$ es el único objeto inicial y los conjuntos de un elemento $\{x\}$ son los objetos terminales.
\end{ejemplo}

En la categoría \textbf{Set}, se sabe que el producto cartesiano entre dos objetos (conjuntos) $A \times B$ es el conjunto de los pares $(a,b)$ tales que $a \in A$ y $b \in B$. Para definir el concepto de producto cartesiano entre dos objetos $A$ y $B$ de una categoría cualquiera, es necesario caracterizar a $A \times B$ sin hacer referencia a sus elementos. 

\begin{definition}[Producto]
El \textbf{producto} de dos objetos $A$ y $B$ en una categoría $\mathscr{C}$ es una terna $(A \times B, \pi_A, \pi_B)$ donde:
\begin{itemize}[label=$\blacktriangleright$]
	\item $\pi_A \in Hom(A \times B, A)$,
	\item $\pi_B \in Hom(A \times B, B)$
	\item y para todo objeto $C$ y para todo par de morfismos $f : C \rightarrow A$, $g : C \rightarrow B$, existe un único mofismo $\langle f, g \rangle : C \rightarrow A \times B$ tal que:
	\begin{itemize}[label=$\bullet$]
		\item $f = \pi_A \circ \langle f, g \rangle$
		\item $g = \pi_B \circ \langle f, g \rangle$
	\end{itemize}
\end{itemize}
\end{definition}

Suponiendo que existen los productos $A \times B$ y $C \times D$ y que se tienen dos morfismos $f : A \rightarrow C$ y $g : B \rightarrow D$, se puede definir un morfismo $f \times g : A \times B \rightarrow C \times D$ tal que $f \times g = \langle f \circ \pi_A , g \circ \pi_B \rangle$.

De forma dual a la definición de producto, se puede definir la noción de coproducto entre dos objetos $A$ y $B$ de una categoría arbitraria como sigue:

\begin{definition}[Coproducto]
El \textbf{coproducto} de dos objetos $A, B$ de una categoría $\mathscr{C}$ es una terna $(A + B, \iota_A, \iota_B)$ donde:
\begin{itemize}[label=$\blacktriangleright$]
	\item $\iota_A \in Hom(A, A + B)$,
	\item $\iota_B \in Hom(B, A + B)$
	\item y para todo objeto $C$ y para todo par de morfismos $f : A \rightarrow C$, $g : B \rightarrow C$ existe un único morfismo $[f,g] : A + B \rightarrow C$ tal que se cumplen las siguientes ecuaciones:
	\begin{itemize}[label=$\bullet$]
		\item $f = [f,g] \circ \iota_A$
		\item $g = [f,g] \circ \iota_B$
	\end{itemize}
\end{itemize}
\end{definition}

Una última relación que puede establecerse entre dos objetos de una categoría es el exponencial. En \textbf{Set}, el exponencial de dos conjuntos $A$ y $B$ es el conjunto $B^A$ de todas las funciones que van de $A$ en $B$, es decir que toman un elemento de $A$ y devuelven un elemento de $B$. Esta noción puede generalizarse a una categoría arbitraria que tenga productos cartesianos.

\begin{definition}[Exponencial]
Sea $\mathscr{C}$ una categoría con productos binarios y sean $A,B \in \text{\bf ob} \ \mathscr{C}$. Un objeto $B^A$ es un \textbf{exponencial} si existe un morfismo $\varepsilon : B^A \times A \rightarrow B$ tal que para todo morfismo $g : C \times A \rightarrow B$ existe un único morfismo $\tilde{g} : C \rightarrow B^A$ tal que $g = \varepsilon \circ (\tilde{g} \times id_A)$.
\end{definition}

Si se quisiera construir una categoría cuyos objetos son categorías, se necesitaría contar con morfismos entre categorías. Estos existen y se llaman funtores, son en cierta manera una generalización del concepto de función de conjuntos para categorías. Un funtor permite construir una nueva categoría a partir de otra dada.
 
\begin{definition}[Funtor]\label{def:funtor}
Sean $\mathscr{C}$ y $\mathscr{D}$ dos categorías. Un \textbf{funtor} $\mathit{F} : \mathscr{C} \rightarrow \mathscr{D}$ asigna:
\begin{itemize}[noitemsep,label=$\blacktriangleright$]
	\item a cada objeto $A \in \mathbf{ob} \ \mathscr{C}$, un objeto $\mathit{F}(A) \in \mathbf{ob} \ \mathscr{D}$;
	\item a cada morfismo $f : A \rightarrow B \in \mathbf{mor} \ \mathscr{C}$, un morfismo $\mathit{F}(f) : \mathit{F}(A) \rightarrow \mathit{F}(B) \in \mathbf{mor} \ \mathscr{D}$ tal que: 
	\begin{itemize}[noitemsep,label=$\bullet$]
		\item para todo $A \in \mathbf{ob} \ \mathscr{C}$, $\mathit{F}(id_A) = id_{\mathit{F}(A)}$;
		\item para todos $f,g \in \mathbf{mor} \ \mathscr{C}$ tales que tenga sentido la composición $g \circ f$, se tiene que $\mathit{F}(g \circ f) = \mathit{F}(g) \circ \mathit{F}(f)$.
	\end{itemize}
\end{itemize}
\end{definition}

Se dice que un funtor es un \textbf{endofuntor} si la categoría de salida y la de llegada son la misma, es decir, $\mathit{F} : \mathscr{C} \rightarrow \mathscr{C}$.

Siguiendo con la misma lógica, uno podría construir morfismos entre funtores. Es decir, algún tipo de construcción matemática que lleve de un funtor dado a otro. Este concepto se denomina transformación natural y se define como sigue:

\begin{definition}[Transformaci\'on Natural]
Sean $\mathit{F}, \mathit{G} : \mathscr{C} \rightarrow \mathscr{D}$ dos funtores (entre las mismas categorías). Una \textbf{transformación natural} $\eta : \mathit{F} \rightarrow \mathit{G}$ asigna a cada $A \in \mathbf{ob} \ \mathscr{C}$ un morfismo $\eta_A : \mathit{F}(A) \rightarrow \mathit{G}(A)$ tal que para todo $f \in Hom(A,B)$ se cumple que: 
\begin{equation*}
	\eta_B \circ \mathit{F}(f) = \mathit{G}(f) \circ \eta_A 
\end{equation*}
\end{definition}

Para cerrar la sección, se introducirá la noción de monoide, Los monoides son un tipo de estructura algebraica abstracta introducida por primera vez por Arthur Cayley.

\begin{definition}[Monoide]\label{def:monoid}
Un \textbf{monoide} es un conjunto $M$ dotado de una operación asociativa $M \times M \rightarrow M$, $(m,n) \rightarrow mn$ tal que existe un elemento neutro:
\begin{equation*}
\exists e \in M, \forall m \in M, (em = me = m).
\end{equation*}
\end{definition}

El elemento neutro de un monoide es único. Por esa razón, en general el elemento neutro es considerado una constante, es decir, una operación 0-aria (sin argumentos). Se utilizará esta representación en la formalización de los monoides.

\section{Introducción a las mónadas}\label{monconc:monadas}

Se considerarán dos variantes de la definición de mónadas. La primera es la definición clásica y la segunda define a una mónada como un sistema de extensión o 3-tupla Kleisli. La primera es muy utilizada en la literatura ya que es la definición matemática y está definida en torno a transformaciones naturales, pero la segunda es más fácil de utilizar desde una perspectiva computacional. Como ambas definiciones son equivalentes \cite{moggi:1991}, se utilizará una u otra según sea conveniente.

\subsection{Definición clásica de Mónadas}\label{monadas:usual}
Se define a continuación el concepto de mónada de la manera clásica dentro de la teoría de categorías.

\begin{definition}[Mónada]
Dada una categoría $\mathscr{C}$, una \textbf{mónada} sobre $\mathscr{C}$ es una tupla $(\mathit{T},\mu,\eta)$, donde:
\begin{itemize}[noitemsep,label=$\blacktriangleright$]
	\item $\mathit{T} : \mathscr{C} \rightarrow \mathscr{C}$ es un funtor,
	\item $\eta : \mathit{Id} \rightarrow \mathit{T}$ y $\mu : \mathit{T} \cdot \mathit{T} \rightarrow \mathit{T}$ son transformaciones naturales
	\item y se cumplen las siguientes identidades:
	\begin{equation*}
		\mu_X \circ \mathit{T}\mu_X = \mu_X \circ \mu_{\mathit{T}X} \text{, } \qquad \mu_X \circ \mathit{T}\eta_X = id_{\mathit{T}X} \text{, } \qquad 
		\mu_X \circ \eta_{\mathit{T}X} = id_{\mathit{T}X} 
	\end{equation*}
\end{itemize}
\end{definition}

A continuación se presentan algunos ejemplos de mónadas clásicas que son ampliamente utilizadas en computación.

\begin{ejemplo}[M\'onada \textit{Error}]
Sea $\mathit{T} : \text{\bf Set} \rightarrow \text{\bf Set}$ el funtor $\mathit{T} X = X + E$, donde $E$ es un conjunto de errores. Intuitivamente un elemento de $\mathit{T} X$ puede ser un elemento de $X$ (un valor) o un error pertenenciente a $E$. Luego se definen $\eta$ y $\mu$ como siguen:
\begin{itemize}[noitemsep, label=$\blacktriangleright$]
	\item Para cada conjunto $X$, se define $\eta_X : \mathit{Id} X \rightarrow \mathit{T} X$ como $\eta_X (x) = inl(x)$.
	\item Para cada conjunto $X$, se define $\mu_X : \mathit{T T} X \rightarrow \mathit{T} X$ como $\mu_X (inl(tx)) = tx$ si $tx \in X + E$ y $\mu_X (inr(e)) = inr(e)$ si $e \in E$. Es decir que si se tiene un error se propaga el error y si se tiene un elemento de $\mathit{T} X$ se devuelve dicho elemento. 
\end{itemize}
\end{ejemplo}

\begin{ejemplo}[M\'onada \textit{State}]
Sea $\mathit{T} : \text{\bf Set} \rightarrow \text{\bf Set}$ el funtor $\mathit{T} X = (X \times S)^S$, donde $S$ es un conjunto no vacío de estados. Intuitivamente, $\mathit{T} X$ es una computación que toma un estado y retorna el valor resultante junto con el estado modificado. Luego se definen $\eta$ y $\mu$ como sigue:
\begin{itemize}[noitemsep, label=$\blacktriangleright$]
	\item Para cada conjunto $X$, se define $\eta_X : \mathit{Id} X \rightarrow \mathit{T} X$ como $\eta_X (x) = (\lambda s : S . \langle x , s \rangle)$.
	\item Para cada conjunto $X$, se define $\mu_X : \mathit{T T} X \rightarrow \mathit{T} X$ como \\ \mbox{$\mu_X (f) = (\lambda s : S .$ let $\langle f' , s' \rangle = f(s)$ in $f'(s'))$}, es decir que $\mu_X (f)$ es la computación que, dado un estado $s$, primero computa el par computación-estado $f(s) = \langle f' , s' \rangle$ y luego retorna el par valor-estado $f'(s') = \langle x , s'' \rangle$.   
\end{itemize}
\end{ejemplo}

\subsection{Definición alternativa de Mónadas}\label{monadas:alt}

Se define ahora la noción de sistema de extensión, también llamado 3-tupla Kleisli. Esta definición también parte de la teoría de categorías pero no es la más utilizada en la literatura. Sin embargo, como se explica más adelante, es la que más se acerca a la forma de utilizar las mónadas en los lenguajes de programación funcional. 

\begin{definition}[Sistema de extensión]\label{def:sistext}
Un \textbf{sistema de extensión} sobre una categoría $\mathscr{C}$ es una tupla $(\mathit{T},\eta,\_^*)$, donde 
\begin{itemize}[noitemsep,label=$\blacktriangleright$]
	\item $\mathit{T} : \mathbf{ob} \ \mathscr{C} \rightarrow \mathbf{ob} \ \mathscr{C}$,
	\item para cada $A \in \mathbf{ob} \ \mathscr{C}$, $\eta_A : A \rightarrow \mathit{T}A$,
	\item para cada $f : A \rightarrow \mathit{T}B$,  $f^* : \mathit{T}A \rightarrow \mathit{T}B$,
	\item y se cumplen las siguientes ecuaciones:
	\begin{itemize}[noitemsep,label=$\bullet$]
		\item $\eta^*_A = id_{\mathit{T}A}$
		\item $f^* \circ \eta_A = f$ para cada $f : A \rightarrow \mathit{T}B$
		\item $g^* \circ f^* = (g^* \circ f)^*$ para cada $f : A \rightarrow \mathit{T}B$ y $g : B \rightarrow \mathit{T}C$.
	\end{itemize}
\end{itemize}
\end{definition}

Intuitivamente $\eta_A$ es la inclusión de valores en computaciones (lo que en programación funcional usualmente se conoce como \textit{return}) y $f^*$ es la extensión de una función $f$ que va de valores a computaciones a una función que va de computaciones a computaciones, la cual primero evalúa una computación y luego aplica $f$ al valor resultante (lo que generalmente se conoce como \textit{bind} o $>$\hspace{-1mm}$>$\hspace{-1mm}$=$).

A continuación se muestra cómo quedan definidos los ejemplos vistos para la definición clásica como sistemas de extensión para que se comprenda mejor el paralelismo entre ambas definiciones. 

\begin{ejemplo}[M\'onada \textit{Error}] 
Tomando el funtor descripto en la versión clásica:
\begin{itemize}[noitemsep, label=$\blacktriangleright$]
	\item Para cada conjunto $X$, se define $\eta_X : \mathit{Id} X \rightarrow \mathit{T} X$ como $\eta_X (x) = inl(x)$ al igual que en la versión clásica.
	\item Para cada función $f : X \rightarrow \mathit{T}Y$, se define $f^* : \mathit{T} X \rightarrow \mathit{T} Y$ como \mbox{$f^*(inl(x)) = f(x)$} si $x \in X$ y $f^*(inr(e)) = inr(e)$ si $e \in E$.   
\end{itemize}
\end{ejemplo}

\begin{ejemplo}[M\'onada \textit{State}] 
Tomando el funtor descripto en la versión clásica:
\begin{itemize}[noitemsep, label=$\blacktriangleright$]
	\item Para cada conjunto $X$, se define $\eta_X : \mathit{Id} X \rightarrow \mathit{T} X$ como $\eta_X (x) = (\lambda s : S . \langle x , s \rangle)$ al igual que en la primera versión.
	\item Para cada función $f : X \rightarrow \mathit{T}Y$, se define $f^* : \mathit{T} X \rightarrow \mathit{T} Y$ como \\ \mbox{$f^*(g) = (\lambda s : S .$ let $\langle x , s' \rangle = g(s)$ in $f(x)(s'))$}.
\end{itemize}
\end{ejemplo}

\subsection{Funtores, mónadas y producto cartesiano}\label{monadas:cartesian}

La fortaleza de los funtores es una forma de compatibilidad entre funtores y productos. En adelante se trabajará con funtores y mónadas que son fuertes respecto del producto cartesiano. A continuación se definen las nociones de funtor fuerte y mónada fuerte.

\begin{definition}[Funtor fuerte]
Un \textbf{funtor} $\mathit{F} : \mathscr{C} \rightarrow \mathscr{C}$  es \textbf{fuerte} si viene equipado con una transformación natural $\sigma_{X,Y} : \mathit{F}X \times Y \rightarrow \mathit{F} (X \times Y)$, de manera que se cumplen las siguientes ecuaciones:
\begin{equation*}
	\pi_1 = \mathit{F}(\pi_1) \circ \sigma_{X,\mathbf{1}} \text{, \quad} \sigma \circ (\sigma \times id) \circ \alpha = \mathit{F}(\alpha) \circ \sigma 
\end{equation*} 
donde $\pi_1$ y $\pi_2$ son las proyecciones del producto cartesiano y $\alpha = \langle \langle \pi_1 , \pi_1 \circ \pi_2 \rangle , \pi_2 \circ \pi_2 \rangle$ representa su asociatividad.
\end{definition}

\begin{definition}[Mónada fuerte]
Una \textbf{mónada} $(\mathit{T},\mu,\eta)$ sobre $\mathscr{C}$ es \textbf{fuerte} si el funtor subyacente $\mathit{T}$ es fuerte y la fortaleza es compatible con $\mu$ y $\eta$: 
\begin{equation*}
\eta_{A \times B} = \sigma_{A,B} \circ (\eta_A \times \text{id}) \text{, \quad} \sigma_{A,B} \circ (\mu_A \times \text{id}) = \mu_{A \times B} \circ \mathit{T}\sigma_{A,B} \circ \sigma_{\mathit{T}A,B}
\end{equation*}
\end{definition}

Hay una definición similar de fortaleza $\bar{\sigma}_{X,Y} : X \times \mathit{F}Y \rightarrow \mathit{F} (X \times Y)$ que actúa sobre el lado derecho, pero como el producto cartesiano es simétrico, se puede obtener de la fortaleza izquierda como $\bar{\sigma} = \mathit{F} \gamma \circ \sigma \circ \gamma$, donde $\gamma = \langle \pi_2 , \pi_1 \rangle$ intercambia los elementos del producto cartesiano.

%\subsection{Estructuras monoidales}\label{monadas:monoidal}
Otra forma en la que un funtor puede ser compatible con el producto cartesiano es si es un funtor monoidal. A continuación se definen los conceptos de funtor monoidal y mónada monoidal.

\begin{definition}[Funtor monoidal]\label{def:monoidalfuntor}
Un \textbf{funtor monoidal} es un funtor $\mathit{F}: \mathscr{C} \rightarrow \mathscr{C}$ equipado con una estructura monoidal $(m,e)$, donde $m : \mathit{F} X \times \mathit{F} Y \rightarrow \mathit{F} (X \times Y)$ es una transformación natural y $e : \mathbf
{1} \rightarrow \mathit{F} \mathbf{1}$ es un morfismo tal que las siguientes ecuaciones se cumplen:
\begin{equation*}
\pi_1 = \mathit{F}(\pi_1) \circ m_{A,\mathbf{1}} \circ (id_{\mathit{F}A} \times e) \text{, \qquad} \pi_2 = \mathit{F}(\pi_2) \circ m_{\mathbf{1},A} \circ (e \times id_{\mathit{F}A}) \text{,}
\end{equation*}
\begin{equation*}
\mathit{F}(\alpha) \circ m_{X \times Y, Z} \circ (m_{X,Y} \times id_{\mathit{F}Z}) = m_{X, Y \times Z} \circ (id_{\mathit{F}X} \times m_{Y,Z}) \circ \alpha
\end{equation*}
Además, si la estructura monoidal es compatible con $\gamma$, entonces el funtor monoidal es simétrico.
\end{definition}

\begin{definition}[Mónada monoidal]
Una \textbf{mónada monoidal} es una mónada $(\textit{T},\mu,\eta)$ que tiene una estructura monoidal $(m,e)$ en su funtor subyacente $\mathit{T}$ tal que $e = \eta_{\mathbf{1}}$ y las estructuras monoidal y monádica son compatibles:
\begin{equation*}
\eta_{A \times B} = m_{A,B} \circ (\eta_A \times \eta_B) \text{, \quad} m_{A,B} \circ (\mu_A \times \mu_B) = \mu_{A \times B} \circ \mathit{T}m_{A,B} \circ m_{\mathit{T}A \times \mathit{T}B}.
\end{equation*}
\end{definition}

%\subsection{Mónadas conmutativas}\label{monads:commutative}
Dada una mónada fuerte $(\mathit{T},\mu,\eta)$, $\mathit{T}$ como funtor puede ser equipado con dos estructuras monoidales canóncas:
\begin{align*}
&\phi : \mathit{T}A \times \mathit{T}B \rightarrow \mathit{T} (A \times B) & &\psi : \mathit{T}A \times \mathit{T}B \rightarrow \mathit{T}(A \times B) \\
&\phi = \mu \circ \mathit{T}\bar{\sigma} \circ \sigma & &\psi = \mu \circ \mathit{T}\sigma \circ \bar{\sigma}
\end{align*}
y $e = \eta_1 : \mathbf{1} \rightarrow \mathit{T}\mathbf{1}$ en ambos casos. 

Se dice que una mónada es \textbf{conmutativa} cuando estas dos estructuras coinciden.

En esta tesina se utilizará la categoría \textbf{Set}, la cual es la categoría de conjuntos y funciones, y las mónadas que se presenten serán sobre esta categoría. El objeto terminal $\mathbf{1} = \{\star\}$ es un conjunto unitario. Una consecuencia particular de esto es que cualquier funtor $\mathit{F}$ y mónada $(\mathit{T},\mu,\eta)$ sobre esta categoría son fuertes, y cada uno admite una única fortaleza posible $\sigma$ ($\bar{\sigma}$). 

Por ejemplo, la mónada del conjunto partes $\mathcal{P}$ (y su variante finita $\mathcal{P}_f$) tiene fortaleza $\sigma(X,y) = X \times \{y\}$. En general, la fórmula de fortaleza de un funtor sobre \textbf{Set} puede ser expresada como $\sigma(v,y) = \mathit{F}(\lambda x : X . (x,y))(v)$. Cuando la mónada es conmutativa, hay sólo una estructura monoidal posible. En consecuencia, si una mónada es monoidal entonces es conmutativa \cite{kock:1970}. 

\section{Mónadas Concurrentes}\label{monconc:mc}

La teoría de concurrencia está compuesta por una amplia variedad de modelos basados en diferentes conceptos. Hoare et al. \cite{hoare:2011} se plantearon si es posible tener un tratamiento comprensible de la concurrencia en el cual la memoria compartida, el pasaje de mensajes y los modelos de intercalación e independencia de computaciones puedan ser vistos como parte de la misma teoría con el mismo núcleo de axiomas. Con esta motivación crearon un modelo simple de concurrencia basado en estructuras algebraicas, dos de las cuales resultan interesantes para este trabajo: bimonoides ordenados y monoides concurrentes. Más tarde, Rivas y Jaskelioff \cite{rivas:2019} extendieron este modelo al nivel de funtores y mónadas, dando lugar a las mónadas concurrentes. En las siguientes secciones se detallarán las características principales de cada uno de estos modelos.

\subsection{Ley de intercambio}\label{mc:interchange}

Ya estaba establecido que la composición secuencial y concurrente son estructuras monoidales, donde la concurrencia es además conmutativa. La pregunta que surge luego es cómo estas operaciones se relacionan entre sí. Se podría pensar en un principio que la ley de intercambio $(p * r) ; (q * s) = (p ; q) * (r ; s)$ de 2-categorías o bi-categorías debería cumplirse.  Sin embargo, la presencia de esta ley implicaría que ambas estructuras monoidales coinciden, derivando en que las operaciones de secuenciación y concurrencia son la misma. Esto se puede ver aplicando el argumento Eckmann-Hilton. 

\begin{thm}[argumento Eckmann-Hilton] Sea $X$ un conjunto con dos operaciones binarias $;$ y $*$ tal que $e_;$ es el elemento neutro de $;$, $e_*$ es el elemento neutro de $*$ y la ley de intercambio $(a * b) ; (c * d) = (a ; c) * (b ; d)$ se cumple. Entonces, ambas operaciones $;$ y $*$ coinciden, y ambas son conmutativas y asociativas. 
\end{thm}
\begin{proof}
Primero se muestra que ambos elementos neutros coinciden:
\begin{equation*}
e_; = e_; ; e_; = (e_* * e_;);(e_; * e_*) = (e_* ; e_;) * (e_; ; e_*) = e_* * e_* = e_*
\end{equation*}
Como los neutros coinciden, se lo puede llamar simplemente $e$. Se muestra ahora que ambas operaciones coinciden:
\begin{equation*}
a ; b = (e * a) ; (b * e) = (e ; b) * (a ; e) = b * a = (b ; e) * (e ; a) = (b * e) ; (e * a) = b ; a
\end{equation*}
Usando el mismo argumento se puede ver también que la operación es conmutativa. La prueba de asociatividad es análoga. 
\end{proof}

Como solución a esto, surge la idea de considerar un orden en los procesos, de manera que pueda debilitarse la ley de intercambio. En \cite{hoare:2009} se introduce una generalización del álgebra de Kleene para programas secuenciales \cite{kozen:1994}, llamada Álgebra Concurrente de Kleene. Esta es un álgebra que mezcla primitivas de composición concurrente ($*$) y secuencial ($;$), cuya característica principal es la presencia de una versión ordenada de la ley de intercambio de 2-categorías o bi-categorías. 
\begin{equation*}
(p * r) ; (q * s) \sqsubseteq (p ; q) * (r ; s)
\end{equation*}

Esta ley intuitivamente tiene sentido, por ejemplo, en un modelo de concurrencia de intercalación. Si se tiene una traza $t = t_1 ; t_2$ donde $t_1$ es una intercalación de dos trazas $t_p$ y $t_r$, y $t_2$ de $t_q$ y $t_s$, entonces $t$ es también una intercalación de $t_p ; t_q$ y $t_r ; t_s$. 

\subsection{Dos modelos}\label{mc:models}
Se introducirán a continuación dos modelos utilizados por Hoare et al. \cite{hoare:2011} para desarrollar su teoría, los cuales servirán de ejemplo en las secciones que siguen.

\paragraph{Modelo de Trazas} Sea $A$ un conjunto. Luego $Trazas_A$ es el conjunto de secuencias finitas de elementos de $A$. El conjunto partes $\mathcal{P}(Trazas_A)$ será el conjunto soporte del modelo. Se tienen las siguientes operaciones binarias sobre $\mathcal{P}(Trazas_A)$:
\begin{enumerate}
	\item $T_1 * T_2$ es el conjunto de las intercalaciones de las trazas de $T_1$ y $T_2$.
	\item $T_1 ; T_2$ es el conjunto de las concatenaciones entre las trazas de $T_1$ y $T_2$.
\end{enumerate}
El conjunto $\{\epsilon\}$ funciona como elemento neutro para ambas operaciones $;$ y $*$, donde $\epsilon$ es la secuencia vacía. El orden está dado por la inclusión de conjuntos. 

\paragraph{Modelo de Recursos} Sea $(\Sigma,\bullet,u)$ un monoide parcial conmutativo, dado por una operación parcial binaria $\bullet$ y elemento neutro $u$. La igualdad significa que ambos lados están definidos y son iguales, o que ninguno está definido. El conjunto partes $\mathcal{P}(\Sigma)$ tiene una estructura de monoide ordenado conmutativo $(*,\text{emp})$ definido por:
\begin{align*}
p * q &= \{\sigma_0 \bullet \sigma_1 | (\sigma_0, \sigma_1) \in dom(\bullet) \land \sigma_0 \in p \land \sigma_1 \in q\} \\
emp &= \{u\}
\end{align*}
El conjunto de funciones monótonas $\mathcal{P}(\Sigma) \rightarrow \mathcal{P}(\Sigma)$ es el conjunto soporte del modelo. Estas funciones representan transformadores de predicados. Las operaciones se definen mediante las siguientes fórmulas, donde $F_i$ itera sobre los transformadores de predicados e $Y_i$ se utiliza para iterar sobre los subconjuntos de $\Sigma$:
\begin{align*}
(F * G) Y &= \bigcup \{FY_1 * GY_2 | Y_1 * Y_2 \subseteq Y\} \\
\mathtt{nothing} \ Y &= Y \cap \text{emp}  \\
(F ; G) Y &= F(G(Y))  \\
\mathtt{skip} \ Y &= Y \\
\end{align*}
La idea es que se comienza con una postcondición $Y$, luego se la separa en dos afirmaciones separadas $Y_1$ e $Y_2$ y se aplica la regla de concurrencia hacia atrás para obtener una precondición $FY_1 * GY_2$ para la composición paralela de $F$ y $G$. Se realilza la unión de todas estas descomposiciones de manera de obtener la precondición más débil posible. 

El orden del modelo está dado por el orden reverso punto a punto, es decir $F \sqsubseteq G$ significa que $\forall X \subseteq \mathcal{P}(\Sigma), FX \supseteq GX$. Según esta definición, el elemento más pequeño es la función $\lambda X.\Sigma$, la cual se corresponde con el transformador de precondición más débil para la divergencia. 

\subsection{Monoides concurrentes}\label{mc:monoid}

Como se va a utilizar un orden combinado con estructuras alegbraicas, se necesita una noción de compatibilidad de las operaciones con el orden.
Sea $(A, \sqsubseteq)$ un orden parcial, entonces una operación $\oplus : A \times A \rightarrow A$ es \textit{compatible} con el orden si $a \sqsubseteq b$ y $c \sqsubseteq d$ implica que $a \oplus c \sqsubseteq b \oplus d$. 
Se define primero una aproximación a la noción de monoide concurrente, el cual tiene dos estructuras monoidales y un orden compatible con ellas, pero no incluye ninguna relación especial entre ellas.  

\begin{definition}[Bimonoide ordenado]\label{def:bimonord}
Un \textbf{bimonoide ordenado} es un conjunto parcialmente ordenado $(A,\sqsubseteq)$ junto con dos estructuras monoidales $(A,;,\mathtt{skip})$ y $(A,*,\mathtt{nothing})$ tal que $;$ y $*$ son compatibles con $\sqsubseteq$ y $*$ es conmutativa.
\end{definition}

Podría ser tentador requerir que ambos elementos neutros de un bimonoide ordenado sean iguales, pero, por ejemplo, en el Modelo de Recursos no lo son. El Modelo de Recursos es un ejemplo de un bimonoide ordenado que no es un monoide concurrente. A continuación se define la noción de monoide concurrente.

\begin{definition}[Monoide concurrente]\label{def:monoidconc}
Un \textbf{monoide concurrente} es un bimonoide ordenado tal que los neutros coinciden, es decir $\mathtt{nothing} = \mathtt{skip}$, y la siguiente ley de intercambio se cumple:
\begin{equation*}
(a * b) ; (c * d) \sqsubseteq (a ; c) * (b ; d)
\end{equation*}
\end{definition}

En esta estructura no hay reducción de la operación $*$ a una intercalación del operador $;$. El orden une a ambas estructuras sin reducir una a la otra. En el Modelo de Trazas ambos elementos neutros coinciden y la ley de intercambio se cumple, por lo que, además de bimonoide ordenado, es un monoide concurrente. 


\subsection{Generalización a nivel de funtores y mónadas}\label{mc:lifting}

Para elevar la teoría descripta en la sección anterior al nivel de funtores, se modela el operador de secuenciación $;$ como la multiplicación monádica y su elemento neutro $\mathtt{skip}$ como la unidad monádica. Sólo faltan tres elementos: la operación de mezcla $*$, su elemento neutro $\mathtt{nothing}$ y el orden $\sqsubseteq$. La operación de mezcla se modela como una familia de transformaciones naturales $m_{A,B} : \mathit{T} A \times \mathit{T} B \rightarrow \mathit{T} (A \times B)$. Como se necesita que $*$ sea un monoide conmutativo, se requiere que $\mathit{T}$ tenga una estructura de funtor monoidal simétrica, incluyendo un morfismo unidad $e : \mathbf{1} \rightarrow \mathit{T} \mathbf{1}$ que representa $\mathtt{skip}$. 

En cuanto al orden, lo que se necesita es un orden sobre computaciones que mida el grado de secuencialidad en ellas. Esto es una estructura de orden $\sqsubseteq_I$ sobre $\mathit{T} I$ para cada conjunto $I$, sujeto a algún tipo de compatibilidad con el resto de la estructura que modela computaciones. Se define a continuación la noción de funtor ordenado, la cual establece una condición de compatibilidad de un funtor con un orden parcial, siguiendo la idea de Hughes y Jacobs \cite{hughes:2004}:

\begin{definition}[Funtor ordenado]
Un \textbf{orden} para un \textbf{funtor} $\mathit{F}$ es una asignación de un orden parcial $\sqsubseteq_I$ en $\mathit{F} I$ para cada conjunto $I$, tal que para cada morfismo $f : I \rightarrow J$, el morfismo $\mathit{F} f : \mathit{F} I \rightarrow \mathit{F} J$ es una función monótona respecto de $\sqsubseteq_I$ y $\sqsubseteq_J$.
\end{definition}

A continuación se define la compatibilidad para la estructura de mónada. Para esto se utiliza una familia de estructuras ordenadas, lo cual es una instancia de lo que definieron Katsumata y Sato como una mónada ordenada \cite{katsumata:2013}.

\begin{definition}[Mónada ordenada]
Un \textbf{orden} para una \textbf{mónada} $(\mathit{T},\mu,\eta)$ es una asignación de un orden $\sqsubseteq_I$ sobre $\mathit{T} I$ para cada conjunto $I$, tal que la categoría Kleisli de $T$ se enriquece en la categoría de órdenes con el orden correspondiente a $\mathcal{K}\ell(\mathit{T})(A,B)$ definido por $f \sqsubseteq_{A,B} g$ si y sólo si $\forall a : \mathbf{1} \rightarrow A, f \circ a \sqsubseteq_B g \circ a$.
\end{definition}

Esta noción de orden sólo se relaciona con la estructura monádica. En comparación a los bimonoides ordenados, se corresponde con la condición de que $;$ sea compatible con $\sqsubseteq$. La compatibilidad entre $*$ y $\sqsubseteq$ se postula como sigue:

\begin{definition}[Funtor monoidal ordenado]
Sea $\mathit{T}$ un funtor con una estructura monoidal $(m,e)$ y una asignación de un orden $\sqsubseteq_I$ sobre $\mathit{T} I$ para cada conjunto $I$. Se dice que el orden es compatible con $(m,e)$ si $v \sqsubseteq_A v'$ y $w \sqsubseteq_B w'$ implican que $m \circ \langle v,w \rangle \sqsubseteq_{A \times B} m \circ \langle v',w' \rangle$ para cada $v,v' : \mathbf{1} \rightarrow \mathit{T} A$ y $w,w' : \mathbf{1} \rightarrow \mathit{T} B$. 
\end{definition}

A partir de todas estas definiciones previas se definen las variantes monádicas de los bimonoides ordenados y monoides concurrentes como sigue:

\begin{definition}[Mónada monoidal ordenada]\label{def:monmonord}
Una mónada $(\mathit{T},\eta,\_^*)$ es una mónada monoidal ordenada si está dotada de una estructura de funtor monoidal simétrico 
\begin{equation*}
m_{X,Y} : \mathit{T} X \times \mathit{T} Y \rightarrow \mathit{T} (X \times Y) \text{, \qquad} e : \mathbf{1} \rightarrow \mathit{T} \mathbf{1} 
\end{equation*}
y una relación de orden $\sqsubseteq$ sobre $\mathit{T}$ compatible con $(m,e)$.
\end{definition}

\textit{Aclaración: se escribe $f \star g$ para representar $m \circ (f \times g)$ y $f \bullet g$ para denotar $f^* \circ g$.}

\begin{definition}[Mónada concurrente]\label{def:monconc}
Una mónada monoidal ordenada $\mathit{T}$ es una mónada concurrente si $e = \eta_{\mathbf{1}} : \mathbf{1} \rightarrow \mathit{T} \mathbf{1}$ y se cumple la siguiente ley de intercambio:
\begin{equation*}
(h \star i) \bullet (f \star g) \sqsubseteq (h \bullet f) \star (i \bullet g)
\end{equation*}
\end{definition}

Esta definición puede probarse mostrando que las mónadas conmutativas (como los monoides conmutativos) son mónadas concurrentes (como los monoides concurrentes). 

\begin{ejemplo}
Sea $(\mathit{T},\mu,\eta)$ una mónada conmutativa. Entonces $\mathit{T}$ tiene una única estructura de mónada monoidal $(m,\eta_{\mathbf{1}})$, y se puede definir la estructura de orden por el orden diagonal. La ley de intercambio se reduce a las condiciones de mónada monoidal. 
\end{ejemplo}

Rivas y Jaskelioff \cite{rivas:2019} también muestran que estas estructuras al nivel de mónada efectivamente generalizan aquellas al nivel de los monoides probando el siguiente lema.

\begin{lema}
Sea $(\mathit{T},\mu,\eta)$ una mónada monoidal ordenada (mónada concurrente). Entonces $\mathit{T} \mathbf{1}$ es un bimonoide ordenado (monoide concurrente). 
\end{lema}

Como en el caso de los monoides, hay dos estructuras de bimonoide ordenado sobre $\mathit{T} \mathbf{1}$, que resultan de la simetría de los axiomas de los bimonoides ordenados.  En esta estructura, como en los monoides, se puede también ir en la dirección contraria: dado un bimonoide ordenado, este puede ser elevado a una mónada monoidal ordenada.

\begin{lema}
Sea $(A,\sqsubseteq,*,\mathtt{nothing},;,\mathtt{skip})$ un bimonoide ordenado. Este puede convertirse en una mónada monoidal ordenada con funtor soporte $\mathit{T}_A X = A \times X$, operaciones y orden. Más aún, si $\mathtt{nothing} = \mathtt{skip}$ (es decir que es un monoide concurrente), entonces $e = \eta_{\mathbf{1}}$ (es decir que es una mónada concurrente).
\end{lema}

Este resultado será utilizado más adelante en este trabajo para dar una representación alternativa de la mónada delay. 

El ejemplo del Modelo de Trazas puede ser generalizado a una mónada concurrente parametrizando el conjunto sobre el cual se toman las trazas. 

\begin{ejemplo}
La mónada $Tr_L(X) = \mathcal{P}_f(Trazas_{L \times X})$ es concurrente. La estructura de orden se define como la inclusión de conjuntos al igual que antes.
\end{ejemplo}
