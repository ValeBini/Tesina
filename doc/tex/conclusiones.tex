\chapter{Conclusiones y trabajo futuro}

Para realizar esta tesina se estudió en profundidad el concepto de monoides concurrentes y mónadas concurrentes, así como también los tipos coinductivos el uso de la coinducción en general para luego comprender la definición del tipo \textit{delay} y cómo este se utiliza para representar formalmente la no terminación de programas. 

En este trabajo se utilizó el lenguaje y asistente de pruebas Agda para realizar formalizaciones de diversas estructuras algebraicas que ayudaron a construir la formalización de las mónadas concurrentes. Entre ellas se encuentran los monoides, las mónadas, los funtores monoidales y los monoides concurrentes. Posteriormente se analizó el caso de la mónada \textit{delay} con el objetivo de demostrar que esta es una mónada concurrente. La demostración de la ley de intercambio mostró diversas complicaciones que llevaron a tomar la decisión de simplificar el problema a los números conaturales. Al intentar demostrar la misma ley para la versión reducida del problema, se presentaron dificultades en torno al soporte para coinducción elegido. Finalmente, se decidió cambiar este soporte por otro, logrando demostrar que los conaturales forman un monoide concurrente y definiendo una mónada concurrente que tiene ciertas similitudes con la mónada \textit{delay}. 

Las principales conclusiones de este trabajo son: 

\begin{enumerate}
\item La formalización de estructuras algebraicas puede realizarse en Agda mediante la utilización de tipos \AgdaKeyword{record}. Estos tipos admiten la definición de campos, los cuales se utilizaron para definir tanto los elementos que conforman la estructura como las propiedades que se deben cumplir para que dichos elementos efectivamente constituyan la estructura deseada. Al realizar las formalizaciones de esta manera, una instancia de uno de estos tipos no sólo define un ejemplo de la estructura formalizada, sino que también demuestra que el ejemplo definido cumple con las características necesarias para serlo. 

\item Agda es un lenguaje y asistente de pruebas muy potente pero puede llegar a traer muchas complicaciones a la hora de representar la coinducción. En general, la coinducción tiene conflictos con todos los lenguajes que no permitan la no terminación de programas puesto que, como se mencionó anteriormente, las pruebas por coinducción suelen ser infinitas. Esto hace que sea difícil convencer a los asistentes de pruebas de que las demostraciones son productivas y están bien definidas. 

\item El soporte para coinducción con tipos de tamaño definido (\textit{sized types}) ayuda al chequeo de terminación de programas de Agda, permitiendo realizar un número elevado de demostraciones que con el soporte de notación musical no eran posibles. Sin embargo, puede llegar a presentar problemas para reconocer y unificar valores que son iguales. 

\item Los números conaturales forman un monoide concurrente con la suma y el máximo como operaciones y el cero como elemento neutro. Esto quedó demostrado al generar la instancia de \AgdaRecord{ConcurrentMonoid} para el tipo \AgdaDatatype{Conat $\infty$}.

\item Si se tiene un conjunto $S$ que es un monoide concurrente, luego el funtor $\mathit{T}_S \ X = S \times X$ puede dotarse de una estructura de mónada concurrente. Esta implicancia se demostró en la prueba \AgdaFunction{cmonoid$\Rightarrow$cmonad}. 

\item La mónada \textit{writer} para los números conaturales constituye una mónada concurrente. La prueba \AgdaFunction{writerConatConcurrent} lo demuestra utilizando las pruebas mencionadas en los dos items anteriores. 
\end{enumerate}

Por cuestiones de tiempo y extensión de la tesina, quedaron algunas tareas pendientes para realizar más adelante. A continuación se detallan las principales:
\begin{enumerate}
\item Debido a la forma en la que se realizó la formalización de las mónadas concurrentes, al generar instancias de dicha estructura uno se ve obligado a utilizar la igualdad proposicional para los tipos de retorno. A futuro podía pensarse en modificar la formalización de manera que incluya como parámetro una noción de igualdad para el tipo de retorno. Así podrían darse instancias de mónadas concurrentes donde los valores de retorno se comparen mediante otros tipos de igualdad.

\item El soporte para coinducción utilizando \textit{sized types} parece más prometedor que el primero. Sería interesante intentar definir la mónada \textit{delay} utilizando una representación con dicho soporte y analizar si con esa representación se puede demostrar la ley de intercambio y, por lo tanto, probar que el tipo \textit{delay} constituye una mónada concurrente. 
\end{enumerate}