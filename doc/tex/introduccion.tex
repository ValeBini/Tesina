\chapter{Introducción}

En este primer capítulo se busca dar una introducción al trabajo, explicando la motivación del mismo y presentando también sus objetivos. Además se da una breve descripción de la estructura que tendrá la tesina y qué contenidos se incluyen en cada parte. 

\section{Motivación y estado del arte} 

La ley de Moore postula que el número de transistores que se pueden
poner en un chip de computadora se duplica (aproximadamente) cada un par
de años. De manera práctica, esto significa que para obtener
mejoras en la velocidad de ejecución de un programa simplemente hay
que esperar: después de un par de años, los programas que se escriben
hoy ejecutarán más rápido. 

Sin embargo, esta ley ya no se verifica en la práctica. Por razones físicas, es cada vez más costoso aumentar el número de transistores de
un chip y este tipo de progreso ya no puede ser garantizado. Por lo tanto, los
programadores se ven obligados a encontrar otras maneras de mejorar el
desempeño de sus programas. Una de las maneras más naturales que
surgió fue pensar que se pueden resolver varias tareas al mismo tiempo
y, de esa manera, evitar puntos ociosos en los que sólo se está esperando un evento del
entorno (por ejemplo, entrada/salida bloqueante). En los últimos años, con la
masificación de los microprocesadores con múltiples núcleos, se hizo aún
más evidente la necesidad de proveer al programador con la capacidad
de concurrencia.

Si bien los sistemas operativos modernos proveen primitivas para
manejar la concurrencia, no siempre los lenguajes de programación incorporan estas características de manera natural. Los lenguajes de
programación imperativos, en general, se basan en la idea de un modelo
de ejecución secuencial, donde la computación se desarrolla siguiendo
una única serie de pasos. Es así que lenguajes como C o Java pueden
manejar concurrencia pero sin proponer cambios radicales en la
concepción del lenguaje: simplemente incluyen librerías que capturan esta
capacidad de manera externa. La misma situación se refleja en los lenguajes de programación funcional, sobre todo cuando se escriben programas con
efectos. 

Los programas con efectos suelen representarse utilizando
mónadas, de manera que los programas funcionales toman una apariencia
fundamentalmente imperativa. El uso de mónadas para estructuras semánticas de lenguajes con efectos fue desarrollado originalmente por E. Moggi \cite{moggi:1989, moggi:1991}. Poco más tarde, P. Wadler \cite{wadler:1992} adaptó el
concepto de manera interna en los lenguajes de programación funcional,
dando origen a la programación con mónadas. Existe una gran variedad
de efectos que pueden ser capturados usando mónadas internas, por
ejemplo, estado, excepciones, entornos, continuaciones, etc. Al igual que en la programación imperativa, al considerar la concurrencia de los efectos, la
manera usual de propuesta es mediante llamadas a funciones \textit{ad-hoc} en lugar de usar primitivas bien fundadas que deriven de alguna estructura
matemática como lo hacen las operaciones de las mónadas.

Las mónadas concurrentes fueron recientemente introducidas en una
pre-impresión por M. Jaskelioff y E. Rivas \cite{rivas:2019}. Esta
definición surge de categorificar la noción de monoide concurrente,
definido por C. A. R. Hoare et al. \cite{hoare:2011} hace unos
años. Este concepto fue definido buscando obtener primitivas bien fundadas para la concurrencia, extendiendo las mónadas con nuevos operadores. Si bien esta estructura es
matemáticamente bien fundada, basada en monoides concurrentes, es
considerablemente complicado encontrar y probar modelos válidos de
esta estructura. 

La mónada Delay fue introducida por Capretta \cite{capretta:2005} con el
objetivo de capturar el efecto de la no terminación de programas de manera explícita y
uniforme. Su estructura fue estudiada en varios artículos, entre los
que se puede mencionar la tesis de N. Veltri \cite{veltri:2017}. Una hipótesis particular es que la mónada \textit{delay} puede ser dotada de una estructura de mónada concurrente. Agda es un asistente de pruebas basado en teoría de tipos con soporte
para inducción-recursión que ya ha sido utilizado para estudiar la
mónada \textit{delay}.

\section{Objetivos del trabajo}

Esta tesina tiene dos objetivos principales. El primero de ellos es formalizar los conceptos de monoide concurrente y mónada concurrente en el ámbito del lenguaje de pruebas Agda. Se busca realizar esta formalización de manera completa, incluyendo dentro de las estructuras las pruebas de las propiedades necesarias para demostrar que los elementos que las conforman efectivamente cumplen con las leyes de cada estructura. 

El segundo es estudiar la mónada \textit{delay} y su implementación en Agda para luego adaptar sus operaciones a la formalización propuesta con el propósito principal de probar o refutar la hipótesis de que a la mónada \textit{delay} se le puede dar una estructura de mónada concurrente.

\section{Estructura de la tesina}

La tesina está dividida en dos partes. La primera, denominada Preliminares, contiene tres capítulos que exponen el marco teórico que fue utilizado para el desarrollo de este trabajo. En el primero de ellos se da una introducción al lenguaje Agda, describiendo sus principales características y mostrando ejemplos de uso. El segundo presenta una introducción teórica a las mónadas concurrentes, partiendo de la teoría de categorías sobre la cual se definen las mónadas y luego yendo a lo más específico hasta dar el concepto de mónada concurrente y sus principales características. Por último, en el tercero se introduce la noción de coinducción y la definición y propiedades de la mónada \textit{delay}. 

En la segunda parte, cuyo título coincide con el de la tesina, se pueden encontrar dos capítulos. Cada capítulo se corresponde con uno de los objetivos del trabajo. En el primero se da la formalización de diversas estructuras algebraicas, partiendo desde algunas más básicas como los monoides, las mónadas y los funtores monoidales para luego llegar a las más complejas que son la meta de este trabajo: los monoides concurrentes y las mónadas concurrentes. Para cada estructura se explica cada uno de los campos introducidos y se hace un paralelismo con el marco teórico para que quede clara la relación entre la teoría y la formalización. En el siguiente capítulo se analiza el caso de la mónada \textit{delay}. Primero se define el tipo \textit{delay} en Agda junto con varias funciones útiles para manipularlo. Luego se prueba que este puede dotarse de una estructura de mónada y también de funtor monoidal utilizando las formalizaciones definidas. Finalmente, se intenta probar o refutar que este puede dotarse de una estructura de mónada concurrente.

\section{Código fuente}

El código fuente del trabajo puede descargarse en el siguiente link: \href{https://github.com/ValeBini/Tesina/releases/tag/v1.0}{Código Tesina}. Dentro de la carpeta se encuentra un archivo llamado \texttt{README.agda} que contiene un módulo \AgdaModule{README} en el que se indica qué archivos se corresponden con cada parte del trabajo. La versión de Agda que se utilizó en el desarrollo de esta tesina es la v2.6.1. 