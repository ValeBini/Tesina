\chapter*{Agradecimientos}

Con este trabajo culmina un camino largo que lleva casi siete años. En todo este tiempo hubo muchos momentos felices, pero también otros cargados de estrés en los que parecía que la meta estaba cada vez más lejos. Quiero agradecer en esta sección a todas las personas que estuvieron en cada uno de esos momentos. 

En primer lugar, quiero agradecer a todos mis compañeros. A todas esas personas con quienes compartí cursadas, TPs, rendidas o sólo charlas en el DCC, gracias por transitar este camino conmigo. También quiero dar gracias a todos los docentes que, con tanta dedicación y esfuerzo, buscan transmitirnos conocimiento y tantas otras cosas todos los días. Un gracias especial para aquellos cuyas materias me tocó cursar en pandemia y siguieron para adelante a pesar de la frustración y la falta de motivación que las circunstancias generaban. Gracias en general a toda la comunidad de la LCC de la cual estoy orgullosa de formar parte y deseo seguir participando desde cerca, ahora como ex-alumna y docente.

Gracias también a Exe, quien aceptó acompañarme en esta parte final y siempre me guió con mucha paciencia y dedicación. Aprendí mucho de él todo este tiempo y fue un placer hacer este trabajo final bajo su conducción.

A mis amigas, que están siempre firmes festejando conmigo cada logro y cada paso, gracias por estar, nada sería lo mismo sin ustedes. Gracias también a toda mi familia extendida, todavía no saben bien qué estudio pero siempre me preguntan y se interesan por todo lo que hago. A mis primos, que son como mis hermanos, gracias por estar siempre. 

Por último, quiero dar un gracias enorme a mi mamá, papá y hermanas que se bancaron mil y una crisis y llantos durante todos estos años y me acompañaron en todo. Ellos me enseñaron el valor del esfuerzo y a enfrentar cada desafío con amor y pasión, son mi motor y mi ejemplo a seguir todos los días. Un agradecimiento muy especial también para Bruno que, sobre todo este último tiempo, fue el principal receptor de todos mis nervios, malhumores y estrés y supo rescatarme de ellos y volverme a levantar cada vez. Él sabe todo lo que costó transitar este ultimo tramo y cuánto esperé que llegue este momento. Quiero dedicarles este trabajo y toda mi carrera a ellos y a mis dos ángeles del cielo, mi nono y mi nona, quienes sé que estuvieron conmigo todo este tiempo.

 

 
