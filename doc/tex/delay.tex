\chapter{La Mónada \textit{Delay}}\label{chapter:delay}

Como se discutió previamente, la teoría de tipos de Martin-Löf es un lenguaje de programación funcional rico con tipos dependientes y, a su vez, un sistema de lógica constructiva. Sin embargo, esto trae una limitación respecto de los lenguajes de programación funcional estándar, ya que obliga a que todas las computaciones deban terminar. Esta restricción tiene dos razones principales: hacer que el chequeo de tipos de los tipos dependientes sea decidible y representar pruebas como programas (una prueba que no termina es inconsistente). 

El tipo de dato \textit{delay} fue introducido por Capretta \cite{capretta:2005} con el objetivo de facilitar la representación de la no-terminación de funciones en la teoría de tipos de Martin-Löf. Lo que busca es explotar los tipos coinductivos para modelar computaciones infinitas. Los habitantes del tipo \textit{delay} son valores ``demorados'', los cuales pueden no terminar y, por lo tanto, no retornar un valor nunca. El tipo de dato \textit{delay} es una mónada y constituye una alternativa constructiva de la mónada \textit{maybe}. 

Se introducirá primero la noción de coinducción y un conjunto coinductivo muy especial: los números conaturales. Luego se presentará la definición de la mónada \textit{delay} y sus principales características. Finalmente, se describirán algunos de los distintos soportes para coinducción que Agda proporciona. 

\section{Introducción a la Coinducción}\label{delay:coind}

El principio de inducción está bien establecido en el área de las matemáticas y las ciencias de la computación. En esta última, se utiliza principalmente para razonar sobre tipos de datos definidos inductivamente, tales como listas finitas, árboles finitos y números naturales. La coinducción es el pincipio dual de la inducción y puede ser utilizado para razonar sobre tipos de datos definidos coinductivamente, tales como flujos de datos, trazas infinitas o árboles infinitos, pero no está tan difundido ni se comprende tan bien en general. Es por esta razón que esta sección comenzará primero con un repaso de lo conocido: la inducción, para luego introducir la coinducción haciendo un paralelismo entre ambas. A continuación se desarrolla la noción de definición inductiva. 

\begin{definition}[Definición inductiva]
Sean:
\begin{itemize}
\item $U$ un conjunto que se denominará \textbf{universo}.
\item $B$ un subconjunto no vacío de $U$ que tendrá el nombre de \textbf{base}. 
\item $K$ un conjunto no vacío de funciones llamado \textbf{constructor}. Cada función $f \in K$ tiene cierta aridad $ar(f) = n \in \mathbb{N}$, de manera que $f : U^n \rightarrow U$.
\end{itemize}
Un conjunto $A$ está \textbf{definido inductivamente} por $B$, $K$ y $U$ si es el mínimo conjunto que satisface que:
\begin{itemize}
\item $B \subseteq A$
\item si $a_1, ..., a_n \in A$ y $f(a_1, ..., a_n) = a$, entonces $a \in A$, donde $f \in K$. 
\end{itemize}
Cuando esto sucede se dice que $A$ fue generado por la base $B$ y las reglas de inducción $f \in K$. 
\end{definition}

Que $A$ sea el mínimo conjunto que cumple estas condiciones implica que si se tiene otro conjunto $A'$ que satisface las mismas condiciones, entonces $A \subseteq A'$.
De la mano con las definiciones inductivas viene el principio de inducción estructural, el cual sirve para demostrar propiedades sobre conjuntos definidos inductivamente. 

\begin{thm}[Principio de inducción estructural]
Sea $A \subseteq U$ definido inductivamente por la base $B$ y el constructor $K$. Sea $P$ una propiedad que verifica:
\begin{itemize}
\item $\forall x \in B$, $P(x)$ es verdadera.
\item Para cada $f \in K$, si $f(a_1, ..., a_{ar(f)}) = a$ y $P(a_i)$ se cumple para cada $i = 1, ..., ar(f)$, entonces $P(a)$ se cumple. 
\end{itemize}
Entonces se cumple $P(x) \ \forall x \in A$.
\end{thm}

Como se mencionó previamente, el principio de coinducción es el principio dual de la inducción. Asimismo, las definiciones coinductivas son duales a las definiciones inductivas. Un conjunto se define coinductivamente de la siguiente manera:

\begin{definition}[Definición coinductiva]
Sean:
\begin{itemize}
\item $U$ un conjunto que se denominará \textbf{universo}.
\item $B$ un subconjunto  de $U$ que tendrá el nombre de \textbf{base}. 
\item $K$ un conjunto no vacío de funciones llamado \textbf{constructor}. Cada función $f \in K$ tiene cierta aridad $ar(f) = n \in \mathbb{N}$, de manera que $f : U^n \rightarrow U$.
\end{itemize}
Un conjunto $A$ está \textbf{definido coinductivamente} por $B$, $K$ y $U$ si es el máximo conjunto que satisface que:
\begin{itemize}
\item $B \subseteq A$
\item si $a_1, ..., a_n \in A$ y $f(a_1, ..., a_n) = a$, entonces $a \in A$, donde $f \in K$. 
\end{itemize}
\end{definition}

Se puede observar que ambas definiciones son muy similares, la diferencia sustancial está en las palabras \textit{mínimo} y \textit{máximo}. Para ver más clara la diferencia, en el caso de una definición inductiva, se puede pensar que el conjunto a definir comienza siendo el conjunto vacío $\emptyset$ y se van añadiendo elementos iterativamente de acuerdo con las reglas. En una definición coinductiva, por el contrario, se puede imaginar que el conjunto comienza siendo el conjunto de todos los objetos posibles (el universo $U$), e iterativamente se van removiendo los objetos que contradicen las reglas. El hecho de que en este caso $A$ sea el máximo conjunto que cumple las condiciones, implica que si hay otro conjunto $A'$ que también las satisface, entonces $A' \subseteq A$.

Una vez que se tienen conjuntos definidos coinductivamente, se pueden probar propiedades sobre ellos mediante el principio de coinducción estructural, el cual fue introducido por primera vez por Milner y Tofte en 1991 \cite{milner:1991}.

\begin{thm}[Principio de coinducción estructural]
Sea $A \subseteq U$ definido coinductivamente por la base $B$ y el constructor $K$. Sea $P$ una propiedad que verifica:
\begin{itemize}
\item Para cada $f \in K$, si $f(a_1, ..., a_{ar(f)}) = a$ y $P(a_i)$ se cumple para cada $i = 1, ..., ar(f)$, entonces $P(a)$ se cumple. 
\end{itemize}
Entonces se cumple $P(x) \ \forall x \in A$.
\end{thm}

Al igual que antes, el principio es bastante similar al inductivo, excepto que no se pide que la propiedad se cumpla para los casos base. De hecho, un conjunto definido coinductivamente podría no tener un conjunto base ($B = \emptyset$), es el caso, por ejemplo, de los flujos de datos (\textit{streams}) infinitos. Aún teniendo un conjunto base, no quiere decir que todos los elementos del conjunto se construyan a partir de elementos de la base. Esta característica hace parecer que el paso coinductivo no está bien fundado y muchas veces genera dudas sobre la veracidad de las pruebas por coinducción. Sin embargo, siempre que la hipótesis coinductiva se aplique sobre elementos ``más pequeños'', la prueba tendrá sentido. Cualquier dificultad que haga que la propiedad a demostrar no se cumpla se manifiesta en el intento de probar el paso coinductivo.

Para ilustrar mejor el concepto de coinducción, se utilizarán algunos ejemplos presentados por Kozen y Silva \cite{kozen:2017} con el objetivo de promover este principio como una herramienta útil y hacerlo tan familiar e intuitivo como la inducción. 

A continuación se considera el ejemplo del tipo \texttt{Lista de $A$} de listas finitas sobre un alfabeto $A$, definido inductivamente por:
\begin{itemize}[label=$\blacktriangleright$]
	\item \texttt{nil} $\in$ \texttt{Lista de $A$}
	\item si $a \in A$ y $\ell \in$ \texttt{Lista de $A$}, entonces $a$ \texttt{::} $\ell \in$ \texttt{Lista de $A$}. 
\end{itemize}

El tipo de dato definido es la solución  mínima a la ecuación:
\begin{equation}\label{list}
\text{\tt Lista de $A$} = \text{\tt nil} + A \times \text{\tt Lista de $A$}
\end{equation}
Es decir que es el mínimo conjunto tal que se cumplen las condiciones listadas más arriba. Esto significa que uno puede definir funciones con dominio \texttt{Lista de $A$} de manera única por inducción estructural. El tipo de las listas finitas e infinitas sobre $A$ se define coinductivamente como la solución máxima de la ecuación \ref{list}. Esto significa que es el máximo conjunto tal que se cumplen ambas condiciones. En este ejemplo se evidencia que, a pesar de que el conjunto tiene un caso base, los elementos no necesariamente se construyen a partir de él, como es el caso de las listas de longitud infinita.

Si bien Milner y Tofte \cite{milner:1991} fueron los primeros en introducir la coinducción, ellos lo hicieron en base a puntos fijos mínimos y máximos. Existe otro  trasfondo teórico sobre el cual se puede definir formalmente la coinducción, el cual se basa en álgebras iniciales y coálgebras finales. Este enfoque es el que será utilizado en este trabajo y fue introducido principalmente por Jacobs y Rutten \cite{jacobs:2012}. Se definen a continuación los conceptos de álgebra, coálgebra, álgebra inicial y coálgebra final:

\begin{definition}[Álgebra de un funtor]
Dado un endofuntor $\mathit{F}$ sobre una categoría $\mathscr{C}$, un \textbf{álgebra} de $\mathit{F}$ es un objeto $X$ de $\mathscr{C}$ junto con un morfismo $\alpha : \mathit{F}X \rightarrow X$. Dadas dos álgebras $(X, \alpha : \mathit{F}X \rightarrow X)$, $(Y, \beta : \mathit{F}Y \rightarrow Y)$ de $F$, $m : X \rightarrow Y$ es un morfismo de álgebras si se cumple la siguiente ecuación:
\begin{equation*}
m \circ \alpha = \beta \circ \mathit{F}(m)
\end{equation*}
Las álgebras de $\mathit{F}$ junto con sus morfismos forman una categoría llamada $\mathit{F}$-álgebras. 
\end{definition}

\begin{definition}[Coálgebra de un funtor]
Una \textbf{coálgebra} para un endofuntor $\mathit{F}$ sobre una categoría $\mathscr{C}$ es un objeto $A$ junto con un morfismo $u : A \rightarrow \mathit{F} A$. Dadas dos coálgebras $(A, \eta : A \rightarrow \mathit{F}A), \quad (B, \theta : B \rightarrow \mathit{F}B)$, $f : A \rightarrow B$ es un morfismo de coálgebras si respeta la estructura coalgebraica: 
\begin{equation*}
\theta \circ f = \mathit{F}(f) \circ \eta
\end{equation*} 
Las coálgebras de $\mathit{F}$ junto con sus morfismos generan una categoría llamada $\mathit{F}$-coálgebras.
\end{definition}

\begin{definition}[Álgebra inicial]
Un \textbf{álgebra inicial} para un endofuntor $\mathit{F}$ sobre una categoría $\mathscr{C}$ es un objeto inicial en la categoría de las $\mathit{F}$-álgebras.
\end{definition}

\begin{definition}[Coálgebra final]
Una \textbf{coálgebra final} para un endofuntor $\mathit{F}$ sobre una categoría $\mathscr{C}$ es un objeto terminal en la categoría de las $\mathit{F}$-coálgebras. 
\end{definition}

Formalmente, el tipo de las listas finitas sobre $A$ es un álgebra inicial para una signatura que consiste en una constante (\texttt{nil}) y un constructor binario (\texttt{::}). El tipo de las listas finitas e infinitas sobre $A$ forman la coálgebra final de la signatura (\texttt{nil}, \texttt{::}). Los tipos coinductivos se definen como elementos de una coálgebra final para un endofuntor dado en la categoría \textbf{Set}. 

Se muestran a continuación algunos ejemplos más presentados por Kozen y Silva en \cite{kozen:2017}.

\begin{ejemplo}[Flujo de datos infinitos]
El conjunto $A^{\omega}$ de flujos de datos infinitos sobre un alfabeto $A$ es (el conjunto soporte de) la coálgebra final del funtor $\mathit{F}X = A \times X$.
\end{ejemplo}

\begin{ejemplo}[Cadenas infinitas]
El conjunto $A^{\infty}$ de las cadenas finitas e infinitas sobre un alfabeto $A$ es (el conjunto soporte de) la coálgebra final del funtor $\mathit{F}X = \mathds{1} + A \times X$.
\end{ejemplo}

Mientras que los tipos inductivos se definen mediante sus constructores, los tipos coinductivos usualmente se presentan junto con sus destructores. Por ejemplo, los flujos de datos o \textit{streams} admiten dos operaciones $hd: A^{\omega} \rightarrow A$ y $tl : A^{\omega} \rightarrow A^{\omega}$, los cuales representan la función $head$ que devuelve el primer elemento del \textit{stream} y la función $tail$ que devuelve la cola del \textit{stream}. La existencia de los destructores es una consecuencia del hecho de que $A^{\omega}$ es una coálgebra para el funtor $\mathit{F}X = A \times X$. Todas estas coálgebras vienen equipadas con una función estructural $\langle obs, cont \rangle  : X \rightarrow A \times X$; para $A^{\omega}$ se tiene que $obs = hd$ y $cont = tl$.

\section{Números Conaturales}\label{delay:conat}

Un conjunto coinductivo muy importante es el de los números conaturales, también llamados conúmeros. Como su nombre lo indica, es el conjunto dual al de los números naturales. Para introducir este conjunto, al igual que en la sección anterior, se partirá de lo conocido, siguiendo el estilo utilizado por Mike Gordon \cite{gordon:2017}.

La inducción matemática es la inducción estructural aplicada a la estructura de los números naturales $(\mathbb{N},0,S)$, donde $\mathbb{N}$ es un conjunto, $0 \in \mathbb{N}$ es una constante u operador de aridad $0$ (conjunto base $B = \{0\}$) y $S : \mathbb{N} \rightarrow \mathbb{N}$ es una función de aridad $1$ ($K = \{S\}$). La estructura de los números naturales está caracterizada por los cinco axiomas de Peano:
\begin{itemize}
\item $0 \in \mathbb{N}$
\item $\forall n \in \mathbb{N}, S(n) \in \mathbb{N}$
\item $\forall n \in \mathbb{N}, S(n) \neq 0$
\item $\forall m \in \mathbb{N}, \forall n \in \mathbb{N}, S(m) = S(n) \Rightarrow m = n$
\item $\forall M, 0 \in M \land (\forall n \in M, S(n) \in M) \Rightarrow \mathbb{N} \subseteq M$
\end{itemize}

La estructura $(\mathbb{N},0,S)$ es una instancia de la clase de estructuras $(\mathcal{A},z,s)$, donde $\mathcal{A}$ es un conjunto, $z \in \mathcal{A}$ y $s: \mathcal{A} \rightarrow \mathcal{A}$. Estas estructuras se denominan estructuras de Peano. Para cada una de ellas existe una única función $f : \mathbb{N} \rightarrow \mathcal{A}$ tal que $f(0) = z$ y $\forall n \in \mathbb{N}, f(S(n)) = s(f(n))$. Una estructura de Peano es un álgebra (pueden llamarse también álgebras de Peano) y $(\mathbb{N},0,S)$ es un álgebra inicial en la categoría de las álgebras de Peano.
De manera dual, la estructura de los conaturales es una coálgebra final. 

El conjunto de los números conaturales $\bar{\mathbb{N}}$ puede definirse coloquialmente como el máximo conjunto que cumple que $0 \in \bar{\mathbb{N}}$ y si $n \in \bar{\mathbb{N}}$ entonces $S(n) \in \bar{\mathbb{N}}$ (las mismas condiciones que los números naturales). Sin embargo, como se mencionó anteriormente, los conjuntos coinductivos suelen definirse por sus destructores en lugar de sus constructores. Esto hace que la definición sea más intuitiva y se comprenda mejor. 

El dual del constructor unario $S$ es la función predecesor $P$ definida como $P(n) = n-1$, donde $P$ es una función parcial definida sólo para los números distintos de $0$, por lo cual $\mathrm{Dom}(P) = \{n \ |\  n > 0\}$. Los constructores de aridad $0$ representan elementos distinguidos del conjunto soporte, por lo que no es evidente cuáles son sus destructores correspondientes ya que no hay una componente del constructor que retornar. Para enfrentar esto, los destructores correspondientes a constructores 0-arios retornan un valor arbitrario que representa ``sin componentes''. Este puede ser denotado como $\star$ o $()$. El destructor dual al constructor 0-ario $0$ es la función parcial $is0 : \{0\} \rightarrow \{\star\}$, por lo que necesariamente $is0(0) = \star$ y $\mathrm{Dom}(is0) = \{0\}$. 

Si una coálgebra tiene más de un destructor, entonces todos sus destructores son funciones parciales. El dual de un álgebra de Peano $(\mathcal{A},z,s)$ es una coálgebra de Peano $(\mathcal{C},isz,p)$, donde $isz$ y $p$ son destructores: $isz: \mathcal{C} \nrightarrow  \{\star\}$ y $p : \mathcal{C} \nrightarrow \mathcal{C}$. Los dominios de $isz$ y $p$ forman una partición de $\mathcal{C}$, por lo que si $x \in \mathcal{C}$ entonces o bien $x \in \mathrm{Dom}(isz)$ o bien $x \in \mathrm{Dom}(p)$, pero no ambos. Los números conaturales son la coálgebra de Peano $(\bar{\mathbb{N}},is0,P)$, con la propiedad de que para cualquier coálgebra de Peano $(\mathcal{C},isz,p)$ hay una única función $g : \mathcal{C} \rightarrow \bar{\mathbb{N}}$ tal que para cada $x \in \mathcal{C}$:
\begin{itemize}
\item si $x \in \mathrm{Dom}(isz)$ entonces $g(x) \in \mathrm{Dom}(is0)$ y $is0(g(x)) = isz(x)$;
\item si $x \in \mathrm{Dom}(p)$ entonces $g(x) \in \mathrm{Dom}(P)$ y $P(g(x)) = g(p(x))$.
\end{itemize}

La coálgebra $(\bar{\mathbb{N}},is0,P)$ es entonces una coálgebra terminal o final en la categoría de las coálgebras de Peano.  

Intuitivamente, un número conatural puede ser o bien el sucesor de otro (está en el dominio de la función $P$ y se puede obtener su predecesor) o bien $0$ (está en el dominio de $is0$), evidenciando el paralelismo entre ambas definiciones. La diferencia con los naturales es que, en este caso, un conatural puede ser sucesor de otro ``infinitamente'', puesto que no hay condición alguna que indique que al aplicar la función predecesor sucesivamente debe llegarse obligatoriamente a $0$. Esto da lugar a un valor especial $\infty$ tal que $P(\infty) = \infty$. El conjunto soporte $\bar{\mathbb{N}}$ sería de alguna manera equivalente a $\mathbb{N} \cup \{\infty\}$. 


\section{Mónada \textit{Delay}}\label{delay:delay}

El tipo de dato \textit{delay} consituye una mónada fuerte, lo cual hace posible manejar computaciones que posiblemente no terminen como si fuera cualquier otra computación con efectos laterales. A continuación se presenta su definición formal siguiendo el estilo utilizado en \cite{chapman:2019}. 

\begin{definition}[\textit{Delay}]
Sea $X$ un tipo. Cada habitante de $\mathbf{D} X$ es una computación posiblemente infinita que, si termina, retorna un valor de tipo $X$. Se define $\mathbf{D} X$ como un tipo coinductivo mediante las siguientes reglas:
\begin{equation*}
\dfrac{}{\mathtt{now} \ x : \mathbf{D} X} 	\qquad  	\dfrac{c : \mathbf{D} X}{\mathtt{later} \ c : \mathbf{D} X}
\end{equation*}
\end{definition}

Sea $R$ una relación de equivalencia sobre $X$. La relación se eleva a una relación de equivalencia $\sim_R$ sobre $\mathbf{D}X$ que se denomina $R$-bisemejanza fuerte (\textit{$R$-strong bisimilarity} en inglés). 

\begin{definition}[$R$-bisemejanza fuerte]
Dada una relación de equivalencia $R$ sobre $X$, se define la relación $\sim_R$ sobre $\mathbf{D}X$ coinductivamente mediante las siguientes reglas:
\begin{equation*}
\dfrac{p : x_1 \ R \ x_2}{\mathtt{now}_{\sim} \ p : \mathtt{now} \ x_1 \sim_R \mathtt{now} \ x_2}  	\qquad  	\dfrac{p : c_1 \sim_R c_2}{\mathtt{later}_{\sim} \ p : \mathtt{later} \ c_1 \sim_R \mathtt{later} \ c_2}
\end{equation*}
\end{definition}

La $\equiv$-bisemejanza fuerte se denomina simplemente bisemejanza fuerte y se denota $\sim$.

En algunos casos, uno está interesado en la terminación de las computaciones y no exactamente en el tiempo exacto en el cual terminan. Es deseable entonces tener una relación que considere iguales dos computaciones si terminan con valores iguales, aunque una tarde más en terminar que la otra. Es decir, que identifique computaciones que sólo difieren en una cantidad finita de aplicaciones del constructor \texttt{later}. Esta relación se llama $R$-bisemejanza débil (\textit{$R$-weak bisimilarity} en inglés) y se define en términos de \textit{convergencia}. Esta última es una relación binaria entre $\mathbf{D}X$ y $X$ que relaciona computaciones que terminan con sus valores de terminación. 

\begin{definition}[Convergencia]
La relación de \textbf{convergencia} denotada con $\downarrow$ entre $\mathbf{D}X$ y $X$ se define inductivamente mediante las siguientes reglas:
\begin{equation*}
\dfrac{p : x_1 \equiv x_2}{\mathtt{now}_{\downarrow}  \ p : \mathtt{now} \ x_1 \downarrow x_2}  	\qquad  	\dfrac{p : c \downarrow x}{\mathtt{later}_{\downarrow} \ p : \mathtt{later} \ c \downarrow x}
\end{equation*}
\end{definition}

\begin{definition}[$R$-bisemejanza débil]
Dada una relación de equivalencia $R$ sobre $X$, se define la relación $\approx_R$ sobre $\mathbf{D}X$ coinductivamente mediante las siguientes reglas:
\begin{equation*}
\dfrac{p_1 : c_1 \downarrow x_1 \quad p_2 : x_1 \ R \ x_2 \quad p_3 : c_2 \downarrow x_2}{\downarrow_{\approx} \ p_1 \ p_2 \ p_3 : c_1 \approx_R c_2}  	\qquad  	\dfrac{p : c_1 \approx_R c_2}{\mathtt{later}_{\approx} \ p : \mathtt{later} \ c_1 \approx_R \mathtt{later} \ c_2}
\end{equation*}
\end{definition}

La $\equiv$-bisemejanza débil se denomina simplemente bisemejanza débil y se denota $\approx$. En este caso, se modifica el primer constructor por simplicidad:

\begin{equation*}
\dfrac{p_1 : c_1 \downarrow x \quad p_2 : c_2 \downarrow x}{\downarrow_{\approx} \ p_1 \ p_2 : c_1 \approx c_2}
\end{equation*}

El tipo \textit{delay} $\mathbf{D}$ es una mónada fuerte. La unidad $\eta$ es el constructor \texttt{now}, mientras que la multiplicación $\mu$ es la ``concatenación'' de constructores \texttt{later}:
\begin{align*}
& \mu : \mathbf{D} (\mathbf{D} X) \rightarrow \mathbf{D} X  \\
& \mu \ (\mathtt{now} \ c) = c \\
& \mu \ (\mathtt{later} \ c) = \mathtt{later} \ (\mu \ c)
\end{align*}


\section{Coinducción en Agda}\label{coind:agda}

Se describirán a continuación los dos soportes de coinducción en Agda que se utilizaron en esta Tesina. El primero se basa en una notación particular, la notación musical, la cual permite manejar términos potencialmente infinitos. A pesar de ser una notación práctica e intuitiva, este soporte tiene algunos problemas con el chequeo de terminación de Agda, lo que limita bastante las propiedades que pueden demostrarse usándolo. Es por eso que se utilizó luego otro enfoque, basado en tipos de tamaño limitado (\textit{sized types} en inglés), el cual ayuda al chequeo de terminación de Agda haciendo un seguimiento de la profundidad de las estructuras de datos mediante la definición de límites en la profundidad. 

\subsection{Notación Musical}\label{coind:agda:musical}

Para mostrar la notación musical se utilizará como ejemplo el conjunto de los números \textit{conaturales}. Así como las coálgebras son el dual de las álgebras, los números conaturales son el dual de los números naturales y se definen en Agda como sigue:

\ExecuteMetaData[latex/Coind.tex]{musconat}

El operador \textit{delay} (\AgdaDatatype{$\infty$}) se utiliza para etiquetar ocurrencias coinductivas. El tipo \AgdaDatatype{$\infty$ A} se interpretea como una computación suspendida o demorada de tipo \AgdaDatatype{A}. Este operador viene equipado con funciones \textit{delay} y \textit{force}:

\ExecuteMetaData[latex/Coind.tex]{delayforce}

La función \textit{delay} (\AgdaFunction{$\sharp\_$}) toma un valor de tipo \AgdaDatatype{A} y lo devuelve suspendido dentro de un valor de tipo \AgdaDatatype{$\infty$ A}. Por el contrario, la función \textit{force} (\AgdaFunction{$\flat$}), toma un valor de tipo \AgdaDatatype{$\infty$ A} y lo desencapsula devolviendo un valor de tipo \AgdaDatatype{A}.

Los valores de tipos coinductivos pueden ser construidos usando corecursión, la cual no debe necesariamente terminar, pero sí ser productiva. Por ejemplo, el infinito puede ser difinido como se muestra a continuación.

\ExecuteMetaData[latex/Coind.tex]{inf}

Como aproximación a la productividad, en el chequeo de terminación se pide que en la definición de funciones corecursivas las llamadas recursivas aparezcan bajo la aplicación directa de un construcor coinductivo. Esta restricción en general hace que programar con tipos coinductivos sea incómodo, y es por eso que se buscan técnicas alternativas para asegurar que las definiciones corecursivas estén bien definidas. 

\subsection{\textit{Sized Types}}

Agda tiene un soporte nativo para \textit{sized types}. Estos son tipos que cuentan con un índice que indica el número de desencapsulamientos que pueden realizarse sobre los habitantes de este tipo. Estos índices, llamados tamaños o \textit{sizes}, asisten al chequeo de terminación evaluando que las definiciones corecursivas estén bien definidas. 

En Agda existe un tipo \AgdaPrimitiveType{Size} de tamaños y un tipo \AgdaPrimitiveType{Size<} \AgdaArgument{i} cuyos habitantes son los tamaños estrictamente menores a \AgdaArgument{i}. Si se tiene un tamaño \AgdaArgument{j} \AgdaSymbol{:} \AgdaPrimitiveType{Size<} \AgdaArgument{i}, este es forzado a ser \AgdaArgument{j} \AgdaSymbol{:} \AgdaPrimitiveType{Size}. La relación de orden de los tamaños es transitiva, lo que implica que si se tiene que \AgdaArgument{j} \AgdaSymbol{:} \AgdaPrimitiveType{Size<} \AgdaArgument{i} y \AgdaArgument{k} \AgdaSymbol{:} \AgdaPrimitiveType{Size<} \AgdaArgument{j}, entonces \AgdaArgument{k} \AgdaSymbol{:} \AgdaPrimitiveType{Size<} \AgdaArgument{i}. La relación de orden es, además, bien fundada, lo cual se usa para definir funciones corecursivas productivas. Existe también una operación sucesor de tamaños \AgdaFunction{$\uparrow$} y un tamaño ``infinito'' \AgdaArgument{$\infty$} tal que para cada tamaño \AgdaArgument{i}, \AgdaArgument{i} \AgdaSymbol{:} \AgdaPrimitiveType{Size<} \AgdaArgument{$\infty$}. Finalmente, un \textit{sized type} es un tipo indexado por \AgdaPrimitiveType{Size}.

Para ejemplificar el uso de los \textit{sized types}, se definen a continuación los conaturales utilizando esta técnica.

\ExecuteMetaData[latex/Coind.tex]{sizedconat}

Ambos tipos, \AgdaDatatype{Conat} y \AgdaRecord{Conat$'$} están indexados por un tamaño \AgdaArgument{i}. Este índice debe entenderse como una cota superior del número de veces que puede aplicarse \AgdaField{force}. Más precisamente, cuando se aplica \AgdaField{force} a un \AgdaArgument{n$'$} \AgdaSymbol{:} \AgdaRecord{Conat$'$} \AgdaArgument{i} el valor resultante es es un \AgdaArgument{n} \AgdaSymbol{:} \AgdaDatatype{Conat} \AgdaArgument{j} de una profundidad estrictamente menor \AgdaArgument{j} \AgdaSymbol{<} \AgdaArgument{i}. Un caso especial es el valor \AgdaArgument{$\infty$n$'$} \AgdaSymbol{:} \AgdaRecord{Conat$'$} \AgdaArgument{$\infty$} de índice infinito, cuyo resultado de aplicar \AgdaField{force} es \AgdaArgument{$\infty$n} \AgdaSymbol{:} \AgdaDatatype{Conat} \AgdaArgument{$\infty$}, el cual también tiene índice infinito. De esta manera los tamaños establecen productividad en las definiciones recursivas. Al final, sólo interesan los valores \AgdaArgument{n} \AgdaSymbol{:} \AgdaDatatype{Conat} \AgdaArgument{$\infty$} de índice infinito.

Si una función corecursiva en \AgdaDatatype{Conat} \AgdaArgument{i} sólo se llama a sí misma con índices menores \AgdaArgument{j} \AgdaSymbol{<} \AgdaArgument{i}, se garantiza la productividad y, por lo tanto, está bien definida. En la siguiente definición del valor \AgdaFunction{infty} se muestran los argumentos implícitos de tamaño explícitamente de manera que se evidencie cómo aseguran la productividad:

\ExecuteMetaData[latex/Coind.tex]{sizedinf}