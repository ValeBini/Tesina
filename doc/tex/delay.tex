\chapter{La Mónada \textit{Delay}}\label{chapter:delay}

Como se discutió previamente, la teoría de tipos de Martin-Löf es un lenguaje de programación funcional rico con tipos dependientes y, a su vez, un sistema de lógica constructiva. Sin embargo, esto trae una limitación respecto de los lenguajes de programación funcional estándar, ya que obliga a que todas las computaciones deban terminar. Esta restricción tiene dos razones principales: hacer que el chequeo de tipos de los tipos dependientes sea decidible y representar pruebas como programas (una prueba que no termina es inconsistente). 

El tipo de dato \textit{delay} fue introducido por Capretta \cite{capretta:2005} con el objetivo de facilitar la representación de la no-terminación de funciones en la teoría de tipos de Martin-Löf. Lo que busca es explotar los tipos coinductivos para modelar computaciones infinitas. Los habitantes del tipo \textit{delay} son valores ``demorados'', los cuales pueden no terminar y, por lo tanto, no retornar un valor nunca. El tipo de dato \textit{delay} es una mónada y constituye una alternativa constructiva de la mónada \textit{maybe}. 

Se introducirá primero la noción de coinducción, junto con los soportes para coinducción que Agda proporciona. Luego se presentará la definición de la mónada \textit{delay} y sus principales características.

\section{Introducción a la Coinducción}\label{delay:coind}

El principio de inducción está bien establecido en el área de las matemáticas y las ciencias de la computación. En esta última, se utiliza principalmente para razonar sobre tipos de datos definidos inductivamente, tales como listas finitas, árboles finitos y números naturales. La coinducción es el pincipio dual de la inducción y puede ser utilizado para razonar sobre tipos de datos definidos coinductivamente, tales como flujos de datos, trazas infinitas o árboles infinitos, pero no está tan difundido ni se comprende tan bien en general. 

Para ilustrar mejor el concepto de coinducción, se utilizarán algunos ejemplos presentados por Kozen y Silva \cite{kozen:2017} con el objetivo de promover este principio como una herramienta útil y hacerlo tan familiar e intuitivo como la inducción. 

A continuación se considera el ejemplo del tipo \texttt{Lista de $A$} de listas finitas sobre un alfabeto $A$, definido inductivamente por:
\begin{itemize}[label=$\blacktriangleright$]
	\item \texttt{nil} $\in$ \texttt{Lista de $A$}
	\item si $a \in A$ y $\ell \in$ \texttt{Lista de $A$}, entonces $a$ \texttt{::} $\ell \in$ \texttt{Lista de $A$}. 
\end{itemize}

El tipo de dato definido es la solución  mínima a la ecuación:
\begin{equation}\label{list}
\text{\tt Lista de $A$} = \text{\tt nil} + A \times \text{\tt Lista de $A$}
\end{equation}
Es decir, que es el mínimo conjunto tal que se cumplen las condiciones listadas más arriba. Esto significa que uno puede definir funciones con dominio \texttt{Lista de $A$} de manera única por inducción estructural. El tipo de las listas finitas e infinitas sobre $A$ se define coinductivamente como la solución máxima de la ecuación \ref{list}. Esto significa que es el máximo conjunto tal que se cumplen ambas condiciones. 

Formalmente, el tipo de las listas finitas sobre $A$ es un álgebra inicial para una signatura que consiste en una constante (\texttt{nil}) y un constructor binario (\texttt{::}). El tipo de las listas finitas e infinitas sobre $A$ forman la coálgebra final de la signatura (\texttt{nil}, \texttt{::}). Se definen a continuación los conceptos de álgebra, coálgebra, álgebra inicial y coálgebra final:

\begin{definition}[Álgebra de un funtor]
Dado un endofuntor $\mathit{F}$ sobre una categoría $\mathscr{C}$, un \textbf{álgebra} de $\mathit{F}$ es un objeto $X$ de $\mathscr{C}$ junto con un morfismo $\alpha : \mathit{F}X \rightarrow X$. Dadas dos álgebras $(X, \alpha : \mathit{F}X \rightarrow X)$, $(Y, \beta : \mathit{F}Y \rightarrow Y)$ de $F$, $m : X \rightarrow Y$ es un morfismo de álgebras si se cumple la siguiente ecuación:
\begin{equation*}
m \circ \alpha = \beta \circ \mathit{F}(m)
\end{equation*}
Las álgebras de $\mathit{F}$ junto con sus morfismos forman una categoría llamada $\mathit{F}$-álgebras. 
\end{definition}

\begin{definition}[Coálgebra]
Una \textbf{coálgebra} para un endofuntor $\mathit{F}$ sobre una categoría $\mathscr{C}$ es un objeto $A$ junto con un morfismo $u : A \rightarrow \mathit{F} A$. Dadas dos coálgebras $(A, \eta : A \rightarrow \mathit{F}A), \quad (B, \theta : B \rightarrow \mathit{F}B)$, $f : A \rightarrow B$ es un morfismo de coálgebras si respeta la estructura coalgebraica: 
\begin{equation*}
\theta \circ f = \mathit{F}(f) \circ \eta
\end{equation*} 
Las coálgebras de $\mathit{F}$ junto con sus morfismos generan una categoría llamada $\mathit{F}$-coálgebras.
\end{definition}

\begin{definition}[Álgebra inicial]
Un \textbf{álgebra inicial} para un endofuntor $\mathit{F}$ sobre una categoría $\mathscr{C}$ es un objeto inicial en la categoría de las $\mathit{F}$-álgebras.
\end{definition}

\begin{definition}[Coálgebra final]
Una \textbf{coálgebra final} para un endofuntor $\mathit{F}$ sobre una categoría $\mathscr{C}$ es un objeto terminal en la categoría de las $\mathit{F}$-coálgebras. 
\end{definition}

Formalmente, los tipos coinductivos se definen como elementos de una coálgebra final para un endofuntor dado en la categoría \textbf{Set}. 

\begin{ejemplo}[Flujo de datos infinitos]
El conjunto $A^{\omega}$ de flujos de datos (o \textit{streams} en inglés) infinitos sobre un alfabeto $A$ es (el conjunto soporte de) la coálgebra final del funtor $\mathit{F}X = A \times X$.
\end{ejemplo}

\begin{ejemplo}[Cadenas infinitas]
El conjunto $A^{\infty}$ de las cadenas finitas e infinitas sobre un alfabeto $A$ es (el conjunto soporte de) la coálgebra final del funtor $\mathit{F}X = \mathds{1} + A \times X$.
\end{ejemplo}

Mientras que los tipos inductivos se definen mediante sus constructores, los tipos coinductivos usualmente se presentan junto con sus destructores. Por ejemplo, los flujos de datos o \textit{streams} admiten dos operaciones $hd: A^{\omega} \rightarrow A$ y $tl : A^{\omega} \rightarrow A^{\omega}$, los cuales representan la función $head$ que devuelve el primer elemento del \textit{stream} y la función $tail$ que devuelve la cola del \textit{stream}. La existencia de los destructores es una consecuencia del hecho de que $A^{\omega}$ es una coálgebra para el funtor $\mathit{F}X = A \times X$. Todas estas coálgebras vienen equipadas con una función estructural $\langle obs, cont \rangle  : X \rightarrow A \times X$; para $A^{\omega}$ se tiene que $obs = hd$ y $cont = tl$.

Las pruebas por coinducción tienen un paso coinductivo (análogo al paso inductivo) pero no caso base. Aunque esto parezca incorrecto o genere cierta desconfianza en dichas pruebas, cualquier dificultad que haga que la propiedad a demostrar no se cumpla se manifiesta en el intento de probar el paso coinductivo.

\subsection{Coinducción en Agda}\label{coind:agda}

Se describirán a continuación los dos soportes de coinducción en Agda que se utilizaron en esta Tesina. El primero se basa en una notación particular, la notación musical, la cual permite manejar términos potencialmente infinitos. A pesar de ser una notación práctica e intuitiva, este soporte tiene algunos problemas con el chequeo de terminación de Agda, lo que limita bastante las propiedades que pueden demostrarse usándolo. Es por eso que se utilizó luego otro enfoque, basado en tipos de tamaño limitado (\textit{sized types} en inglés), el cual ayuda al chequeo de terminación de Agda haciendo un seguimiento de la profundidad de las estructuras de datos mediante la definición de límites en la profundidad. 

\subsubsection{Notación Musical}\label{coind:agda:musical}

Para mostrar la notación musical se utilizará como ejemplo el conjunto de los números \textit{conaturales}. Así como las coálgebras son el dual de las álgebras, los números conaturales son el dual de los números naturales y se definen en Agda como sigue:

\ExecuteMetaData[latex/Coind.tex]{musconat}

El operador \textit{delay} (\AgdaDatatype{$\infty$}) se utiliza para etiquetar ocurrencias coinductivas. El tipo \AgdaDatatype{$\infty$ A} se interpretea como una computación suspendida o demorada de tipo \AgdaDatatype{A}. Este operador viene equipado con funciones \textit{delay} y \textit{force}:

\ExecuteMetaData[latex/Coind.tex]{delayforce}

La función \textit{delay} (\AgdaFunction{$\sharp\_$}) toma un valor de tipo \AgdaDatatype{A} y lo devuelve suspendido dentro de un valor de tipo \AgdaDatatype{$\infty$ A}. Por el contrario, la función \textit{force} (\AgdaFunction{$\flat$}), toma un valor de tipo \AgdaDatatype{$\infty$ A} y lo desencapsula devolviendo un valor de tipo \AgdaDatatype{A}.

Los valores de tipos coinductivos pueden ser construidos usando corecursión, la cual no debe necesariamente terminar, pero sí ser productiva. Como aproximación a la productividad, en el chequeo de terminación se pide que las definiciones corecursivas sean protegidas por constructores coinductivos. Por ejemplo, el infinito puede ser difinido como se muestra a continuación.

\ExecuteMetaData[latex/Coind.tex]{inf}

\subsubsection{\textit{Sized Types}}



\section{Mónada \textit{Delay}}\label{delay:delay}