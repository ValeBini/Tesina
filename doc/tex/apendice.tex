\chapter{Pruebas complementarias}\label{apendice:pruebas}

\section{Pruebas análogas para bisemejanza fuerte}\label{apendice:fuerte}
 
En la sección \ref{casodelay:monada} se prueba que el tipo \textit{delay} es una mónada tomando como relación de igualdad la bisemejanza débil. El siguiente bloque de código corresponde a la prueba análoga, tomando en este caso como relación de igualdad la bisemejanza fuerte. 

\ExecuteMetaData[latex/Instances/Monad.tex]{strong}

La estructura de funtor monoidal para el mismo tipo de dato fue introducida en la sección \ref{casodelay:monoidal}, también considerando como relación de igualdad la bisemejanza débil. A continuación se muestra el código correspondiente a la estructura de funtor monoidal para el tipo \textit{delay} con la bisemejanza fuerte como igualdad.

\ExecuteMetaData[latex/Instances/Monoidal.tex]{strong}

\section{Los conaturales con notación musical: lemas auxiliares}\label{apendice:lemasmus}

Los siguientes lemas se utilizan en la sección \ref{casodelay:conat:ichange} para mostrar los intentos de prueba de la ley de intercambio para números conaturales definidos con notación musical. 

\AgdaFunction{$\equiv\Rightarrow\gtrsim$} prueba que la desigualdad es reflexiva respecto de la igualdad proposicional:
\ExecuteMetaData[latex/ConaturalsM.tex]{propmayor}

Los lemas \AgdaFunction{maxzero$_1$}, \AgdaFunction{maxzero$_2$}, \AgdaFunction{sumzero$_1$} y \AgdaFunction{sumzero$_2$} prueban que \AgdaInductiveConstructor{zero} es neutro a derecha de \AgdaFunction{max} y \AgdaFunction{sum} probando las dos desigualdades (ya que en los demás lemas se necesitaba de esta manera):
\ExecuteMetaData[latex/ConaturalsM.tex]{zerolemas}

\AgdaFunction{sym-sum} y \AgdaFunction{sym-max} prueban que ambas operaciones son simétricas:
\ExecuteMetaData[latex/ConaturalsM.tex]{syms}

\AgdaFunction{$\gtrsim$sum} prueba que la suma es compatible con la relación de orden, para ello se prueba también \AgdaFunction{$\gtrsim$sumzero} que prueba lo mismo para el caso en que el argumento \AgdaBound{n$_2$} es \AgdaInductiveConstructor{zero}:
\ExecuteMetaData[latex/ConaturalsM.tex]{sumcong}

Análogamente, \AgdaFunction{$\gtrsim$max} prueba que \AgdaFunction{max} es compatible con la relación de orden, utilizando \AgdaFunction{$\gtrsim$maxzero} para el caso en que \AgdaBound{n$_2$} $=$ \AgdaInductiveConstructor{zero}:
\ExecuteMetaData[latex/ConaturalsM.tex]{maxcong}

\section{Los conaturales con \textit{sized types}: lemas auxiliares}\label{apendice:lemassz}

Los lemas que se presentan a continuación se utlizan en la sección \ref{casodelay:sized:concurrent} para probar que los números conaturales bajo la representación que utiliza \textit{sized types} forman un monoide concurrente.

El primer lema representa la conmutatividad de la suma. Para probar tal propiedad se requiere un lema extra que postula que sumar $1$ a la suma \AgdaBound{m} \AgdaFunction{+} \AgdaField{force} \AgdaBound{n} es igual a realizar la suma \hbox{\AgdaBound{m} \AgdaFunction{+} \AgdaInductiveConstructor{suc} \AgdaBound{m}}, lo cual es similar a definir la suma al revés en el sentido de ir reduciendo el conúmero de la derecha en lugar del de la izquierda. Este lema se prueba junto con otro que permite cambiar el constructor \AgdaInductiveConstructor{suc} de lado en la suma de manera mutuamente recursiva.
\ExecuteMetaData[latex/Instances/ConcurrentMonoid.tex]{commsuma}

El lema \AgdaFunction{\_+-cong\_} representa la propiedad que indica que si se tienen dos pares de conúmeros bisemejantes \AgdaBound{m$_1$} \AgdaFunction{$\sim$} \AgdaBound{m$_2$} y \AgdaBound{n$_1$} \AgdaFunction{$\sim$} \AgdaBound{n$_2$}, la suma de los dos primeros es bisemejante a la suma de los dos segundos.
\ExecuteMetaData[latex/Instances/ConcurrentMonoid.tex]{congsuma}

El lema \AgdaFunction{\_max-cong\_} representa la propiedad análoga a la anterior para el operador \AgdaFunction{max}.
\ExecuteMetaData[latex/Instances/ConcurrentMonoid.tex]{congmax}

El siguiente lema, \AgdaFunction{pred-max}, demuestra que el máximo de los predecesores de dos conúmeros es igual al predecesor del máximo de los mismos.
\ExecuteMetaData[latex/Instances/ConcurrentMonoid.tex]{predmax}

Por último, el lema \AgdaFunction{max$\leq$+} prueba que el máximo de dos conúmeros siempre es menor o igual a su suma.
\ExecuteMetaData[latex/Instances/ConcurrentMonoid.tex]{maxmenorsuma}