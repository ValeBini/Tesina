\chapter{El caso de la Mónada Delay}\label{chapter:casodelay}

Como se mencionó en el capítulo \ref{chapter:delay}, el tipo \textit{delay} fue introducido por Capretta \cite{capretta:2005} para representar la posible no terminación de programas en la teoría de tipos de Martin-Löf. Sus habitantes son valores ``demorados'', los cuales pueden no terminar nunca. El objetivo de este capítulo es hacer un análisis de este tipo respecto de las estructuras algebraicas previamente definidas. 

Inicialmente se dará la definición de este tipo en Agda, utilizando para ello la notación musical para tipos coinductivos que fue descripta en la sección \ref{coind:agda:musical}. {\color{red} Comentario: agrego un link al repo de donde sacamos esto?} Junto con el tipo se definirán diversas relaciones sobre él, entre las que se encuentran las bisemejanzas débil y fuerte y una relación de orden. 

Una vez introducido el tipo, se demostrará que este tiene estructura de mónada y de funtor monoidal. Para esto se crearán instancias de las estructuras correspondientes para el tipo \textit{delay}. El objetivo final de este capítulo es probar o refutar que el tipo definido tiene estructura de mónada concurrente. La principal dificultad para esto subyace en la prueba de la ley de intercambio, ya que los demás ingredientes están presentes en las pruebas de mónada y funtor monoidal. 

\section{Definición del tipo \textit{delay} con notación musical}\label{casodelay:defmus}

El tipo \textit{delay} se define con notación musical mediante una estructura \AgdaKeyword{data} parametrizada. El parámetro será un conjunto \AgdaBound{A} de tipo \AgdaPrimitiveType{Set}, el cual representa el tipo de los valores de retorno (en caso de que el programa termine). Dado entonces un \AgdaBound{A} \AgdaSymbol{:} \AgdaPrimitiveType{Set}, el tipo \AgdaBound{A} \AgdaDatatype{$\bot$} representa el tipo $\mathbf{D}$\AgdaBound{A} definido en la sección \ref{delay:delay}. 

\ExecuteMetaData[latex/Delay.tex]{bottom}

El constructor \AgdaInductiveConstructor{now} toma un valor \AgdaBound{x} \AgdaSymbol{:} \AgdaBound{A} y genera un valor de tipo \AgdaBound{A} \AgdaDatatype{$\bot$}. La expresión \AgdaInductiveConstructor{now} \AgdaBound{x} representa un programa que simplemente retorna el valor \AgdaBound{x} sin demoras. 
El constructor \AgdaInductiveConstructor{later} toma un \AgdaBound{x} de tipo \AgdaFunction{$\infty$} (\AgdaBound{A} \AgdaDatatype{$\bot$}), es decir un valor de tipo \AgdaBound{A} \AgdaDatatype{$\bot$} suspendido o demorado. Que el valor este suspendido o demorado implica que puede ser potencialmente infinito. El constructor \AgdaInductiveConstructor{later} retorna otro valor de tipo \AgdaBound{A} \AgdaDatatype{$\bot$}. Intuitivamente, lo que hace este constructor es ``agregar una demora'' al valor recibido. 

Un habitante del tipo \AgdaBound{A} \AgdaDatatype{$\bot$} puede ser una secuencia finita de constructores \AgdaInductiveConstructor{later} que finalmente retorna un valor \AgdaBound{x} \AgdaSymbol{:} \AgdaBound{A}, es decir algo del estilo \AgdaInductiveConstructor{later} (\AgdaFunction{$\sharp$} (\AgdaInductiveConstructor{later} (\AgdaFunction{$\sharp$} ... (\AgdaInductiveConstructor{now} \AgdaBound{x}) ...))); o puede ser una secuencia infinita de constructores \AgdaInductiveConstructor{later} que nunca retorna. Este último es el caso del valor \AgdaFunction{never} que se define a continuación. 

\ExecuteMetaData[latex/Delay.tex]{never}

El tipo \AgdaDatatype{\_$\bot$} viene con dos formas de igualdad (bisemejanza débil y fuerte) y una relación de orden. La bisemejanza fuerte es más fuerte que el orden y, a su vez, este último es más fuerte que la bisemejanza débil. Las tres relaciones se definen utilizando un único tipo \AgdaKeyword{data}, el cual estará indexado por un valor de tipo \AgdaDatatype{Kind} que indica qué tipo de relación es. Este último se define de la siguiente manera:

\ExecuteMetaData[latex/Delay.tex]{kind}

El constructor \AgdaInductiveConstructor{strong} representa la bisemejanza fuerte, mientras que \AgdaInductiveConstructor{other} \AgdaBound{k} representa la relación de orden si \AgdaBound{k} = \AgdaInductiveConstructor{geq}, o la bisemejanza débil si \AgdaBound{k} = \AgdaInductiveConstructor{weak}. La igualdad entre tipos de igualdad, es decir entre valores del tipo \AgdaDatatype{Kind}, es decidible. El operador \AgdaFunction{$\stackrel{?}{=}$-Kind} toma dos tipos de igualdad y decide si son iguales o no, dando a su vez una prueba de ello. 

\ExecuteMetaData[latex/Delay.tex]{deckind} 

Como se puede ver, la definición de esta función es muy sencilla. Para los casos en que ambos tipos son el mismo la prueba es simplemente \AgdaFunction{refl}, y en los casos en que no lo son, la prueba es el patrón absurdo, puesto que no hay manera de dar una prueba de igualdad entre ellos. Esta función sirve para definir un predicado que indica si la relación es de igualdad o no. Este predicado será verdadero para \AgdaInductiveConstructor{strong} y \AgdaInductiveConstructor{other weak}, pero no para \AgdaInductiveConstructor{other geq}. Será de utilidad tener este predicado a la hora de probar que las igualdades son relaciones de equivalencia.

\ExecuteMetaData[latex/Delay.tex]{equality}

Una vez introducido el tipo \AgdaDatatype{Kind}, se definen las relaciones propiamente dichas. Esto se realiza mediante la creación de un módulo llamado \AgdaModule{Equality}, el cual toma como parámetros un conjunto \AgdaBound{A} \AgdaSymbol{:} \AgdaPrimitiveType{Set}, que será el tipo de retorno, y una relación \AgdaBound{$\_\sim\_$} \AgdaSymbol{:} \AgdaBound{A} \AgdaSymbol{$\rightarrow$} \AgdaBound{A} \AgdaSymbol{$\rightarrow$} \AgdaPrimitiveType{Set} que establece una noción de igualdad entre valores de tipo \AgdaBound{A}. 

\ExecuteMetaData[latex/Delay.tex]{modequality}

Las relaciones se definen a través del tipo \AgdaDatatype{Rel}, el cual toma como parámetro un \AgdaDatatype{Kind} que indica qué tipo de relación es y dos valores de tipo \AgdaBound{A} \AgdaDatatype{$\bot$}, los cuales va a comparar mediante la relación correspondiente. El tipo \AgdaDatatype{Rel} tiene cuatro constructores:
\begin{itemize}[noitemsep]
\item El constructor \AgdaInductiveConstructor{now} indica que, si se tiene una prueba de que dos valores \AgdaBound{x} e \AgdaBound{y} de tipo \AgdaBound{A} están relacionados por la relación \AgdaBound{$\sim$}, es decir que son iguales en \AgdaBound{A}, entonces los términos (\AgdaInductiveConstructor{now} \AgdaBound{x}) y (\AgdaInductiveConstructor{now} \AgdaBound{y}) están relacionados en \AgdaBound{A} \AgdaDatatype{$\bot$} para los tres tipos de relación posibles. Esto está dado por la utilización de la variable \AgdaBound{k} que puede tomar cualquier valor de tipo \AgdaDatatype{Kind} y quiere decir que (\AgdaInductiveConstructor{now} \AgdaBound{x}) y (\AgdaInductiveConstructor{now} \AgdaBound{y}) son bisemejantes tanto débil como fuertemente, y que además (\AgdaInductiveConstructor{now} \AgdaBound{x}) es mayor o igual a (\AgdaInductiveConstructor{now} \AgdaBound{y}).

\item El constructor \AgdaInductiveConstructor{later} también sirve para las tres relaciones. En este caso, los \AgdaBound{x} e \AgdaBound{y} que recibe como parámetros implícitos son de tipo \AgdaFunction{$\infty$} \AgdaBound{A} \AgdaDatatype{$\bot$} (valores de tipo \AgdaBound{A} \AgdaDatatype{$\bot$} posiblemente infinitos). Este constructor pide una prueba (posiblemente infinita) de que (\AgdaField{$\flat$} \AgdaBound{x}) y (\AgdaField{$\flat$} \AgdaBound{y}) están relacionados por la relación \AgdaBound{k} y afirma que entonces (\AgdaInductiveConstructor{later} \AgdaBound{x}) y (\AgdaInductiveConstructor{later} \AgdaBound{y}) están relacionados por la misma relación \AgdaBound{k}.

\item El constructor \AgdaInductiveConstructor{later$^r$} se puede utilizar únicamente para bisemejanza débil. Dados \AgdaBound{x} \AgdaSymbol{:} \AgdaBound{A} \AgdaDatatype{$\bot$} e \AgdaBound{y} \AgdaSymbol{:} \AgdaFunction{$\infty$} \AgdaBound{A} \AgdaDatatype{$\bot$} que son parámetros implícitos y una prueba de que \AgdaBound{x} y (\AgdaField{$\flat$} \AgdaBound{y}) son débilmente bisemejantes, este constructor afirma que entonces \AgdaBound{x} y (\AgdaInductiveConstructor{later} \AgdaBound{y}) son débilmente bisemejantes también. Intuitivamente, \AgdaInductiveConstructor{later$^r$} permite agregar un constructor \AgdaInductiveConstructor{later} en el lado derecho de una bisemejanza débil.

\item El constructor \AgdaInductiveConstructor{later$^l$} es análogo a \AgdaInductiveConstructor{later$^r$} e, intuitivamente, permite agregar un constructor \AgdaInductiveConstructor{later} en el lado izquierdo de una relación que, en este caso, puede ser bisemejanza débil o desigualdad. Esto se evidencia en el uso de la variable \AgdaBound{k} que tiene tipo \AgdaDatatype{OtherKind} ya que luego en el tipo de relación que se le pasa a \AgdaDatatype{Rel} se utiliza (\AgdaInductiveConstructor{other} \AgdaBound{k}), quedando claro por qué al definir el tipo \AgdaDatatype{Kind} se separa el tipo \AgdaInductiveConstructor{strong} de los otros dos. 
\end{itemize}

En el uso que permite cada constructor se puede ver que la relación de tipo \AgdaInductiveConstructor{strong} es la más restrictiva, permitiendo únicamente agregar constructores \AgdaInductiveConstructor{later} a ambos lados de la igualdad, mientras que la desigualdad (\AgdaInductiveConstructor{other geq}) es un poco más débil ya que permite también agregar sólo a izquierda y la bisemejanza débil es la más débil de todas ya que permite agregar a ambos lados por separado.

Luego del tipo \AgdaDatatype{Rel} se definen los operadores \AgdaFunction{$\cong$}, \AgdaFunction{$\gtrsim$}, \AgdaFunction{$\lesssim$} y \AgdaFunction{$\approx$}, los cuales representan la bisemejanza fuerte, la desigualdad (mayor o igual), la desigualdad invertida (menor o igual) y la bisemejanza débil, respectivamente. 

A continuación de esto se definen otros operadores que tienen que ver con la noción de convergencia dada por la definición \ref{def:convergencia}. \AgdaBound{x} \AgdaFunction{$\Downarrow$[} \AgdaBound{k} \AgdaFunction{]} \AgdaBound{y} indica que \AgdaBound{x} se relaciona mediante la relación \AgdaBound{k} con (\AgdaInductiveConstructor{now} \AgdaBound{y}). En el caso especial de que la relación sea bisemejanza débil, se escribe \AgdaBound{x} \AgdaFunction{$\Downarrow$} \AgdaBound{y} y se dice que \AgdaBound{x} converge al valor \AgdaBound{y} \AgdaSymbol{:} \AgdaBound{A}. Por otro lado, \AgdaBound{x} \AgdaFunction{$\Downarrow$} indica que la computación \AgdaBound{x} termina, es decir que existe algún valor \AgdaBound{v} tal que \AgdaBound{x} \AgdaFunction{$\Downarrow$} \AgdaBound{v}. Por último, \AgdaBound{x} \AgdaFunction{$\Uparrow$[} \AgdaBound{k} \AgdaFunction{]} indica que \AgdaBound{x} se relaciona con el valor especial \AgdaFunction{never} mediante la relación \AgdaBound{k}. En el caso de que esa relación sea (\AgdaInductiveConstructor{other weak}) se escribe \AgdaBound{x} \AgdaFunction{$\Uparrow$} y se dice que la computación \AgdaBound{x} no termina.

Luego de dar todas las definiciones, se prueba que ambas bisemejanzas son relaciones de equivalencia y que la desigualdad es un orden parcial. Antes de poder demostrar esto, se prueba un conjunto de lemas que serán de utilidad para tal propósito. Todas estas pruebas se sitúan dentro de un módulo sin nombre parametrizado con el conjunto \AgdaBound{A} de retorno y su relación de igualdad. Esto se hace para fijar tal conjunto y su relación y poder utilizarlos a lo largo de todas las demostraciones. {\color{red} Vale la pena pegar todo el choclo de demostraciones y explicar una por una?}

\ExecuteMetaData[latex/Delay.tex]{lemas}




\section{Prueba de que el tipo \textit{delay} es una mónada}

\section{Prueba de que el tipo \textit{delay} es un funtor monoidal}

\section{¿Se puede probar que \textit{delay} es una mónada concurrente?}

\section{Reduciendo el problema a los conaturales}

\subsection{Definición de los conaturales con notación musical}

\subsection{¿Se puede probar que los conaturales forman un monoide concurrente?}

\section{Cambio de paradigma: \textit{sized types}}

\subsection{Definición de los conaturales utilizando \textit{sized types}}

\subsection{Prueba de que los conaturales forman un monoide concurrente}

\subsection{Prueba alternativa de que \textit{delay} es una mónada concurrente}