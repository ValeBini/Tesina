\documentclass[a4paper,spanish,fixlanguage]{book}
\usepackage[spanish]{babel}
\usepackage[utf8x]{inputenc}
\usepackage[T1]{fontenc}
\usepackage{amsmath,mathtools}    
\usepackage{stmaryrd}
\usepackage[dvipsnames]{xcolor}      %MidnightBlue color in links
\usepackage[resetlabels]{multibib}   %multiple bibliographies
\usepackage[margin=1.55in]{geometry} %page size
%\usepackage{cmbright}                %font
\usepackage{amsthm}                  %theorem styles
\usepackage[spanish]{babelbib}       %bibliografía en español
\usepackage{catchfilebetweentags}    %inclusión de pedazos de código
\usepackage{url}
\usepackage{float}                   %H option in figure
\usepackage[all]{xy}                 %categorical schemas
\usepackage{mathrsfs}
\usepackage{epigraph}
\usepackage{proof}                   %infer, used in exponential.tex
\usepackage{rotating}
\usepackage{latex/agda}
\usepackage[hidelinks]{hyperref}
%\usepackage{changepage}
\usepackage{bookmark}
\usepackage[header]{appendix}
\usepackage{dirtree}                 %directory tree of the appendix
\usepackage{sty/macros}              %macros
\usepackage{sty/unicode}             %unicode characters
\usepackage{sty/prefs}
\usepackage{enumitem} 				 %itemize setup
\usepackage{amssymb}
\usepackage{amsfonts}
\usepackage{dsfont}


%\newcites{Intro}{Referencias hist\'{o}ricas}
\newcommand{\gitcode}{\url{http://www.github.com/ValeBini/Tesina}}

\begin{document}

\hypersetup{pageanchor=false}
\begin{titlepage}  %Título

	\centering % Centrar
	\scshape % Font linda para las cosas chicas
	
	\textit{\large Facultad de Ciencias Exactas, Ingeniería y Agrimensura - UNR\\Departamento de Ciencias de la Computación }
	
	\vspace{2\baselineskip}
	
	%------------------------------------------------
	%	Título
	%------------------------------------------------
	
	\rule{\textwidth}{1.6pt}\vspace*{-\baselineskip}\vspace*{2pt} % Línea horizontal grande
	\rule{\textwidth}{0.4pt} % Línea horizontal suave
	
	\vspace{\baselineskip} % Espacio pre-título
	
	{\LARGE \textbf{Formalización de mónadas concurrentes en}}
	
	{\LARGE \textbf{Agda: el caso de la mónada
Delay}} % Título
	\vspace{\baselineskip} % Espacio post-título
	
	\rule{\textwidth}{0.4pt}\vspace*{-\baselineskip}\vspace{3.2pt} % Línea delgada
	\rule{\textwidth}{1.6pt} % Línea grande
	
	\vspace{3\baselineskip} % Espacio post título
	
	%------------------------------------------------
	%	Subtítulo
	%------------------------------------------------
	
	{\LARGE \textbf{Tesina de Grado}}

	{\Large Licenciatura en Ciencias de la Computación}
	\vspace*{3\baselineskip} % Espaciooo

	
	%------------------------------------------------
	%	Realizadores
	%------------------------------------------------
	
	\Large{\textbf{Autora:}}
	
	%\vspace{0.5\baselineskip} % Espacio
	
    {\itshape\LARGE Bini, Valentina María}
	
	%\vspace{0.5\baselineskip} % Espacio	
	
	%\large{Legajo: B-5926/9}
	
	\vspace{2\baselineskip} % Espacio

        \Large{\textbf{Director:}}
	
	%\vspace{0.5\baselineskip} % Espacio
	
    {\itshape\LARGE Rivas, Exequiel}

	\vspace*{6\baselineskip} %Espacio pre imagen
    
    %Imagen
    \begin{figure}[h] %El h es para que quede en el lugar
	\centering
	\includegraphics[width=0.4\linewidth]{img/UNR.png} \hfill \includegraphics[width=0.4\linewidth]{img/FCEIA.jpeg}
	\end{figure}
    
    \vfill % Espacio
	
	%------------------------------------------------
	%	Publicador o fecha
	%------------------------------------------------
		
	\vspace{0.2\baselineskip} % Espacio
	
	%13-12-2022 % Fecha
	

\end{titlepage}


%\thispagestyle{empty}

\chapter*{Resumen}

En los últimos años, la concurrencia ha cobrado mucha importancia en el mundo de la programación, sobre todo debido a la masificación de los procesadores con múltiples núcleos. Los lenguajes de programación funcional, en general, proveen la capacidad de concurrencia mediante funciones \textit{ad-hoc}, y no mediante primitivas bien fundadas del lenguaje. 

Los programas con efectos suelen representarse en los lenguajes de programación funcional mediante el uso de mónadas. Nace entonces el concepto de mónada concurrente en la búsqueda de obtener primitivas bien fundadas para la concurrencia en los lenguajes de programación funcional, extendiendo las mónadas con un nuevo operador: la intercalación de computaciones. La propiedad principal de las mónadas concurrentes es la ley de intercambio, axioma que postula la relación que debe existir entre la secuenciación de computaciones (el operador original \textit{bind}) y el nuevo operador. 
 
La mónada \textit{delay} fue definida con el objetivo de capturar el efecto de no terminación de programas de manera explícita y uniforme. Los habitantes del tipo \textit{delay} son valores ``demorados'', los cuales pueden no terminar y, por lo tanto, no retornar un valor nunca.

En este trabajo se presenta una formalización del concepto de mónada concurrente en el lenguaje y asistente de pruebas Agda, así como también otras formalizaciones de conceptos previos como las mónadas, los funtores monoidales y los monoides concurrentes. Luego se analiza el caso particular de la mónada \textit{delay}, con el objetivo de probar o refutar que esta puede dotarse de una estructura de mónada concurrente. La principal dificultad que se presenta a la hora de realizar esta prueba es la demostración de la ley de intercambio. Se busca entonces una simplificación del problema y se demuestra que los números conaturales forman un monoide concurrente, obteniendo luego una mónada concurrente alternativa a \textit{delay}: la mónada \textit{writer} con los conaturales como monoide.
%\thispagestyle{empty}

\frontmatter

\setcounter{tocdepth}{1}
\tableofcontents

\mainmatter

\chapter{Introducción}

En este primer capítulo se busca dar una introducción al trabajo, explicando la motivación del mismo y presentando también sus objetivos. Además se da una breve descripción de la estructura que tendrá la tesina y qué contenidos se incluyen en cada parte. 

\section{Motivación y estado del arte} 

La ley de Moore postula que el número de transistores que se pueden
poner en un chip de computadora se duplica (aproximadamente) cada un par
de años. De manera práctica, esto significa que para obtener
mejoras en la velocidad de ejecución de un programa simplemente hay
que esperar: después de un par de años, los programas que se escriben
hoy ejecutarán más rápido. Sin embargo, esta ley ya no se verifica en la práctica. Por razones físicas, es cada vez más costoso aumentar el número de transistores de
un chip y este tipo de progreso ya no puede ser garantizado. Por lo tanto, los
programadores se ven obligados a encontrar otras maneras de mejorar el
desempeño de sus programas. Una de las maneras más naturales que
surgió fue pensar que se pueden resolver varias tareas al mismo tiempo
y, de esa manera, evitar puntos ociosos en los que sólo se está esperando un evento del
entorno (por ejemplo, entrada/salida bloqueante). En los últimos años, con la
masificación de los microprocesadores con múltiples núcleos, se hizo aún
más evidente la necesidad de proveer al programador con la capacidad
de concurrencia.

Si bien los sistemas operativos modernos proveen primitivas para
manejar la concurrencia, no siempre los lenguajes de programación incorporan estas características de manera natural. Los lenguajes de
programación imperativos, en general, se basan en la idea de un modelo
de ejecución secuencial, donde la computación se desarrolla siguiendo
una única serie de pasos. Es así que lenguajes como C o Java pueden
manejar concurrencia pero sin proponer cambios radicales en la
concepción del lenguaje: simplemente incluyen librerías que capturan esta
capacidad de manera externa.

La misma situación se refleja en los lenguajes de programación
funcional, sobre todo cuando se escriben programas con
efectos. Los programas con efectos suelen representarse utilizando
mónadas, de manera que los programas funcionales toman una apariencia
fundamentalmente imperativa. El uso de mónadas para estructuras semánticas de lenguajes con efectos fue desarrollado originalmente por E. Moggi \cite{moggi:1989, moggi:1991}. Poco más tarde, P. Wadler \cite{wadler:1992} adaptó el
concepto de manera interna en los lenguajes de programación funcional,
dando origen a la programación con mónadas. Existe una gran variedad
de efectos que pueden ser capturados usando mónadas internas, por
ejemplo, estado, excepciones, entornos, continuaciones, etc. Al igual que en la programación imperativa, al considerar la concurrencia de los efectos, la
manera usual de propuesta es mediante llamadas a funciones \textit{ad-hoc} en lugar de usar primitivas bien fundadas que deriven de alguna estructura
matemática como lo hacen las operaciones de las mónadas.

Las mónadas concurrentes fueron recientemente introducidas en una
pre-impresión por M. Jaskelioff y E. Rivas \cite{rivas:2019}. Esta
definición surge de categorificar la noción de monoide concurrente,
definido por C. A. R. Hoare et al. \cite{hoare:2011} hace unos
años. Este concepto fue definido buscando obtener primitivas bien fundadas para la concurrencia, extendiendo las mónadas con nuevos operadores. Si bien esta estructura es
matemáticamente bien fundada, basada en monoides concurrentes, es
considerablemente complicado encontrar y probar modelos válidos de
esta estructura. 

La mónada Delay fue introducida por Capretta \cite{capretta:2005} con el
objetivo de capturar el efecto de no terminación de programas de manera explícita y
uniforme. Su estructura fue estudiada en varios artículos, entre los
que se puede mencionar la tesis de N. Veltri \cite{veltri:2017}. Una hipótesis particular es que la mónada \textit{delay} puede ser dotada de una estructura de mónada concurrente. Agda es un asistente de pruebas basado en teoría de tipos con soporte
para inducción-recursión que ya ha sido utilizado para estudiar la
mónada \textit{delay}.

\section{Objetivos del trabajo}

Esta tesina tiene dos objetivos principales. El primero de ellos es formalizar los conceptos de monoide concurrente y mónada concurrente en el ámbito del lenguaje de pruebas Agda. Se busca realizar esta formalización de manera completa, incluyendo dentro de las estructuras las pruebas de las propiedades necesarias para demostrar que los elementos que las conforman efectivamente cumplen con las leyes de cada estructura. 

El segundo es estudiar la mónada \textit{delay} y su implementación en Agda para luego adaptar sus operaciones a la formalización propuesta con el propósito principal de probar o refutar la hipótesis de que a la mónada \textit{delay} se le puede dar una estructura de mónada concurrente.

\section{Estructura de la tesina}

La tesina está dividida en dos partes. La primera, denominada Preliminares, contiene tres capítulos que exponen el marco teórico que fue utilizado para el desarrollo de este trabajo. En el primero de ellos se da una introducción al lenguaje Agda, describiendo sus principales características y mostrando ejemplos de uso. El segundo presenta una introducción teórica a las mónadas concurrentes, partiendo de la teoría de categorías sobre la cual se definen las mónadas y luego yendo a lo más específico hasta dar el concepto de mónada concurrente y sus principales características. Por último, en el tercero se introduce la noción de coinducción y la definición y propiedades de la mónada \textit{delay}. 

En la segunda parte, cuyo título coincide con el de la tesina, se pueden encontrar dos capítulos. Cada capítulo se corresponde con uno de los objetivos del trabajo. En el primero se da la formalización de diversas estructuras algebraicas, partiendo desde algunas más básicas como los monoides, las mónadas y los funtores monoidales para luego llegar a las más complejas que son la meta de este trabajo: los monoides concurrentes y las mónadas concurrentes. Para cada estructura se explica cada uno de los campos introducidos y se hace un paralelismo con el marco teórico para que quede clara la relación entre la teoría y la formalización. En el siguiente capítulo se analiza el caso de la mónada \textit{delay}. Primero se define el tipo \textit{delay} en Agda junto con varias funciones útiles para manipularlo. Luego se prueba que este puede dotarse de una estructura de mónada y también de funtor monoidal utilizando las formalizaciones definidas. Finalmente, se intenta probar o refutar que este puede dotarse de una estructura de mónada concurrente.

\section{Código fuente}

El código fuente del trabajo puede descargarse en el siguiente link: \href{https://github.com/ValeBini/Tesina/releases/tag/v1.0}{Código Tesina}. Dentro de la carpeta se encuentra un archivo llamado \texttt{README.agda} que contiene un módulo \AgdaModule{README} en el que se indica qué archivos se corresponden con cada parte del trabajo. La versión de Agda que se utilizó en el desarrollo de esta tesina es la v2.6.1. 

\part{Preliminares}\label{part:pre}

\chapter{Introducci\'on a Agda} \label{chapter:agda}

Agda es un lenguaje de programación funcional desarrollado inicialmente por Ulf Norell en la Universidad de Chalmers como parte de su tesis doctoral \cite{norell:thesis} que se caracteriza por tener tipos dependientes. A diferencia de otros lenguajes donde hay una separación clara entre el mundo de los tipos y el de los valores, en un lenguaje con tipos dependientes estos universos están más entremezclados. Los tipos pueden contener valores arbitrarios (lo que los hace \textit{depender} de ellos) y pueden aparecer también como argumentos o resultados de funciones.

El hecho de que los tipos puedan contener valores, permite que se puedan escribir propiedades de ciertos valores como tipos. Los elementos de estos tipos son pruebas de que la propiedad que representan es verdadera. Esto hace que los lenguajes con tipos dependientes puedan ser utilizados como una lógica. Esta fue la idea principal de la teoría de tipos desarrollada por Martin Löf, en la cual está basado el desarrollo de Agda. Una característica importante de esta teoría es su enfoque constructivista, en el cual para demostrar la existencia de un objeto debemos construirlo.

Para poder utilizar a Agda como una lógica se necesita que sea consistente, y es por eso que se requiere que todos los programas sean totales, es decir que no tienen permitido fallar o no terminar. En consecuencia, Agda incluye mecanismos que comprueban la terminación de los programas.

El objetivo de esta sección es presentar una introducción a Agda, haciendo énfasis en las características necesarias para exponer la temática de esta tesina. 

\section{Tipos de datos y \textit{Pattern Matching}}\label{agda:tipos}

Un concepto clave en Agda es el \textit{pattern matching} sobre tipos de datos algebraicos. Al agregar los tipos dependientes el \textit{pattern matching} se hace aún más poderoso. Se verá este tema más en detalle en la sección \ref{agda:dependent}. Para comenzar, en esta sección se describirán las funciones y tipos de datos con tipos simples. 

Los tipos de datos se definen utilizando una declaración \AgdaKeyword{data} en la que se especifica el nombre y el tipo del tipo de dato a definir, así como los constructores y sus respectivos tipos. En el siguiente bloque de código se puede ver una forma de definir el tipo de los booleanos:

\ExecuteMetaData[latex/Agda.tex]{booleans}

El tipo de \AgdaDatatype{Bool} es \AgdaPrimitiveType{Set}, el tipo de los tipos simples (se profundizará esto en la sección \ref{additional:univ}). Las funciones sobre \AgdaDatatype{Bool} pueden definirse por \textit{pattern matching}:

\ExecuteMetaData[latex/Agda.tex]{not}

Las funciones en Agda no tienen permitido fallar, por lo que una función debe cubrir todos los casos posibles. Esto será constatado por el \textit{type checker}, el cual lanzará un error si hay casos no definidos. 

Otro tipo de dato que puede ser útil son los números naturales. 

\ExecuteMetaData[latex/Agda.tex]{naturals}

La suma sobre números naturales puede ser definida como una función recursiva (también utilizando \textit{pattern matching}). 

\ExecuteMetaData[latex/Agda.tex]{add}

Para garantizar la terminación de la función, las llamadas recursivas deben ser aplicadas sobre argumentos más pequeños que los originales. En este caso, \AgdaFunction{\_+\_} pasa el chequeo de terminación ya que el primer argumento se hace más pequeño en la llamada recursiva. 

Si el nombre de una función contiene guiones bajos (\AgdaSymbol{\_}), entonces puede ser utilizado como un operador en el cual los argumentos se posicionan donde están los guiones bajos. En consecuencia, la función \AgdaFunction{\_+\_} puede ser utilizada como un operador infijo escribiendo \AgdaArgument{n} \AgdaFunction{+} \AgdaArgument{m} en lugar de \AgdaFunction{\_+\_} \AgdaArgument{n m}. 

La precedencia y asociatividad de un operador se definen utilizando una declaración \AgdaKeyword{infix}. Para mostrar esto se agregará, además de la suma, una función producto (la cual tiene más precedencia que la suma). La precedencia y asociatividad de ambas funciones podrían escribirse de la siguiente manera:

\ExecuteMetaData[latex/Agda.tex]{precedence}

La palabra clave \AgdaKeyword{infixl} indica que se asocia a izquierda (de igual manera existe \AgdaKeyword{infixr} para asociar a derecha o \AgdaKeyword{infix} si no se asocia hacia ningún lado) y el número que sigue indica la precedencia del operador, operadores con mayor número tendrán más precedencia que operadores con menor número.

\subsection{Tipos de datos parametrizados}\label{tipos:parametrized}

Los tipos de datos pueden estar parametrizados por otros tipos de datos. El tipo de las listas de elementos de tipo arbitrario se define de la siguiente manera:

\ExecuteMetaData[latex/Agda.tex]{lists}

En este ejemplo el tipo \AgdaDatatype{List} está parametrizado por el tipo \AgdaArgument{A}, el cual define el tipo de dato que tendrán los elementos de las listas. \AgdaDatatype{List} \AgdaArgument{$\mathbb{N}$} es el tipo de las listas de números naturales. 


\subsection{Patrones con puntos}\label{tipos:puntos}

En algunos casos, al definir una función por \textit{pattern matching}, ciertos patrones de un argumento fuerzan que otro argumento tenga un único valor posible que tipe correctamente. Para indicar que el valor de un argumento fue deducido por chequeo de tipos y no observado por \textit{pattern matching},  se le agrega delante un punto (\AgdaSymbol{.}). Para mostrar un ejemplo de uso de un patrón con punto, se considerará el siguiente tipo de dato \AgdaDatatype{Square} definido como sigue:

\ExecuteMetaData[latex/Agda.tex]{square}

El tipo \AgdaDatatype{Square} \AgdaArgument{n} representa una propiedad sobre el número \AgdaArgument{n}, la cual dice
dicho número es un cuadrado perfecto. Un habitante de tal tipo es una prueba de que el número \AgdaArgument{n} efectivamente es un cuadrado perfecto. Si se quisiera definir entonces una función \AgdaFunction{root} que tome un natural y una prueba de que dicho natural es un cuadrado perfecto, y devuelva su raíz cuadrada, podría realizarse de la siguiente manera:

\ExecuteMetaData[latex/Agda.tex]{root}

Se puede observar que al \textit{matchear} el argumento de tipo \AgdaDatatype{Square} \AgdaArgument{n} con el constructor \AgdaFunction{sq} aplicado a un natural \AgdaArgument{m}, \AgdaArgument{n} se ve forzado a ser igual a \AgdaArgument{m} \AgdaFunction{*} \AgdaArgument{m}.


\subsection{Patrones absurdos}\label{tipos:absurdos}

Otro tipo de patrón especial es el patrón absurdo. Usar un patrón absurdo en uno de los casos del \textit{pattern matching} al definir una función significa que no es necesario dar una definición para ese caso ya que no es posible dar un argumento para la función que caiga dentro de ese caso. El tipo de dato definido a continuación será de utilidad para ver un ejemplo de este tipo de patrones:

\ExecuteMetaData[latex/Agda.tex]{even}

El tipo \AgdaDatatype{Even} \AgdaArgument{n} representa, al igual que \AgdaDatatype{Square} \AgdaArgument{n}, una propiedad sobre \AgdaArgument{n}. En este caso la propiedad afirma que \AgdaArgument{n} es un número par. Un habitante de este tipo es una prueba de que dicha proposición se cumple. 

Si se quisiera definir una función que, dado un número y una prueba de que es par, devuelva el resultado de dividirlo por dos, podría realizarse de la siguiente manera:

\ExecuteMetaData[latex/Agda.tex]{half}

Se puede ver que en el caso del \AgdaNumber{1}, no existe una prueba de que ese número sea par, y por lo tanto no debemos dar una definición para ese caso. Requerir la pureba de paridad nos asegura que no hay riesgo de intentar dividir por dos un número impar. 

\subsection{El constructor \texttt{with}}\label{tipos:with}

A veces no alcanza con hacer \textit{pattern matching} sobre los argumentos de una función, sino que se necesita analizar por casos el resultado de alguna computación intermedia. Para esto se utiliza el constructor \AgdaKeyword{with}. 

Si se tiene una expresión \AgdaArgument{e} en la definición de una función \AgdaFunction{f}, se puede abstraer \AgdaFunction{f} sobre el valor de \AgdaArgument{e}. Al hacer esto se agrega a \AgdaFunction{f} un argumento extra, sobre el cual se puede hacer \textit{pattern matching} al igual que con cualquier otro argumento. 

Para proveer un ejemplo de uso del constructor \AgdaKeyword{with}, se definirá a continuación la relación de orden \AgdaFunction{\_<\_} sobre los números naturales.

\ExecuteMetaData[latex/Agda.tex]{less}

Si se quisiera definir entonces, utilizando esta función, una función \AgdaFunction{min} que calcule el mínimo entre dos números naturales \AgdaArgument{x} e \AgdaArgument{y}, se debería analizar cuál es el resultado de calcular \AgdaArgument{x} \AgdaFunction{<} \AgdaArgument{y}. Esto se escribe haciendo uso del constructor \AgdaKeyword{with} como sigue:

\ExecuteMetaData[latex/Agda.tex]{min}

El argumento extra que se agrega está separado por una barra vertical y corresponde al valor de la expresión \AgdaArgument{x} \AgdaFunction{<} \AgdaArgument{y}. Se puede realizar esta abstracción sobre varias expresiones a la vez, separándolas entre ellas mediante barras verticales. Las abstracciones \AgdaKeyword{with} también pueden anidarse. En el lado izquierdo de las ecuaciones, los argumentos abstraídos con \AgdaKeyword{with} deben estar separados también con barras verticales. 

En este caso, el valor que tome \AgdaArgument{x} \AgdaFunction{<} \AgdaArgument{y} no cambia nada la información que se tiene sobre los argumentos \AgdaArgument{x} e \AgdaArgument{y}, por lo que volver a escribirlos no es necesario, puede reemplazarse la parte izquierda por tres puntos como se muestra a continuación:

\ExecuteMetaData[latex/Agda.tex]{min2} 

\section{Tipos dependientes}\label{agda:dependent}

Como se mencionó anteriormente, una de las principales características de Agda es que tiene tipos dependientes. El tipo dependiente más básico de todos son las funciones dependientes, en las cuales el tipo del resultado depende del valor del argumento. En Agda se escribe \AgdaSymbol{(}\AgdaArgument{x} \AgdaSymbol{:} \AgdaDatatype{A}\AgdaSymbol{)} \AgdaSymbol{$\rightarrow$} \AgdaDatatype{B} para indicar el tipo de una función que toma un argumento \AgdaArgument{x} de tipo \AgdaDatatype{A} y devuelve un resultado de tipo \AgdaDatatype{B}, donde \AgdaArgument{x} puede aparecer en \AgdaDatatype{B}. Un caso especial de esto es cuando \AgdaArgument{x} es un tipo en sí mismo. Se podría definir, por ejemplo:

\ExecuteMetaData[latex/Agda.tex]{id}

\AgdaFunction{identity} es una función dependiente que toma como argumento un tipo \AgdaDatatype{A} y un elemento de \AgdaDatatype{A} y retorna dicho elemento. De esta manera se codifican las funciones polimórficas en Agda. 

A continuación se muestra un ejemplo de una función dependiente menos trivial, la cual toma una función dependiente y la aplica a cierto argumento:

\ExecuteMetaData[latex/Agda.tex]{apply}

Existen otros tipos dependientes además de las funciones. Uno muy utilizado son los pares dependientes, los cuales consisten en un par ordenado o tupla de dos elementos en la cual el tipo del segundo elemento depende del valor del primero. Estos pares son llamados usualmente \textit{$\Sigma$-types} (\textit{sigma types}) y pueden definirse de dos maneras. La primera, utilizando una declaración \AgdaKeyword{data}, se muestra a continuación. La otra se define utilizando un tipo \AgdaKeyword{record} y se verá en la sección \ref{records:dep}.

\ExecuteMetaData[latex/Agda.tex]{sigmadata}

Como se puede observar, el tipo del segundo elemento es \AgdaBound{B} y depende de un valor de tipo \AgdaBound{A} que se le pase como argumento. Para construir un par de este tipo se utiliza el constructor \AgdaInductiveConstructor{$\_$,$\_$}. Este toma un valor \AgdaBound{a} de tipo \AgdaBound{A} y luego un valor \AgdaBound{b} de tipo \AgdaBound{B a}, por lo cual queda evidenciado que el tipo del valor \AgdaBound{b} depende del valor de \AgdaBound{a}.

\subsection{Argumentos Implícitos}\label{dependent:implicit}

Los tipos dependientes sirven, entre otras cosas, para definir funciones polimórficas. En los ejemplos provistos en la sección anterior se da de forma explícita el tipo al cual cierta función polimórfica se debe aplicar. Usualmente esto es diferente. En general se espera que el tipo sobre el cual se va a aplicar una función polimórfica sea inferido por el \textit{type checker}. Para solucionar este problema, Agda utiliza un mecanismo de \textit{argumentos implícitos}. 

Para declarar un argumento implícito de una función, se utilizan llaves en lugar de paréntesis. \AgdaSymbol{\{}\AgdaArgument{x} \AgdaSymbol{:} \AgdaDatatype{A}\AgdaSymbol{\}} \AgdaSymbol{$\rightarrow$} \AgdaDatatype{B} significa lo mismo que \AgdaSymbol{(}\AgdaArgument{x} \AgdaSymbol{:} \AgdaDatatype{A}\AgdaSymbol{)} \AgdaSymbol{$\rightarrow$} \AgdaDatatype{B}, excepto que cuando se utiliza una función de este tipo el verificador de tipos intenta inferir el argumento por su cuenta. 

Con esta nueva sintaxis puede definirse una nueva versión de la función identidad, donde no es necesario explicitar el tipo argumento:

\ExecuteMetaData[latex/Agda.tex]{id2} 

Se puede observar que el tipo argumento es implícito tanto cuando la función se aplica como cuando es definida. No hay restricciones sobre cuáles o cuántos argumentos pueden ser implícitos, así como tampoco hay garantías de que estos puedan ser efectivamente inferidos por el \textit{type checker}. 

Para dar explícitamente un argumento implícito se usan también llaves. \AgdaFunction{f} \AgdaSymbol{\{}\AgdaArgument{v}\AgdaSymbol{\}} asigna \AgdaArgument{v} al primer argumento implícito de \AgdaFunction{f}. Si se requiere explicitar un argumento que no es el primero, se escribe \AgdaFunction{f} \AgdaSymbol{\{}\AgdaArgument{x} \AgdaSymbol{=} \AgdaArgument{v}\AgdaSymbol{\}}, lo cual asigna \AgdaArgument{v} al argumento implícito llamado \AgdaArgument{x}. El nombre de un argumento implícito se obtiene de la declaración del tipo de la función. 

Si se desea, por el contrario, que el verificador de tipos infiera un término que debería darse explícitamente, se puede reemplazar por un guión bajo. Por ejemplo:

\ExecuteMetaData[latex/Agda.tex]{one'}

\section{Familias de Tipos de datos}\label{agda:family}

Se definió en el apartado \ref{tipos:parametrized} el tipo de las listas de tipo arbitrario parametrizado por \AgdaArgument{A}. Estas listas pueden tener cualquier largo, tanto una lista vacía como una lista con un millón de elementos son de tipo \AgdaDatatype{List} \AgdaArgument{A}. En ciertos casos es útil que el tipo restrinja el largo que tiene la lista, y es así como surgen las listas de largo definido, llamadas comúnmente vectores, que se definen como sigue:

\ExecuteMetaData[latex/Agda.tex]{vec}

El tipo de \AgdaDatatype{Vec} \AgdaArgument{A} es \AgdaDatatype{$\mathbb{N}$} \AgdaSymbol{$\rightarrow$} \AgdaPrimitiveType{Set}. Esto significa que \AgdaDatatype{Vec} \AgdaArgument{A} es una familia de tipos indexada por los números naturales. Por lo tanto, para cada \AgdaArgument{n} natural, \AgdaDatatype{Vec} \AgdaArgument{A n} es un tipo. Los constructores pueden construir elementos de cualquier tipo dentro de la familia. Hay una diferencia sustancial entre parámetros e índices de un tipo de dato. Se dice que \AgdaDatatype{Vec} está parametrizado por un tipo \AgdaArgument{A} e indexado sobre los números naturales. 

En el tipo del constructor \AgdaFunction{\_::\_} se puede observar un ejemplo de una función dependiente. El primer argumento del constructor es un número natural \AgdaArgument{n} implícito, el cual es el largo de la cola. Es seguro poner \AgdaArgument{n} como argumento implícito ya que el verificador de tipos siempre podrá inferirlo en base al tipo del tercer argumento. 

Lo que tienen de interesante las familias de tipos es lo que sucede cuando se usa \textit{pattern matching} sobre sus elementos. Si se quisiera definir una función que devuelva la cabeza de una lista no vacía, el tipo \AgdaDatatype{Vec} permite expresar el tipo de las listas no vacías, lo cual hace posible definir la función \AgdaFunction{head} de manera segura como se muestra a continuación: 

\ExecuteMetaData[latex/Agda.tex]{head}

La definición es aceptada por el verificador de tipos ya que, aunque no se da un caso para la lista vacía, es exhaustiva. Esto es gracias a que un elemento del tipo \mbox{\AgdaDatatype{Vec} \AgdaArgument{A} \AgdaSymbol{(}\AgdaFunction{suc} \AgdaArgument{n}\AgdaSymbol{)}} sólo puede ser construído por el constructor \AgdaFunction{\_::\_}, y resulta útil ya que la función \AgdaFunction{head} no está correctamente definida para el caso de la lista vacía.  

En algunos casos puede ser necesario dar un único tipo que englobe a todas las listas de todos los largos posibles. Esto puede expresarse mediante la unión disjunta de los vectores de cada posible largo. Expresar la unión disjunta de familias de tipos de datos es un uso muy común de los $\Sigma$\textit{-types}. Escribir \AgdaDatatype{$\Sigma'$ $\mathbb{N}$} \AgdaSymbol{(}\AgdaDatatype{Vec} \AgdaBound{A}\AgdaSymbol{)} (o lo que es equivalente, \AgdaDatatype{$\Sigma'$ $\mathbb{N}$} \AgdaSymbol{($\lambda$} \AgdaBound{n} \AgdaSymbol{$\rightarrow$} \AgdaDatatype{Vec} \AgdaBound{A n}\AgdaSymbol{)}) es conceptualmente lo mismo que escribir \AgdaDatatype{List} \AgdaBound{A}. 

Un ejemplo en el cual es necesaria la unión disjunta de todos los vectores es para definir la función \AgdaFunction{filterVec} que, dado un vector de cierto largo \AgdaBound{n} y un predicado, filtra todos los elementos del vector que cumplen la condición dada. En este caso no es posible saber de antemano qué largo tendrá el vector resultante, puesto que se desconoce cuántos elementos del vector original cumplirán la condición. Es por esto que se define como sigue:

\ExecuteMetaData[latex/Agda.tex]{filtervec} 

\section{Sistema de Módulos}\label{agda:modules}

El objetivo del sistema de módulos de Agda es manejar el espacio de nombres. Un programa se estructura en diversos archivos, cada uno de los cuales tiene un módulo \textit{top-level}, dentro del cual van todas las definiciones. El nombre del módulo principal de un archivo debe coincidir con el nombre de dicho archivo. Si se tiene, por ejemplo, un archivo llamado \texttt{Agda.agda}, al comienzo del archivo se debería encontrar la siguiente línea:

\ExecuteMetaData[latex/Agda.tex]{module} 

Dentro del módulo principal se pueden definir otros módulos. Esto se hace de la misma manera que se define el módulo \textit{top-level}. Por ejemplo:

\ExecuteMetaData[latex/Agda.tex]{moduleNumbers}

Para acceder a entidades definidas en otro módulo hay que anteponer al nombre de la entidad el nombre del módulo en el cual está definida. Para hacer referencia a \AgdaDatatype{Nat} desde fuera del módulo \AgdaModule{Numbers} se debe escribir \AgdaModule{Numbers.Nat}:


\ExecuteMetaData[latex/Agda.tex]{one}


La extensión de los módulos (excepto el módulo principal) se determina por indentación. Si se quiere hacer referencia a las definiciones de un módulo sin anteponer el nombre del módulo constantemente se puede utilizar la sentencia \AgdaKeyword{open}, tanto localmente como en \textit{top-level}:


\ExecuteMetaData[latex/Agda.tex]{two}


\ExecuteMetaData[latex/Agda.tex]{two2}


Al abrir un módulo, se puede controlar qué definiciones se muestran y cuáles no, así como también cambiar el nombre de algunas de ellas. Para esto se utilizan las palabras clave \AgdaKeyword{using} (para restringir cuáles definiciones traer), \AgdaKeyword{hiding} (para esconder ciertas definiciones) y \AgdaKeyword{renaming} (para cambiarles el nombre). Si se quisiera abrir el módulo \AgdaModule{Numbers} ocultando la función \AgdaFunction{suc₂} y cambiando los nombres del tipo y los constructores, se debería escribir:

\ExecuteMetaData[latex/Agda.tex]{hidingrenaming}

\subsection{Módulos Parametrizados}\label{modules:parametrized}

Los módulos pueden ser parametrizados por cualquier tipo de dato. En caso de que se quiera definir un módulo para ordenar listas, por ejemplo, puede ser conveniente asumir que las listas son de tipo $A$ y que tenemos una relación de orden sobre $A$. A continuación se presenta dicho ejemplo: 


\ExecuteMetaData[latex/Agda.tex]{sort}


Cuando se mira desde afuera una función definida dentro de un módulo parametrizado, la función toma como argumentos, además de los propios, los parámetros del módulo. De esta manera se podría definir:


\ExecuteMetaData[latex/Agda.tex]{sort1}


También pueden aplicarse todas las funciones de un módulo parametrizado a los parámetros del módulo de una vez instanciando el módulo de la siguiente manera: 


\ExecuteMetaData[latex/Agda.tex]{sortnat}


Esto crea el módulo \AgdaModule{SortNat} que contiene las funciones \AgdaFunction{insert} y \AgdaFunction{sort}, las cuales ya no tienen como argumentos los parámetros del módulo \AgdaModule{Sort}, sino que directamente trabajan con naturales y la relación sobre naturales \AgdaFunction{<}. 


\ExecuteMetaData[latex/Agda.tex]{sort2}


También se puede instanciar el módulo y abrir directamente el módulo resultante sin darle un nuevo nombre, lo cual se escribe de forma simplificada como sigue:


\ExecuteMetaData[latex/Agda.tex]{opensortnat}


\subsection{Importando módulos desde otros archivos}\label{modules:files}

Se describió hasta ahora la forma de utilizar diferentes módulos dentro de un archivo, el cual tiene siempre un módulo principal. Muchas veces, sin embargo, los programas se dividen en diversos archivos y uno se ve en la necesidad de utilizar un módulo definido en un archivo diferente al actual. Cuando esto sucede, se debe \textit{importar} el módulo correspondiente.  

Los módulos se importan por nombre. Si se tiene un módulo \AgdaModule{A.B.C} en un archivo en la dirección \texttt{/alguna/direccion/local/A/B/C.agda}, este se importa con la sentencia \AgdaKeyword{import} \AgdaModule{A.B.C}. Para que el sistema pueda encontrar el archivo, \texttt{/alguna/direccion/local} debe estar en el \textit{path} de búsqueda de Agda. 

Al importar módulos se pueden utilizar las mismas palabras claves de control de espacio de nombres que al abrir un módulo (\AgdaKeyword{using}, \AgdaKeyword{hiding} y \AgdaKeyword{renaming}). Importar un módulo, sin embargo, no lo abre automáticamente. Se puede abrir de forma separada con una sentencia \AgdaKeyword{open} o usar la forma corta \AgdaKeyword{open import} \AgdaModule{A.B.C}.

\section{Records}\label{agda:records}

Un tipo \AgdaKeyword{record} se define de forma similar a un tipo \AgdaKeyword{data}, donde en lugar de constructores se tienen campos, los cuales son provistos por la palabra clave \AgdaKeyword{field}. Por ejemplo: 

\ExecuteMetaData[latex/Agda.tex]{point}

Esto declara el registro \AgdaRecord{Point} con dos campos naturales \AgdaArgument{x} e \AgdaArgument{y}. Para construir un elemento de \AgdaRecord{Point} se escribe:

\ExecuteMetaData[latex/Agda.tex]{mkpoint}

Si antes de la palabra clave \AgdaKeyword{field} se agrega la palabra clave \AgdaKeyword{constructor}, se puede dar un constructor específico para el registro, el cual permite construir de manera simplificada un elemento del mismo. 

\ExecuteMetaData[latex/Agda.tex]{point2}

Para poder extraer los campos de un \AgdaKeyword{record}, cada tipo \AgdaKeyword{record} viene con un módulo con el mismo nombre. Este módulo está parametrizado por un habitante del tipo y contiene funciones de proyección para cada uno de los campos. En el ejemplo de \AgdaRecord{Point} se obtiene el siguiente módulo:

\ExecuteMetaData[latex/Agda.tex]{modpoint}

Este módulo puede utilizarse como viene o puede instanciarse a un registro en particular. 

\ExecuteMetaData[latex/Agda.tex]{get}

Es posible agregar funciones al módulo de un \AgdaKeyword{record} incluyéndolas en la declaración del mismo luego de los campos. 

\ExecuteMetaData[latex/Agda.tex]{monad}

Como se puede ver en este ejemplo, los tipos \AgdaKeyword{record} pueden ser, al igual que los tipos \AgdaKeyword{data}, parametrizados. En este caso, el \AgdaKeyword{record} \AgdaRecord{Monad} está parametrizado por \AgdaBound{M}. Cuando un \AgdaKeyword{record} está parametrizado, el módulo generado por él tiene los parámetros del \AgdaKeyword{record} como parámetros implícitos.

\subsection{Campos con tipos dependientes}\label{records:dep}

A la hora de definir un \AgdaKeyword{record}, el tipo de un campo puede depender de los valores de todos los campos anteriores. Esto hace que el orden en que se introducen los campos no siempre pueda ser arbitrario. A continuación se muestra la definición de los $\Sigma$\textit{-types} como un tipo \AgdaKeyword{record}, en la cual el tipo del segundo campo depende del valor del primero.

\ExecuteMetaData[latex/Agda.tex]{sigmarecord}

\section{Características adicionales}\label{agda:additional}

Antes de finalizar este capítulo, se describirán algunas características adicionales específicas de Agda que son importantes para comprender la potencia del lenguaje.

\subsection{Universos}\label{additional:univ}

La paradoja de Russell implica que la colección de todos los conjuntos no es en sí misma un conjunto. Si existiera tal conjunto $U$, entonces uno podría formar el subconjunto $A \subseteq U$ de todos los conjuntos que no se contienen a sí mismos. Luego se deduciría que $A \in A \iff A \notin A$, lo cual es una contradicción.

Por razones similares, no todos los tipos de Agda son de tipo \AgdaPrimitiveType{Set}. Por ejemplo, se tiene que \AgdaDatatype{Bool} \AgdaSymbol{:} \AgdaPrimitiveType{Set} y \AgdaDatatype{Nat} \AgdaSymbol{:} \AgdaPrimitiveType{Set}, pero no es cierto que \AgdaPrimitiveType{Set} \AgdaSymbol{:} \AgdaPrimitiveType{Set}. Sin embargo, es necesario y conveniente que \AgdaPrimitiveType{Set} tenga un tipo, es por eso que en Agda se le da el tipo \AgdaPrimitiveType{Set$_1$}:

\AgdaPrimitiveType{Set} \AgdaSymbol{:} \AgdaPrimitiveType{Set$_1$}

Las expresiones de tipo \AgdaPrimitiveType{Set$_1$} se comportan en gran medida como las de tipo \AgdaPrimitiveType{Set}, por ejemplo, pueden ser utilizadas como tipo de otras cosas. Sin embargo, los habitantes de \AgdaPrimitiveType{Set$_1$} son potencialmente \textit{más grandes}. Cuando se tiene \AgdaDatatype{A} \AgdaSymbol{:} \AgdaPrimitiveType{Set$_1$}, entonces se dice a veces que \AgdaDatatype{A} es un \textit{conjunto grande}. Sucesivamente, se tiene que:

\AgdaPrimitiveType{Set$_1$} \AgdaSymbol{:} \AgdaPrimitiveType{Set$_2$} 

\AgdaPrimitiveType{Set$_2$} \AgdaSymbol{:} \AgdaPrimitiveType{Set$_3$}

etcétera. Un tipo cuyos habitantes son tipos se llama \textbf{universo}. Agda provee un número infinito de universos \AgdaPrimitiveType{Set}, \AgdaPrimitiveType{Set$_1$}, \AgdaPrimitiveType{Set$_2$}, \AgdaPrimitiveType{Set$_3$}, ..., cada uno de los cuales es un habitante del siguiente. \AgdaPrimitiveType{Set} es en sí mismo una abreviación de \AgdaPrimitiveType{Set$_0$}. El subíndice \AgdaDatatype{$n$} es el \textbf{nivel} del universo \AgdaPrimitiveType{Set$_n$}. Agda provee también un tipo primitivo especial \AgdaPrimitiveType{Level}, cuyos habitantes son los posibles niveles de los universos. De hecho, la notación \AgdaPrimitiveType{Set$_n$} es una abreviación para \AgdaPrimitiveType{Set $n$}, donde \AgdaDatatype{$n$} \AgdaSymbol{:} \AgdaPrimitiveType{Level}. 

Si bien no hay un número de niveles específico, se sabe que existe un nivel más bajo \AgdaKeyword{lzero}, y que para cada nivel \AgdaDatatype{$n$} existe algún nivel mayor \AgdaKeyword{lsuc} \AgdaDatatype{$n$}. Por lo tanto, el conjunto de niveles es infinito. Además, puede tomarse la cota superior mínima (o supremo) \AgdaDatatype{$n\  \sqcup \ m$} de dos niveles. En resumen, las siguientes operaciones son las únicas operaciones que Agda provee sobre niveles:

\AgdaKeyword{lzero} \AgdaSymbol{:} \AgdaPrimitiveType{Level}

\AgdaKeyword{lsuc} \AgdaSymbol{:} \AgdaSymbol{(}\AgdaDatatype{$n$} : \AgdaPrimitiveType{Level}\AgdaSymbol{)} \AgdaSymbol{$\rightarrow$} \AgdaPrimitiveType{Level} 

\AgdaFunction{$\_\sqcup\_$} \AgdaSymbol{:} \AgdaSymbol{(}\AgdaDatatype{$n$ $m$} \AgdaSymbol{:} \AgdaPrimitiveType{Level}\AgdaSymbol{)} \AgdaSymbol{$\rightarrow$} \AgdaPrimitiveType{Level}


\subsection{Inducción-Recursión}\label{additional:ind-rec}

Una característica fundamental de Agda que la distingue de otros lenguajes similares es el soporte para \textit{inducción-recursión} (en inglés \textit{induction-recursion}). En la teoría de tipos intuicionista, la inducción-recursión es una propiedad que permite declarar simultáneamente un tipo y una función sobre dicho tipo, haciendo posible la creación de tipos más grandes que los tipos inductivos como, por ejemplo, los universos. En una definición inductiva, se dan reglas para generar habitantes de un tipo y luego pueden definirse funciones de ese tipo por inducción en dichos habitantes. En inducción-recursión se permite que las reglas que generan los habitantes de un tipo hagan referencia a la función que a su vez es definida por inducción en los habitantes del tipo. A continuación se muestra un ejemplo de una definición inductiva-recursiva para ilustrar mejor esta característica. 

\ExecuteMetaData[latex/Agda.tex]{ind-rec}

Como se ve en el ejemplo, la manera de hacer una declaración inductiva-recursiva es mediante la palabra clave \AgdaKeyword{mutual}. Esta palabra clave puede utilizarse también para realizar definiciones de funciones mutuamente recursivas. 

\ExecuteMetaData[latex/Agda.tex]{mutual}

\subsection{Coinducción}\label{agda:coinduction}

Agda tiene varios soportes diferentes para coinducción. Se describirán algunos de ellos más adelante en la sección \ref{coind:agda}, junto con sus características principales y su utilidad.
\chapter{M\'onadas Concurrentes}\label{chapter:monconc}

En este capítulo se hará una introducción teórica sobre mónadas concurrentes. Se definirán al comienzo algunos conceptos previos de la teoría de categorías sobre la cual se definen las mónadas. En la segunda sección se hará una introducción a las mónadas en sí mismas, en la cual se darán varias definiciones y ejemplos. Por último, en la tercera, se introducirán los conceptos específicos necesarios para definir una mónada concurrente. 

\section{Teoría de Categorías}\label{monconc:cat}

La teoría de categorías fue desarrollada por Eilenberg y MacLane \cite{eilenberg:1945} en 1945. Esta teoría busca axiomatizar de forma abstracta diversas estructuras matemáticas como una sola, tales como los grupos y los espacios topológicos, mediante el uso de objetos y morfismos. A su vez, esta axiomatización se realiza de una manera nueva sin incluir las nociones de elemento o pertenencia, es decir, sin utilizar conjuntos. Con el concepto de categoría se pretende capturar la esencia de una clase de objetos matemáticos que se relacionan entre sí mediante aplicaciones, poniendo énfasis en la relación entre los objetos y no en la pertenencia como en la teoría de conjuntos.

\begin{definition}[Categoría]
Una \textbf{categoría} $\mathscr{C}$ consiste de:
\begin{itemize}[noitemsep,label=$\blacktriangleright$]
	\item una clase de \textbf{objetos}: $\mathbf{ob} \ \mathscr{C}$;
	\item una clase de \textbf{morfismos} o \textbf{flechas}: $\mathbf{mor} \ \mathscr{C}$;
	\item dos funciones de clase:
	\begin{itemize}[noitemsep,label=$\bullet$]
		\item $dom : \mathbf{mor} \ \mathscr{C} \rightarrow \mathbf{ob} \ \mathscr{C}$ (dominio),
		\item $codom : \mathbf{mor} \ \mathscr{C} \rightarrow \mathbf{ob} \ \mathscr{C}$ (codominio).
	\end{itemize}
	Para cada par de objetos $A, B$ en $\mathbf{ob} \ \mathscr{C}$ se denomina $Hom(A,B)$ al conjunto de flechas o morfismos de $A$ a $B$, es decir:
	\begin{equation*}
		Hom(A,B) := \{f \in \mathbf{mor} \ \mathscr{C} : dom(f) = A, codom(f) = B\}
	\end{equation*}
	\item Y para cada $A, B, C \in \mathbf{ob} \ \mathscr{C}$ una operación  
	\begin{equation*}
		\circ : Hom(A,B) \times Hom(B,C) \rightarrow Hom(A,C)
	\end{equation*}
	llamada \textbf{composición} con las siguientes propiedades: 
	\begin{itemize}[noitemsep,label=$\bullet$]
		\item Se denota $\circ(f,g) = g \circ f$.
		\item \textbf{Asociatividad}: para cada $A,B,C,D \in \mathbf{ob} \ \mathscr{C}$ y $f,g,h \in \mathbf{mor} \ \mathscr{C}$ tales que $f \in Hom(A,B)$, $g \in Hom(B,C)$ y $h \in Hom(C,D)$, \ \ $h \circ (g \circ f) = (h \circ g) \circ f$.
		\item Para cada $A \in \mathbf{ob} \ \mathscr{C}$ existe un \textbf{morfismo identidad} $id_A \in Hom(A,A)$ tal que
		\begin{itemize}[noitemsep,label=$\star$]
			\item $\forall B, \ \forall f \in Hom(A,B), \ f \circ id_A = f$,
			\item $\forall C, \ \forall g \in Hom(C,A), \ id_A \circ g = g$.
		\end{itemize}
	\end{itemize}
	
\end{itemize}
\end{definition}

A continuación se presentan algunos ejemplos de categorías que pueden ser de utilidad para comprender mejor el concepto.

\begin{ejemplo}[Categoría \textbf{Set}]
La categoría \textbf{Set} es aquella tal que:
\begin{itemize}[noitemsep,label=$\blacktriangleright$]
	\item $\text{\bf ob Set} = \text{conjuntos}$
	\item $\text{\bf mor Set} = \text{funciones}$. 
\end{itemize}
\end{ejemplo}
\begin{ejemplo}[Categor\'ia $\mathds{1}$]
La categoría $\mathds{1}$ es aquella tal que:
\begin{itemize}[noitemsep,label=$\blacktriangleright$]
	\item $\text{\bf ob} \ \mathds{1} = \{\star\}$
	\item $\text{\bf mor} \ \mathds{1} = \{\text{id}_{\star}\}$. 
\end{itemize}
\end{ejemplo}

Dentro de los objetos de una categoría, hay dos clases especiales de objetos: iniciales y terminales. Estos se definen como sigue:

\begin{definition}[Objetos iniciales y terminales]
Un objeto $\mathbf{0} \in \text{\bf ob} \ \mathscr{C}$ se dice \textbf{inicial} si $\forall A \in \text{\bf ob} \ \mathscr{C}$, $\exists! \mathbf{0} \rightarrow A$. 
Un objeto $\mathbf{1} \in \text{\bf ob} \ \mathscr{C}$ se dice \textbf{terminal} si $\forall A \in \text{\bf ob} \ \mathscr{C}$, $\exists! A \rightarrow \mathbf{1}$.
\end{definition}

\begin{ejemplo} 
En \textbf{Set}, $\emptyset$ es el único objeto inicial y los conjuntos de un elemento $\{x\}$ son los objetos terminales.
\end{ejemplo}

En la categoría \textbf{Set}, se sabe que el producto cartesiano entre dos objetos (conjuntos) $A \times B$ es el conjunto de los pares $(a,b)$ tales que $a \in A$ y $b \in B$. Para definir el concepto de producto cartesiano entre dos objetos $A$ y $B$ de una categoría cualquiera, es necesario caracterizar a $A \times B$ sin hacer referencia a sus elementos. 

\begin{definition}[Producto]
El \textbf{producto} de dos objetos $A$ y $B$ en una categoría $\mathscr{C}$ es una terna $(A \times B, \pi_A, \pi_B)$ donde:
\begin{itemize}[label=$\blacktriangleright$]
	\item $\pi_A \in Hom(A \times B, A)$,
	\item $\pi_B \in Hom(A \times B, B)$
	\item y para todo objeto $C$ y para todo par de morfismos $f : C \rightarrow A$, $g : C \rightarrow B$, existe un único mofismo $\langle f, g \rangle : C \rightarrow A \times B$ tal que:
	\begin{itemize}[label=$\bullet$]
		\item $f = \pi_A \circ \langle f, g \rangle$
		\item $g = \pi_B \circ \langle f, g \rangle$
	\end{itemize}
\end{itemize}
\end{definition}

Suponiendo que existen los productos $A \times B$ y $C \times D$ y que se tienen dos morfismos $f : A \rightarrow C$ y $g : B \rightarrow D$, se puede definir un morfismo $f \times g : A \times B \rightarrow C \times D$ tal que $f \times g = \langle f \circ \pi_A , g \circ \pi_B \rangle$.

De forma dual a la definición de producto, se puede definir la noción de coproducto entre dos objetos $A$ y $B$ de una categoría arbitraria como sigue:

\begin{definition}[Coproducto]
El \textbf{coproducto} de dos objetos $A, B$ de una categoría $\mathscr{C}$ es una terna $(A + B, \iota_A, \iota_B)$ donde:
\begin{itemize}[label=$\blacktriangleright$]
	\item $\iota_A \in Hom(A, A + B)$,
	\item $\iota_B \in Hom(B, A + B)$
	\item y para todo objeto $C$ y para todo par de morfismos $f : A \rightarrow C$, $g : B \rightarrow C$ existe un único morfismo $[f,g] : A + B \rightarrow C$ tal que se cumplen las siguientes ecuaciones:
	\begin{itemize}[label=$\bullet$]
		\item $f = [f,g] \circ \iota_A$
		\item $g = [f,g] \circ \iota_B$
	\end{itemize}
\end{itemize}
\end{definition}

Una última relación que puede establecerse entre dos objetos de una categoría es el exponencial. En \textbf{Set}, el exponencial de dos conjuntos $A$ y $B$ es el conjunto $B^A$ de todas las funciones que van de $A$ en $B$, es decir que toman un elemento de $A$ y devuelven un elemento de $B$. Esta noción puede generalizarse a una categoría arbitraria que tenga productos cartesianos.

\begin{definition}[Exponencial]
Sea $\mathscr{C}$ una categoría con productos binarios y sean $A,B \in \text{\bf ob} \ \mathscr{C}$. Un objeto $B^A$ es un \textbf{exponencial} si existe un morfismo $\varepsilon : B^A \times A \rightarrow B$ tal que para todo morfismo $g : C \times A \rightarrow B$ existe un único morfismo $\tilde{g} : C \rightarrow B^A$ tal que $g = \varepsilon \circ (\tilde{g} \times id_A)$.
\end{definition}

Si se quisiera construir una categoría cuyos objetos son categorías, se necesitaría contar con morfismos entre categorías. Estos existen y se llaman funtores, son en cierta manera una generalización del concepto de función de conjuntos para categorías. Un funtor permite construir una nueva categoría a partir de otra dada.
 
\begin{definition}[Funtor]
Sean $\mathscr{C}$ y $\mathscr{D}$ dos categorías. Un \textbf{funtor} $\mathit{F} : \mathscr{C} \rightarrow \mathscr{D}$ asigna:
\begin{itemize}[noitemsep,label=$\blacktriangleright$]
	\item a cada objeto $A \in \mathbf{ob} \ \mathscr{C}$, un objeto $\mathit{F}(A) \in \mathbf{ob} \ \mathscr{D}$;
	\item a cada morfismo $f : A \rightarrow B \in \mathbf{mor} \ \mathscr{C}$, un morfismo $\mathit{F}(f) : \mathit{F}(A) \rightarrow \mathit{F}(B) \in \mathbf{mor} \ \mathscr{D}$ tal que: 
	\begin{itemize}[noitemsep,label=$\bullet$]
		\item para todo $A \in \mathbf{ob} \ \mathscr{C}$, $\mathit{F}(id_A) = id_{\mathit{F}(A)}$;
		\item para todos $f,g \in \mathbf{mor} \ \mathscr{C}$ tales que tenga sentido la composición $g \circ f$, se tiene que $\mathit{F}(g \circ f) = \mathit{F}(g) \circ \mathit{F}(f)$.
	\end{itemize}
\end{itemize}
\end{definition}

Se dice que un funtor es un \textbf{endofuntor} si la categoría de salida y la de llegada son la misma, es decir, $\mathit{F} : \mathscr{C} \rightarrow \mathscr{C}$.

Siguiendo con la misma lógica, uno podría construir morfismos entre funtores. Es decir, algún tipo de construcción matemática que lleve de un funtor dado a otro. Este concepto se denomina transformación natural y se define como sigue:

\begin{definition}[Transformaci\'on Natural]
Sean $\mathit{F}, \mathit{G} : \mathscr{C} \rightarrow \mathscr{D}$ dos funtores (entre las mismas categorías). Una \textbf{transformación natural} $\eta : \mathit{F} \rightarrow \mathit{G}$ asigna a cada $A \in \mathbf{ob} \ \mathscr{C}$ un morfismo $\eta_A : \mathit{F}(A) \rightarrow \mathit{G}(A)$ tal que para todo $f \in Hom(A,B)$ se cumple que: 
\begin{equation*}
	\eta_B \circ \mathit{F}(f) = \mathit{G}(f) \circ \eta_A 
\end{equation*}
\end{definition}

Para cerrar la sección, se introducirá la noción de monoide, Los monoides son un tipo de estructura algebraica abstracta introducida por primera vez por Arthur Cayley.

\begin{definition}[Monoide]\label{def:monoid}
Un \textbf{monoide} es un conjunto $M$ dotado de una operación asociativa $M \times M \rightarrow M$, $(m,n) \rightarrow mn$ tal que existe un elemento neutro:
\begin{equation*}
\exists e \in M, \forall m \in M, (em = me = m).
\end{equation*}
\end{definition}

El elemento neutro de un monoide es único. Por esa razón, en general el elemento neutro es considerado una constante, es decir, una operación 0-aria (sin argumentos). Se utilizará esta representación en la formalización de los monoides.

\section{Introducción a las mónadas}\label{monconc:monadas}

Se considerarán dos variantes de la definición de mónadas. La primera es la definición clásica y la segunda define a una mónada como un sistema de extensión o 3-tupla Kleisli. La primera es muy utilizada en la literatura ya que es la definición matemática y está definida en torno a transformaciones naturales, pero la segunda es más fácil de utilizar desde una perspectiva computacional. Como ambas definiciones son equivalentes \cite{moggi:1991}, se utilizará una u otra según sea conveniente.

\subsection{Definición clásica de Mónadas}\label{monadas:usual}
Se define a continuación el concepto de mónada de la manera clásica dentro de la teoría de categorías.

\begin{definition}[Mónada]
Dada una categoría $\mathscr{C}$, una \textbf{mónada} sobre $\mathscr{C}$ es una tupla $(\mathit{T},\mu,\eta)$, donde:
\begin{itemize}[noitemsep,label=$\blacktriangleright$]
	\item $\mathit{T} : \mathscr{C} \rightarrow \mathscr{C}$ es un funtor,
	\item $\eta : \mathit{Id} \rightarrow \mathit{T}$ y $\mu : \mathit{T} \cdot \mathit{T} \rightarrow \mathit{T}$ son transformaciones naturales
	\item y se cumplen las siguientes identidades:
	\begin{equation*}
		\mu_X \circ \mathit{T}\mu_X = \mu_X \circ \mu_{\mathit{T}X} \text{, } \qquad \mu_X \circ \mathit{T}\eta_X = id_{\mathit{T}X} \text{, } \qquad 
		\mu_X \circ \eta_{\mathit{T}X} = id_{\mathit{T}X} 
	\end{equation*}
\end{itemize}
\end{definition}

A continuación se presentan algunos ejemplos de mónadas clásicas que son ampliamente utilizadas en computación.

\begin{ejemplo}[M\'onada \textit{Error}]
Sea $\mathit{T} : \text{\bf Set} \rightarrow \text{\bf Set}$ el funtor $\mathit{T} X = X + E$, donde $E$ es un conjunto de errores. Intuitivamente un elemento de $\mathit{T} X$ puede ser un elemento de $X$ (un valor) o un error pertenenciente a $E$. Luego se definen $\eta$ y $\mu$ como siguen:
\begin{itemize}[noitemsep, label=$\blacktriangleright$]
	\item Para cada conjunto $X$, se define $\eta_X : \mathit{Id} X \rightarrow \mathit{T} X$ como $\eta_X (x) = inl(x)$.
	\item Para cada conjunto $X$, se define $\mu_X : \mathit{T T} X \rightarrow \mathit{T} X$ como $\mu_X (inl(tx)) = tx$ si $tx \in X + E$ y $\mu_X (inr(e)) = inr(e)$ si $e \in E$. Es decir que si se tiene un error se propaga el error y si se tiene un elemento de $\mathit{T} X$ se devuelve dicho elemento. 
\end{itemize}
\end{ejemplo}

\begin{ejemplo}[M\'onada \textit{State}]
Sea $\mathit{T} : \text{\bf Set} \rightarrow \text{\bf Set}$ el funtor $\mathit{T} X = (X \times S)^S$, donde $S$ es un conjunto no vacío de estados. Intuitivamente, $\mathit{T} X$ es una computación que toma un estado y retorna el valor resultante junto con el estado modificado. Luego se definen $\eta$ y $\mu$ como sigue:
\begin{itemize}[noitemsep, label=$\blacktriangleright$]
	\item Para cada conjunto $X$, se define $\eta_X : \mathit{Id} X \rightarrow \mathit{T} X$ como $\eta_X (x) = (\lambda s : S . \langle x , s \rangle)$.
	\item Para cada conjunto $X$, se define $\mu_X : \mathit{T T} X \rightarrow \mathit{T} X$ como \\ \mbox{$\mu_X (f) = (\lambda s : S .$ let $\langle f' , s' \rangle = f(s)$ in $f'(s'))$}, es decir que $\mu_X (f)$ es la computación que, dado un estado $s$, primero computa el par computación-estado $f(s) = \langle f' , s' \rangle$ y luego retorna el par valor-estado $f'(s') = \langle x , s'' \rangle$.   
\end{itemize}
\end{ejemplo}

\subsection{Definición alternativa de Mónadas}\label{monadas:alt}

Se define ahora la noción de sistema de extensión, también llamado 3-tupla Kleisli. Esta definición también parte de la teoría de categorías pero no es la más utilizada en la literatura. Sin embargo, como se explica más adelante, es la que más se acerca a la forma de utilizar las mónadas en los lenguajes de programación funcional. 

\begin{definition}[Sistema de extensión]
Un \textbf{sistema de extensión} sobre una categoría $\mathscr{C}$ es una tupla $(\mathit{T},\eta,\_^*)$, donde 
\begin{itemize}[noitemsep,label=$\blacktriangleright$]
	\item $\mathit{T} : \mathbf{ob} \ \mathscr{C} \rightarrow \mathbf{ob} \ \mathscr{C}$,
	\item para cada $A \in \mathbf{ob} \ \mathscr{C}$, $\eta_A : A \rightarrow \mathit{T}A$,
	\item para cada $f : A \rightarrow \mathit{T}B$,  $f^* : \mathit{T}A \rightarrow \mathit{T}B$,
	\item y se cumplen las siguientes ecuaciones:
	\begin{itemize}[noitemsep,label=$\bullet$]
		\item $\eta^*_A = id_{\mathit{T}A}$
		\item $f^* \circ \eta_A = f$ para cada $f : A \rightarrow \mathit{T}B$
		\item $g^* \circ f^* = (g^* \circ f)^*$ para cada $f : A \rightarrow \mathit{T}B$ y $g : B \rightarrow \mathit{T}C$.
	\end{itemize}
\end{itemize}
\end{definition}

Intuitivamente $\eta_A$ es la inclusión de valores en computaciones (lo que en programación funcional usualmente se conoce como \textit{return}) y $f^*$ es la extensión de una función $f$ que va de valores a computaciones a una función que va de computaciones a computaciones, la cual primero evalúa una computación y luego aplica $f$ al valor resultante (lo que generalmente se conoce como \textit{bind} o $>$\hspace{-1mm}$>$\hspace{-1mm}$=$).

A continuación se muestra cómo quedan definidos los ejemplos vistos para la definición clásica como sistemas de extensión para que se comprenda mejor el paralelismo entre ambas definiciones. 

\begin{ejemplo}[M\'onada \textit{Error}] 
Tomando el funtor descripto en la versión clásica:
\begin{itemize}[noitemsep, label=$\blacktriangleright$]
	\item Para cada conjunto $X$, se define $\eta_X : \mathit{Id} X \rightarrow \mathit{T} X$ como $\eta_X (x) = inl(x)$ al igual que en la versión clásica.
	\item Para cada función $f : X \rightarrow \mathit{T}Y$, se define $f^* : \mathit{T} X \rightarrow \mathit{T} Y$ como \mbox{$f^*(inl(x)) = f(x)$} si $x \in X$ y $f^*(inr(e)) = inr(e)$ si $e \in E$.   
\end{itemize}
\end{ejemplo}

\begin{ejemplo}[M\'onada \textit{State}] 
Tomando el funtor descripto en la versión clásica:
\begin{itemize}[noitemsep, label=$\blacktriangleright$]
	\item Para cada conjunto $X$, se define $\eta_X : \mathit{Id} X \rightarrow \mathit{T} X$ como $\eta_X (x) = (\lambda s : S . \langle x , s \rangle)$ al igual que en la primera versión.
	\item Para cada función $f : X \rightarrow \mathit{T}Y$, se define $f^* : \mathit{T} X \rightarrow \mathit{T} Y$ como \\ \mbox{$f^*(g) = (\lambda s : S .$ let $\langle x , s' \rangle = g(s)$ in $f(x)(s'))$}.
\end{itemize}
\end{ejemplo}

\subsection{Funtores, mónadas y producto cartesiano}\label{monadas:cartesian}

La fortaleza de los funtores es una forma de compatibilidad entre funtores y productos. En adelante se trabajará con funtores y mónadas que son fuertes respecto del producto cartesiano. A continuación se definen las nociones de funtor fuerte y mónada fuerte.

\begin{definition}[Funtor fuerte]
Un \textbf{funtor} $\mathit{F} : \mathscr{C} \rightarrow \mathscr{C}$  es \textbf{fuerte} si viene equipado con una transformación natural $\sigma_{X,Y} : \mathit{F}X \times Y \rightarrow \mathit{F} (X \times Y)$, de manera que se cumplen las siguientes ecuaciones:
\begin{equation*}
	\pi_1 = \mathit{F}(\pi_1) \circ \sigma_{X,\mathbf{1}} \text{, \quad} \sigma \circ (\sigma \times id) \circ \alpha = \mathit{F}(\alpha) \circ \sigma 
\end{equation*} 
donde $\pi_1$ y $\pi_2$ son las proyecciones del producto cartesiano y $\alpha = \langle \langle \pi_1 , \pi_1 \circ \pi_2 \rangle , \pi_2 \circ \pi_2 \rangle$ representa su asociatividad.
\end{definition}

\begin{definition}[Mónada fuerte]
Una \textbf{mónada} $(\mathit{T},\mu,\eta)$ sobre $\mathscr{C}$ es \textbf{fuerte} si el funtor subyacente $\mathit{T}$ es fuerte y la fortaleza es compatible con $\mu$ y $\eta$: 
\begin{equation*}
\eta_{A \times B} = \sigma_{A,B} \circ (\eta_A \times \text{id}) \text{, \quad} \sigma_{A,B} \circ (\mu_A \times \text{id}) = \mu_{A \times B} \circ \mathit{T}\sigma_{A,B} \circ \sigma_{\mathit{T}A,B}
\end{equation*}
\end{definition}

Hay una definición similar de fortaleza $\bar{\sigma}_{X,Y} : X \times \mathit{F}Y \rightarrow \mathit{F} (X \times Y)$ que actúa sobre el lado derecho, pero como el producto cartesiano es simétrico, se puede obtener de la fortaleza izquierda como $\bar{\sigma} = \mathit{F} \gamma \circ \sigma \circ \gamma$, donde $\gamma = \langle \pi_2 , \pi_1 \rangle$ intercambia los elementos del producto cartesiano.

%\subsection{Estructuras monoidales}\label{monadas:monoidal}
Otra forma en la que un funtor puede ser compatible con el producto cartesiano es si es un funtor monoidal. A continuación se definen los conceptos de funtor monoidal y mónada monoidal.

\begin{definition}[Funtor monoidal]
Un \textbf{funtor monoidal} es un funtor $\mathit{F}: \mathscr{C} \rightarrow \mathscr{C}$ equipado con una estructura monoidal $(m,e)$, donde $m : \mathit{F} X \times \mathit{F} Y \rightarrow \mathit{F} (X \times Y)$ es una transformación natural y $e : \mathbf
{1} \rightarrow \mathit{F} \mathbf{1}$ es un morfismo tal que las siguientes ecuaciones se cumplen:
\begin{equation*}
\pi_1 = \mathit{F}(\pi_1) \circ m_{A,\mathbf{1}} \circ (id_{\mathit{F}A} \times e) \text{, \qquad} \pi_2 = \mathit{F}(\pi_2) \circ m_{\mathbf{1},A} \circ (e \times id_{\mathit{F}A}) \text{,}
\end{equation*}
\begin{equation*}
\mathit{F}(\alpha) \circ m_{X \times Y, Z} \circ (m_{X,Y} \times id_{\mathit{F}Z}) = m_{X, Y \times Z} \circ (id_{\mathit{F}X} \times m_{Y,Z}) \circ \alpha
\end{equation*}
Además, si la estructura monoidal es compatible con $\gamma$, entonces el funtor monoidal es simétrico.
\end{definition}

\begin{definition}[Mónada monoidal]
Una \textbf{mónada monoidal} es una mónada $(\textit{T},\mu,\eta)$ que tiene una estructura monoidal $(m,e)$ en su funtor subyacente $\mathit{T}$ tal que $e = \eta_{\mathbf{1}}$ y las estructuras monoidal y monádica son compatibles:
\begin{equation*}
\eta_{A \times B} = m_{A,B} \circ (\eta_A \times \eta_B) \text{, \quad} m_{A,B} \circ (\mu_A \times \mu_B) = \mu_{A \times B} \circ \mathit{T}m_{A,B} \circ m_{\mathit{T}A \times \mathit{T}B}.
\end{equation*}
\end{definition}

%\subsection{Mónadas conmutativas}\label{monads:commutative}
Dada una mónada fuerte $(\mathit{T},\mu,\eta)$, $\mathit{T}$ como funtor puede ser equipado con dos estructuras monoidales canóncas:
\begin{align*}
&\phi : \mathit{T}A \times \mathit{T}B \rightarrow \mathit{T} (A \times B) & &\psi : \mathit{T}A \times \mathit{T}B \rightarrow \mathit{T}(A \times B) \\
&\phi = \mu \circ \mathit{T}\bar{\sigma} \circ \sigma & &\psi = \mu \circ \mathit{T}\sigma \circ \bar{\sigma}
\end{align*}
y $e = \eta_1 : \mathbf{1} \rightarrow \mathit{T}\mathbf{1}$ en ambos casos. 

Se dice que una mónada es \textbf{conmutativa} cuando estas dos estructuras coinciden.

En esta tesina se utilizará la categoría \textbf{Set}, la cual es la categoría de conjuntos y funciones, y las mónadas que se presenten serán sobre esta categoría. El objeto terminal $\mathbf{1} = \{\star\}$ es un conjunto unitario. Una consecuencia particular de esto es que cualquier funtor $\mathit{F}$ y mónada $(\mathit{T},\mu,\eta)$ sobre esta categoría son fuertes, y cada uno admite una única fortaleza posible $\sigma$ ($\bar{\sigma}$). 

Por ejemplo, la mónada del conjunto partes $\mathcal{P}$ (y su variante finita $\mathcal{P}_f$) tiene fortaleza $\sigma(X,y) = X \times \{y\}$. En general, la fórmula de fortaleza de un funtor sobre \textbf{Set} puede ser expresada como $\sigma(v,y) = \mathit{F}(\lambda x : X . (x,y))(v)$. Cuando la mónada es conmutativa, hay sólo una estructura monoidal posible. En consecuencia, si una mónada es monoidal entonces es conmutativa \cite{kock:1970}. 

\section{Mónadas Concurrentes}\label{monconc:mc}

La teoría de concurrencia está compuesta por una amplia variedad de modelos basados en diferentes conceptos. Hoare et al. \cite{hoare:2011} se plantearon si es posible tener un tratamiento comprensible de la concurrencia en el cual la memoria compartida, el pasaje de mensajes y los modelos de intercalación e independencia de computaciones puedan ser vistos como parte de la misma teoría con el mismo núcleo de axiomas. Con esta motivación crearon un modelo simple de concurrencia basado en estructuras algebraicas, dos de las cuales resultan interesantes para este trabajo: bimonoides ordenados y monoides concurrentes. Más tarde, Rivas y Jaskelioff \cite{rivas:2019} extendieron este modelo al nivel de funtores y mónadas, dando lugar a las mónadas concurrentes. En las siguientes secciones se detallarán las características principales de cada uno de estos modelos.

\subsection{Ley de intercambio}\label{mc:interchange}

Ya estaba establecido que la composición secuencial y concurrente son estructuras monoidales, donde la concurrencia es además conmutativa. La pregunta que surge luego es cómo estas operaciones se relacionan entre sí. Se podría pensar en un principio que la ley de intercambio $(p * r) ; (q * s) = (p ; q) * (r ; s)$ de 2-categorías o bi-categorías debería cumplirse.  Sin embargo, la presencia de esta ley implicaría que ambas estructuras monoidales coinciden, derivando en que las operaciones de secuenciación y concurrencia son la misma. Esto se puede ver aplicando el argumento Eckmann-Hilton. 

\begin{thm}[argumento Eckmann-Hilton] Sea $X$ un conjunto con dos operaciones binarias $;$ y $*$ tal que $e_;$ es el elemento neutro de $;$, $e_*$ es el elemento neutro de $*$ y la ley de intercambio $(a * b) ; (c * d) = (a ; c) * (b ; d)$ se cumple. Entonces, ambas operaciones $;$ y $*$ coinciden, y ambas son conmutativas y asociativas. 
\end{thm}
\begin{proof}
Primero se muestra que ambos elementos neutros coinciden:
\begin{equation*}
e_; = e_; ; e_; = (e_* * e_;);(e_; * e_*) = (e_* ; e_;) * (e_; ; e_*) = e_* * e_* = e_*
\end{equation*}
Como los neutros coinciden, se lo puede llamar simplemente $e$. Se muestra ahora que ambas operaciones coinciden:
\begin{equation*}
a ; b = (e * a) ; (b * e) = (e ; b) * (a ; e) = b * a = (b ; e) * (e ; a) = (b * e) ; (e * a) = b ; a
\end{equation*}
Usando el mismo argumento se puede ver también que la operación es conmutativa. La prueba de asociatividad es análoga. 
\end{proof}

Como solución a esto, surge la idea de considerar un orden en los procesos, de manera que pueda debilitarse la ley de intercambio. En \cite{hoare:2009} se introduce una generalización del álgebra de Kleene para programas secuenciales \cite{kozen:1994}, llamada Álgebra Concurrente de Kleene. Esta es un álgebra que mezcla primitivas de composición concurrente ($*$) y secuencial ($;$), cuya característica principal es la presencia de una versión ordenada de la ley de intercambio de 2-categorías o bi-categorías. 
\begin{equation*}
(p * r) ; (q * s) \sqsubseteq (p ; q) * (r ; s)
\end{equation*}

Esta ley intuitivamente tiene sentido, por ejemplo, en un modelo de concurrencia de intercalación. Si se tiene una traza $t = t_1 ; t_2$ donde $t_1$ es una intercalación de dos trazas $t_p$ y $t_r$, y $t_2$ de $t_q$ y $t_s$, entonces $t$ es también una intercalación de $t_p ; t_q$ y $t_r ; t_s$. 

\subsection{Dos modelos}\label{mc:models}
Se introducirán a continuación dos modelos utilizados por Hoare et al. \cite{hoare:2011} para desarrollar su teoría, los cuales servirán de ejemplo en las secciones que siguen.

\paragraph{Modelo de Trazas} Sea $A$ un conjunto. Luego $Trazas_A$ es el conjunto de secuencias finitas de elementos de $A$. El conjunto partes $\mathcal{P}(Trazas_A)$ será el conjunto soporte del modelo. Se tienen las siguientes operaciones binarias sobre $\mathcal{P}(Trazas_A)$:
\begin{enumerate}
	\item $T_1 * T_2$ es el conjunto de las intercalaciones de las trazas de $T_1$ y $T_2$.
	\item $T_1 ; T_2$ es el conjunto de las concatenaciones entre las trazas de $T_1$ y $T_2$.
\end{enumerate}
El conjunto $\{\epsilon\}$ funciona como elemento neutro para ambas operaciones $;$ y $*$, donde $\epsilon$ es la secuencia vacía. El orden está dado por la inclusión de conjuntos. 

\paragraph{Modelo de Recursos} Sea $(\Sigma,\bullet,u)$ un monoide parcial conmutativo, dado por una operación parcial binaria $\bullet$ y elemento neutro $u$. La igualdad significa que ambos lados están definidos y son iguales, o que ninguno está definido. El conjunto partes $\mathcal{P}(\Sigma)$ tiene una estructura de monoide ordenado conmutativo $(*,\text{emp})$ definido por:
\begin{align*}
p * q &= \{\sigma_0 \bullet \sigma_1 | (\sigma_0, \sigma_1) \in dom(\bullet) \land \sigma_0 \in p \land \sigma_1 \in q\} \\
emp &= \{u\}
\end{align*}
El conjunto de funciones monótonas $\mathcal{P}(\Sigma) \rightarrow \mathcal{P}(\Sigma)$ es el conjunto soporte del modelo. Estas funciones representan transformadores de predicados. Las operaciones se definen mediante las siguientes fórmulas, donde $F_i$ itera sobre los transformadores de predicados e $Y_i$ se utiliza para iterar sobre los subconjuntos de $\Sigma$:
\begin{align*}
(F * G) Y &= \bigcup \{FY_1 * GY_2 | Y_1 * Y_2 \subseteq Y\} \\
\mathtt{nothing} \ Y &= Y \cap \text{emp}  \\
(F ; G) Y &= F(G(Y))  \\
\mathtt{skip} \ Y &= Y \\
\end{align*}
La idea es que se comienza con una postcondición $Y$, luego se la separa en dos afirmaciones separadas $Y_1$ e $Y_2$ y se aplica la regla de concurrencia hacia atrás para obtener una precondición $FY_1 * GY_2$ para la composición paralela de $F$ y $G$. Se realilza la unión de todas estas descomposiciones de manera de obtener la precondición más débil posible. 

El orden del modelo está dado por el orden reverso punto a punto, es decir $F \sqsubseteq G$ significa que $\forall X \subseteq \mathcal{P}(\Sigma), FX \supseteq GX$. Según esta definición, el elemento más pequeño es la función $\lambda X.\Sigma$, la cual se corresponde con el transformador de precondición más débil para la divergencia. 

\subsection{Monoides concurrentes}\label{mc:monoid}

Como se va a utilizar un orden combinado con estructuras alegbraicas, se necesita una noción de compatibilidad de las operaciones con el orden.
Sea $(A, \sqsubseteq)$ un orden parcial, entonces una operación $\oplus : A \times A \rightarrow A$ es \textit{compatible} con el orden si $a \sqsubseteq b$ y $c \sqsubseteq d$ implica que $a \oplus c \sqsubseteq b \oplus d$. 
Se define primero una aproximación a la noción de monoide concurrente, el cual tiene dos estructuras monoidales y un orden compatible con ellas, pero no incluye ninguna relación especial entre ellas.  

\begin{definition}[Bimonoide ordenado]
Un \textbf{bimonoide ordenado} es un conjunto parcialmente ordenado $(A,\sqsubseteq)$ junto con dos estructuras monoidales $(A,*,\mathtt{nothing})$ y $(A,;,\mathtt{skip})$ tal que $*$ y $;$ son compatibles con $\sqsubseteq$ y $*$ es conmutativa.
\end{definition}

Podría ser tentador requerir que ambos elementos neutros de un bimonoide ordenado sean iguales, pero, por ejemplo, en el Modelo de Recursos no lo son. El Modelo de Recursos es un ejemplo de un bimonoide ordenado que no es un monoide concurrente. A continuación se define la noción de monoide concurrente.

\begin{definition}[Monoide concurrente]
Un \textbf{monoide concurrente} es un bimonoide ordenado tal que los neutros coinciden, es decir $\mathtt{nothing} = \mathtt{skip}$, y la siguiente ley de intercambio se cumple:
\begin{equation*}
(a * b) ; (c * d) \sqsubseteq (a ; c) * (b ; d)
\end{equation*}
\end{definition}

En esta estructura no hay reducción de la operación $*$ a una intercalación del operador $;$. El orden une a ambas estructuras sin reducir una a la otra. En el Modelo de Trazas ambos elementos neutros coinciden y la ley de intercambio se cumple, por lo que, además de bimonoide ordenado, es un monoide concurrente. 


\subsection{Generalización a nivel de funtores y mónadas}\label{mc:lifting}

Para elevar la teoría descripta en la sección anterior al nivel de funtores, se modela el operador de secuenciación $;$ como la multiplicación monádica y su elemento neutro $\mathtt{skip}$ como la unidad monádica. Sólo faltan tres elementos: la operación de mezcla $*$, su elemento neutro $\mathtt{nothing}$ y el orden $\sqsubseteq$. La operación de mezcla se modela como una familia de transformaciones naturales $m_{A,B} : \mathit{T} A \times \mathit{T} B \rightarrow \mathit{T} (A \times B)$. Como se necesita que $*$ sea un monoide conmutativo, se requiere que $\mathit{T}$ tenga una estructura de funtor monoidal simétrica, incluyendo un morfismo unidad $e : \mathbf{1} \rightarrow \mathit{T} \mathbf{1}$ que representa $\mathtt{skip}$. 

En cuanto al orden, lo que se necesita es un orden sobre computaciones que mida el grado de secuencialidad en ellas. Esto es una estructura de orden $\sqsubseteq_I$ sobre $\mathit{T} I$ para cada conjunto $I$, sujeto a algún tipo de compatibilidad con el resto de la estructura que modela computaciones. Se define a continuación la noción de funtor ordenado, la cual establece una condición de compatibilidad de un funtor con un orden parcial, siguiendo la idea de Hughes y Jacobs \cite{hughes:2004}:

\begin{definition}[Funtor ordenado]
Un \textbf{orden} para un \textbf{funtor} $\mathit{F}$ es una asignación de un orden parcial $\sqsubseteq_I$ en $\mathit{F} I$ para cada conjunto $I$, tal que para cada morfismo $f : I \rightarrow J$, el morfismo $\mathit{F} f : \mathit{F} I \rightarrow \mathit{F} J$ es una función monótona respecto de $\sqsubseteq_I$ y $\sqsubseteq_J$.
\end{definition}

A continuación se define la compatibilidad para la estructura de mónada. Para esto se utiliza una familia de estructuras ordenadas, lo cual es una instancia de lo que definieron Katsumata y Sato como una mónada ordenada \cite{katsumata:2013}.

\begin{definition}[Mónada ordenada]
Un \textbf{orden} para una \textbf{mónada} $(\mathit{T},\mu,\eta)$ es una asignación de un orden $\sqsubseteq_I$ sobre $\mathit{T} I$ para cada conjunto $I$, tal que la categoría Kleisli de $T$ se enriquece en la categoría de órdenes con el orden correspondiente a $\mathcal{K}\ell(\mathit{T})(A,B)$ definido por $f \sqsubseteq_{A,B} g$ si y sólo si $\forall a : \mathbf{1} \rightarrow A, f \circ a \sqsubseteq_B g \circ a$.
\end{definition}

Esta noción de orden sólo se relaciona con la estructura monádica. En comparación a los bimonoides ordenados, se corresponde con la condición de que $;$ sea compatible con $\sqsubseteq$. La compatibilidad entre $*$ y $\sqsubseteq$ se postula como sigue:

\begin{definition}[Funtor monoidal ordenado]
Sea $\mathit{T}$ un funtor con una estructura monoidal $(m,e)$ y una asignación de un orden $\sqsubseteq_I$ sobre $\mathit{T} I$ para cada conjunto $I$. Se dice que el orden es compatible con $(m,e)$ si $v \sqsubseteq_A v'$ y $w \sqsubseteq_B w'$ implican que $m \circ \langle v,w \rangle \sqsubseteq_{A \times B} m \circ \langle v',w' \rangle$ para cada $v,v' : \mathbf{1} \rightarrow \mathit{T} A$ y $w,w' : \mathbf{1} \rightarrow \mathit{T} B$. 
\end{definition}

A partir de todas estas definiciones previas se definen las variantes monádicas de los bimonoides ordenados y monoides concurrentes como sigue:

\begin{definition}[Mónada monoidal ordenada]
Una mónada $(\mathit{T},\eta,\_^*)$ es una mónada monoidal ordenada si está dotada de una estructura de funtor monoidal simétrico 
\begin{equation*}
m_{X,Y} : \mathit{T} X \times \mathit{T} Y \rightarrow \mathit{T} (X \times Y) \text{, \qquad} e : \mathbf{1} \rightarrow \mathit{T} \mathbf{1} 
\end{equation*}
y una relación de orden $\sqsubseteq$ sobre $\mathit{T}$ compatible con $(m,e)$.
\end{definition}

\textit{Aclaración: se escribe $f \star g$ para representar $m \circ (f \times g)$.}

\begin{definition}[Mónada concurrente]
Una mónada monoidal ordenada $\mathit{T}$ es una mónada concurrente si $e = \eta_{\mathbf{1}} : \mathbf{1} \rightarrow \mathit{T} \mathbf{1}$ y se cumple la siguiente ley de intercambio:
\begin{equation*}
(h \star i) \bullet (f \star g) \sqsubseteq (h \bullet f) \star (i \bullet g)
\end{equation*}
\end{definition}

Esta definición puede probarse mostrando que las mónadas conmutativas (como los monoides conmutativos) son mónadas concurrentes (como los monoides concurrentes). 

\begin{ejemplo}
Sea $(\mathit{T},\mu,\eta)$ una mónada conmutativa. Entonces $\mathit{T}$ tiene una única estructura de mónada monoidal $(m,\eta_{\mathbf{1}})$, y se puede definir la estructura de orden por el orden diagonal. La ley de intercambio se reduce a las condiciones de mónada monoidal. 
\end{ejemplo}

Rivas y Jaskelioff \cite{rivas:2019} también muestran que estas estructuras al nivel de mónada efectivamente generalizan aquellas al nivel de los monoides probando el siguiente lema.

\begin{lema}
Sea $(\mathit{T},\mu,\eta)$ una mónada monoidal ordenada (mónada concurrente). Entonces $\mathit{T} \mathbf{1}$ es un bimonoide ordenado (monoide concurrente). 
\end{lema}

Como en el caso de los monoides, hay dos estructuras de bimonoide ordenado sobre $\mathit{T} \mathbf{1}$, que resultan de la simetría de los axiomas de los bimonoides ordenados.  En esta estructura, como en los monoides, se puede también ir en la dirección contraria: dado un bimonoide ordenado, este puede ser elevado a una mónada monoidal ordenada.

\begin{lema}
Sea $(A,\sqsubseteq,*,\mathtt{nothing},;,\mathtt{skip})$ un bimonoide ordenado. Este puede convertirse en una mónada monoidal ordenada con funtor soporte $\mathit{T}_A X = A \times X$, operaciones y orden. Más aún, si $\mathtt{nothing} = \mathtt{skip}$ (es decir que es un monoide concurrente), entonces $e = \eta_{\mathbf{1}}$ (es decir que es una mónada concurrente).
\end{lema}

Este resultado será utilizado más adelante en este trabajo para dar una representación alternativa de la mónada delay. 

El ejemplo del Modelo de Trazas puede ser generalizado a una mónada concurrente parametrizando el conjunto sobre el cual se toman las trazas. 

\begin{ejemplo}
La mónada $Tr_L(X) = \mathcal{P}_f(Trazas_{L \times X})$ es concurrente. La estructura de orden se define como la inclusión de conjuntos al igual que antes.
\end{ejemplo}

\chapter{La Mónada \textit{Delay}}\label{chapter:delay}

Como se discutió previamente, la teoría de tipos de Martin-Löf es un lenguaje de programación funcional rico con tipos dependientes y, a su vez, un sistema de lógica constructiva. Sin embargo, esto trae una limitación respecto de los lenguajes de programación funcional estándar, ya que obliga a que todas las computaciones deban terminar. Esta restricción tiene dos razones principales: hacer que el chequeo de tipos de los tipos dependientes sea decidible y representar pruebas como programas (una prueba que no termina es inconsistente). 

El tipo de dato \textit{delay} fue introducido por Capretta \cite{capretta:2005} con el objetivo de facilitar la representación de la no-terminación de funciones en la teoría de tipos de Martin-Löf. Lo que busca es explotar los tipos coinductivos para modelar computaciones infinitas. Los habitantes del tipo \textit{delay} son valores ``demorados'', los cuales pueden no terminar y, por lo tanto, no retornar un valor nunca. El tipo de dato \textit{delay} es una mónada y constituye una alternativa constructiva de la mónada \textit{maybe}. 

Se introducirá primero la noción de coinducción, junto con los soportes para coinducción que Agda proporciona. Luego se presentará la definición de la mónada \textit{delay} y sus principales características.

\section{Introducción a la Coinducción}\label{delay:coind}

El principio de inducción está bien establecido en el área de las matemáticas y las ciencias de la computación. En esta última, se utiliza principalmente para razonar sobre tipos de datos definidos inductivamente, tales como listas finitas, árboles finitos y números naturales. La coinducción es el pincipio dual de la inducción y puede ser utilizado para razonar sobre tipos de datos definidos coinductivamente, tales como flujos de datos, trazas infinitas o árboles infinitos, pero no está tan difundido ni se comprende tan bien en general. 

Para ilustrar mejor el concepto de coinducción, se utilizarán algunos ejemplos presentados por Kozen y Silva \cite{kozen:2017} con el objetivo de promover este principio como una herramienta útil y hacerlo tan familiar e intuitivo como la inducción. 

A continuación se considera el ejemplo del tipo \texttt{Lista de $A$} de listas finitas sobre un alfabeto $A$, definido inductivamente por:
\begin{itemize}[label=$\blacktriangleright$]
	\item \texttt{nil} $\in$ \texttt{Lista de $A$}
	\item si $a \in A$ y $\ell \in$ \texttt{Lista de $A$}, entonces $a$ \texttt{::} $\ell \in$ \texttt{Lista de $A$}. 
\end{itemize}

El tipo de dato definido es la solución  mínima a la ecuación:
\begin{equation}\label{list}
\text{\tt Lista de $A$} = \text{\tt nil} + A \times \text{\tt Lista de $A$}
\end{equation}
Es decir que es el mínimo conjunto tal que se cumplen las condiciones listadas más arriba. Esto significa que uno puede definir funciones con dominio \texttt{Lista de $A$} de manera única por inducción estructural. El tipo de las listas finitas e infinitas sobre $A$ se define coinductivamente como la solución máxima de la ecuación \ref{list}. Esto significa que es el máximo conjunto tal que se cumplen ambas condiciones. 

Formalmente, el tipo de las listas finitas sobre $A$ es un álgebra inicial para una signatura que consiste en una constante (\texttt{nil}) y un constructor binario (\texttt{::}). El tipo de las listas finitas e infinitas sobre $A$ forman la coálgebra final de la signatura (\texttt{nil}, \texttt{::}). Se definen a continuación los conceptos de álgebra, coálgebra, álgebra inicial y coálgebra final:

\begin{definition}[Álgebra de un funtor]
Dado un endofuntor $\mathit{F}$ sobre una categoría $\mathscr{C}$, un \textbf{álgebra} de $\mathit{F}$ es un objeto $X$ de $\mathscr{C}$ junto con un morfismo $\alpha : \mathit{F}X \rightarrow X$. Dadas dos álgebras $(X, \alpha : \mathit{F}X \rightarrow X)$, $(Y, \beta : \mathit{F}Y \rightarrow Y)$ de $F$, $m : X \rightarrow Y$ es un morfismo de álgebras si se cumple la siguiente ecuación:
\begin{equation*}
m \circ \alpha = \beta \circ \mathit{F}(m)
\end{equation*}
Las álgebras de $\mathit{F}$ junto con sus morfismos forman una categoría llamada $\mathit{F}$-álgebras. 
\end{definition}

\begin{definition}[Coálgebra]
Una \textbf{coálgebra} para un endofuntor $\mathit{F}$ sobre una categoría $\mathscr{C}$ es un objeto $A$ junto con un morfismo $u : A \rightarrow \mathit{F} A$. Dadas dos coálgebras $(A, \eta : A \rightarrow \mathit{F}A), \quad (B, \theta : B \rightarrow \mathit{F}B)$, $f : A \rightarrow B$ es un morfismo de coálgebras si respeta la estructura coalgebraica: 
\begin{equation*}
\theta \circ f = \mathit{F}(f) \circ \eta
\end{equation*} 
Las coálgebras de $\mathit{F}$ junto con sus morfismos generan una categoría llamada $\mathit{F}$-coálgebras.
\end{definition}

\begin{definition}[Álgebra inicial]
Un \textbf{álgebra inicial} para un endofuntor $\mathit{F}$ sobre una categoría $\mathscr{C}$ es un objeto inicial en la categoría de las $\mathit{F}$-álgebras.
\end{definition}

\begin{definition}[Coálgebra final]
Una \textbf{coálgebra final} para un endofuntor $\mathit{F}$ sobre una categoría $\mathscr{C}$ es un objeto terminal en la categoría de las $\mathit{F}$-coálgebras. 
\end{definition}

Formalmente, los tipos coinductivos se definen como elementos de una coálgebra final para un endofuntor dado en la categoría \textbf{Set}. 

\begin{ejemplo}[Flujo de datos infinitos]
El conjunto $A^{\omega}$ de flujos de datos (o \textit{streams} en inglés) infinitos sobre un alfabeto $A$ es (el conjunto soporte de) la coálgebra final del funtor $\mathit{F}X = A \times X$.
\end{ejemplo}

\begin{ejemplo}[Cadenas infinitas]
El conjunto $A^{\infty}$ de las cadenas finitas e infinitas sobre un alfabeto $A$ es (el conjunto soporte de) la coálgebra final del funtor $\mathit{F}X = \mathds{1} + A \times X$.
\end{ejemplo}

Mientras que los tipos inductivos se definen mediante sus constructores, los tipos coinductivos usualmente se presentan junto con sus destructores. Por ejemplo, los flujos de datos o \textit{streams} admiten dos operaciones $hd: A^{\omega} \rightarrow A$ y $tl : A^{\omega} \rightarrow A^{\omega}$, los cuales representan la función $head$ que devuelve el primer elemento del \textit{stream} y la función $tail$ que devuelve la cola del \textit{stream}. La existencia de los destructores es una consecuencia del hecho de que $A^{\omega}$ es una coálgebra para el funtor $\mathit{F}X = A \times X$. Todas estas coálgebras vienen equipadas con una función estructural $\langle obs, cont \rangle  : X \rightarrow A \times X$; para $A^{\omega}$ se tiene que $obs = hd$ y $cont = tl$.

Las pruebas por coinducción tienen un paso coinductivo (análogo al paso inductivo) pero no caso base. Aunque esto parezca incorrecto o genere cierta desconfianza en dichas pruebas, cualquier dificultad que haga que la propiedad a demostrar no se cumpla se manifiesta en el intento de probar el paso coinductivo.

\section{Coinducción en Agda}\label{coind:agda}

Se describirán a continuación los dos soportes de coinducción en Agda que se utilizaron en esta Tesina. El primero se basa en una notación particular, la notación musical, la cual permite manejar términos potencialmente infinitos. A pesar de ser una notación práctica e intuitiva, este soporte tiene algunos problemas con el chequeo de terminación de Agda, lo que limita bastante las propiedades que pueden demostrarse usándolo. Es por eso que se utilizó luego otro enfoque, basado en tipos de tamaño limitado (\textit{sized types} en inglés), el cual ayuda al chequeo de terminación de Agda haciendo un seguimiento de la profundidad de las estructuras de datos mediante la definición de límites en la profundidad. 

\subsection{Notación Musical}\label{coind:agda:musical}

Para mostrar la notación musical se utilizará como ejemplo el conjunto de los números \textit{conaturales}. Así como las coálgebras son el dual de las álgebras, los números conaturales son el dual de los números naturales y se definen en Agda como sigue:

\ExecuteMetaData[latex/Coind.tex]{musconat}

El operador \textit{delay} (\AgdaDatatype{$\infty$}) se utiliza para etiquetar ocurrencias coinductivas. El tipo \AgdaDatatype{$\infty$ A} se interpretea como una computación suspendida o demorada de tipo \AgdaDatatype{A}. Este operador viene equipado con funciones \textit{delay} y \textit{force}:

\ExecuteMetaData[latex/Coind.tex]{delayforce}

La función \textit{delay} (\AgdaFunction{$\sharp\_$}) toma un valor de tipo \AgdaDatatype{A} y lo devuelve suspendido dentro de un valor de tipo \AgdaDatatype{$\infty$ A}. Por el contrario, la función \textit{force} (\AgdaFunction{$\flat$}), toma un valor de tipo \AgdaDatatype{$\infty$ A} y lo desencapsula devolviendo un valor de tipo \AgdaDatatype{A}.

Los valores de tipos coinductivos pueden ser construidos usando corecursión, la cual no debe necesariamente terminar, pero sí ser productiva. Por ejemplo, el infinito puede ser difinido como se muestra a continuación.

\ExecuteMetaData[latex/Coind.tex]{inf}

Como aproximación a la productividad, en el chequeo de terminación se pide que en la definición de funciones corecursivas las llamadas recursivas aparezcan bajo la aplicación directa de un construcor coinductivo. Esta restricción en general hace que programar con tipos coinductivos sea incómodo, y es por eso que se buscan técnicas alternativas para asegurar que las definiciones corecursivas estén bien definidas. 

\subsection{\textit{Sized Types}}

Agda tiene un soporte nativo para \textit{sized types}. Estos son tipos que cuentan con un índice que indica el número de desencapsulamientos que pueden realizarse sobre los habitantes de este tipo. Estos índices, llamados tamaños o \textit{sizes}, asisten al chequeo de terminación evaluando que las definiciones corecursivas estén bien definidas. 

En Agda existe un tipo \AgdaPrimitiveType{Size} de tamaños y un tipo \AgdaPrimitiveType{Size<} \AgdaArgument{i} cuyos habitantes son los tamaños estrictamente menores a \AgdaArgument{i}. Si se tiene un tamaño \AgdaArgument{j} \AgdaSymbol{:} \AgdaPrimitiveType{Size<} \AgdaArgument{i}, este es forzado a ser \AgdaArgument{j} \AgdaSymbol{:} \AgdaPrimitiveType{Size}. La relación de orden de los tamaños es transitiva, lo que implica que si se tiene que \AgdaArgument{j} \AgdaSymbol{:} \AgdaPrimitiveType{Size<} \AgdaArgument{i} y \AgdaArgument{k} \AgdaSymbol{:} \AgdaPrimitiveType{Size<} \AgdaArgument{j}, entonces \AgdaArgument{k} \AgdaSymbol{:} \AgdaPrimitiveType{Size<} \AgdaArgument{i}. La relación de orden es, además, bien fundada, lo cual se usa para definir funciones corecursivas productivas. Existe también una operación sucesor de tamaños \AgdaFunction{$\uparrow$} y un tamaño ``infinito'' \AgdaArgument{$\infty$} tal que para cada tamaño \AgdaArgument{i}, \AgdaArgument{i} \AgdaSymbol{:} \AgdaPrimitiveType{Size<} \AgdaArgument{$\infty$}. Finalmente, un \textit{sized type} es un tipo indexado por \AgdaPrimitiveType{Size}.

Para ejemplificar el uso de los \textit{sized types}, se definen a continuación los conaturales utilizando esta técnica.

\ExecuteMetaData[latex/Coind.tex]{sizedconat}

Ambos tipos, \AgdaDatatype{Conat} y \AgdaRecord{Conat$'$} están indexados por un tamaño \AgdaArgument{i}. Este índice debe entenderse como una cota superior del número de veces que puede aplicarse \AgdaField{force}. Más precisamente, cuando se aplica \AgdaField{force} a un \AgdaArgument{n$'$} \AgdaSymbol{:} \AgdaRecord{Conat$'$} \AgdaArgument{i} el valor resultante es es un \AgdaArgument{n} \AgdaSymbol{:} \AgdaDatatype{Conat} \AgdaArgument{j} de una profundidad estrictamente menor \AgdaArgument{j} \AgdaSymbol{<} \AgdaArgument{i}. Un caso especial es el valor \AgdaArgument{$\infty$n$'$} \AgdaSymbol{:} \AgdaRecord{Conat$'$} \AgdaArgument{$\infty$} de índice infinito, cuyo resultado de aplicar \AgdaField{force} es \AgdaArgument{$\infty$n} \AgdaSymbol{:} \AgdaDatatype{Conat} \AgdaArgument{$\infty$}, el cual también tiene índice infinito. De esta manera los tamaños establecen productividad en las definiciones recursivas. Al final, sólo interesan los valores \AgdaArgument{n} \AgdaSymbol{:} \AgdaDatatype{Conat} \AgdaArgument{$\infty$} de índice infinito.

Si una función corecursiva en \AgdaDatatype{Conat} \AgdaArgument{i} sólo se llama a sí misma con índices menores \AgdaArgument{j} \AgdaSymbol{<} \AgdaArgument{i}, se garantiza la productividad y, por lo tanto, está bien definida. En la siguiente definición del valor \AgdaFunction{infty} se muestran los argumentos implícitos de tamaño explícitamente de manera que se evidencie cómo aseguran la productividad:

\ExecuteMetaData[latex/Coind.tex]{sizedinf}


\section{Mónada \textit{Delay}}\label{delay:delay}

El tipo de dato \textit{delay} consituye una mónada fuerte, lo cual hace posible manejar computaciones que posiblemente no terminen como si fuera cualquier otra computación con efectos laterales. A continuación se presenta su definición formal siguiendo el estilo utilizado en \cite{chapman:2019}. 

\begin{definition}[\textit{Delay}]
Sea $X$ un tipo. Cada habitante de $\mathbf{D} X$ es una computación posiblemente infinita que, si termina, retorna un valor de tipo $X$. Se define $\mathbf{D} X$ como un tipo coinductivo mediante las siguientes reglas:
\begin{equation*}
\dfrac{}{\mathtt{now} \ x : \mathbf{D} X} 	\qquad  	\dfrac{c : \mathbf{D} X}{\mathtt{later} \ c : \mathbf{D} X}
\end{equation*}
\end{definition}

Sea $R$ una relación de equivalencia sobre $X$. La relación se eleva a una relación de equivalencia $\sim_R$ sobre $\mathbf{D}X$ que se denomina $R$-bisemejanza fuerte (\textit{$R$-strong bisimilarity} en inglés). 

\begin{definition}[$R$-bisemejanza fuerte]
Dada una relación de equivalencia $R$ sobre $X$, se define la relación $\sim_R$ sobre $\mathbf{D}X$ coinductivamente mediante las siguientes reglas:
\begin{equation*}
\dfrac{p : x_1 \ R \ x_2}{\mathtt{now}_{\sim} \ p : \mathtt{now} \ x_1 \sim_R \mathtt{now} \ x_2}  	\qquad  	\dfrac{p : c_1 \sim_R c_2}{\mathtt{later}_{\sim} \ p : \mathtt{later} \ c_1 \sim_R \mathtt{later} \ c_2}
\end{equation*}
\end{definition}

La $\equiv$-bisemejanza fuerte se denomina simplemente bisemejanza fuerte y se denota $\sim$.

En algunos casos, uno está interesado en la terminación de las computaciones y no exactamente en el tiempo exacto en el cual terminan. Es deseable entonces tener una relación que considere iguales dos computaciones si terminan con valores iguales, aunque una tarde más en terminar que la otra. Es decir, que identifique computaciones que sólo difieren en una cantidad finita de aplicaciones del constructor \texttt{later}. Esta relación se llama $R$-bisemejanza débil (\textit{$R$-weak bisimilarity} en inglés) y se define en términos de \textit{convergencia}. Esta última es una relación binaria entre $\mathbf{D}X$ y $X$ que relaciona computaciones que terminan con sus valores de terminación. 

\begin{definition}[Convergencia]
La relación de \textbf{convergencia} denotada con $\downarrow$ entre $\mathbf{D}X$ y $X$ se define inductivamente mediante las siguientes reglas:
\begin{equation*}
\dfrac{p : x_1 \equiv x_2}{\mathtt{now}_{\downarrow}  \ p : \mathtt{now} \ x_1 \downarrow x_2}  	\qquad  	\dfrac{p : c \downarrow x}{\mathtt{later}_{\downarrow} \ p : \mathtt{later} \ c \downarrow x}
\end{equation*}
\end{definition}

\begin{definition}[$R$-bisemejanza débil]
Dada una relación de equivalencia $R$ sobre $X$, se define la relación $\approx_R$ sobre $\mathbf{D}X$ coinductivamente mediante las siguientes reglas:
\begin{equation*}
\dfrac{p_1 : c_1 \downarrow x_1 \quad p_2 : x_1 \ R \ x_2 \quad p_3 : c_2 \downarrow x_2}{\downarrow_{\approx} \ p_1 \ p_2 \ p_3 : c_1 \approx_R c_2}  	\qquad  	\dfrac{p : c_1 \approx_R c_2}{\mathtt{later}_{\approx} \ p : \mathtt{later} \ c_1 \approx_R \mathtt{later} \ c_2}
\end{equation*}
\end{definition}

La $\equiv$-bisemejanza débil se denomina simplemente bisemejanza débil y se denota $\approx$. En este caso, se modifica el primer constructor por simplicidad:

\begin{equation*}
\dfrac{p_1 : c_1 \downarrow x \quad p_2 : c_2 \downarrow x}{\downarrow_{\approx} \ p_1 \ p_2 : c_1 \approx c_2}
\end{equation*}

El tipo \textit{delay} $\mathbf{D}$ es una mónada fuerte. La unidad $\eta$ es el constructor \texttt{now}, mientras que la multiplicación $\mu$ es la ``concatenación'' de constructores \texttt{later}:
\begin{align*}
& \mu : \mathbf{D} (\mathbf{D} X) \rightarrow \mathbf{D} X  \\
& \mu \ (\mathtt{now} \ c) = c \\
& \mu \ (\mathtt{later} \ c) = \mathtt{later} \ (\mu \ c)
\end{align*}

\part{Formalización de Mónadas Concurrentes}\label{part:mc}
 

\bookmarksetup{startatroot}

\chapter{Conclusiones y trabajo futuro}

Para realizar esta tesina se estudió en profundidad el concepto de monoides concurrentes y mónadas concurrentes, así como también los tipos coinductivos el uso de la coinducción en general para luego comprender la definición del tipo \textit{delay} y cómo este se utiliza para representar formalmente la no terminación de programas. 

En este trabajo se utilizó el lenguaje y asistente de pruebas Agda para realizar formalizaciones de diversas estructuras algebraicas que ayudaron a construir la formalización de las mónadas concurrentes. Entre ellas se encuentran los monoides, las mónadas, los funtores monoidales y los monoides concurrentes. Posteriormente se analizó el caso de la mónada \textit{delay} con el objetivo de demostrar que esta es una mónada concurrente. La demostración de la ley de intercambio mostró diversas complicaciones que llevaron a tomar la decisión de simplificar el problema a los números conaturales. Al intentar demostrar la misma ley para la versión reducida del problema, se presentaron dificultades en torno al soporte para coinducción elegido. Finalmente, se decidió cambiar este soporte por otro, logrando demostrar que los conaturales forman un monoide concurrente y definiendo una mónada concurrente que tiene ciertas similitudes con la mónada \textit{delay}. 

Las principales conclusiones de este trabajo son: 

\begin{enumerate}
\item La formalización de estructuras algebraicas puede realizarse en Agda mediante la utilización de tipos \AgdaKeyword{record}. Estos tipos admiten la definición de campos, los cuales se utilizaron para definir tanto los elementos que conforman la estructura como las propiedades que se deben cumplir para que dichos elementos efectivamente constituyan la estructura deseada. Al realizar las formalizaciones de esta manera, una instancia de uno de estos tipos no sólo define un ejemplo de la estructura formalizada, sino que también demuestra que el ejemplo definido cumple con las características necesarias para serlo. 

\item Agda es un lenguaje y asistente de pruebas muy potente pero puede llegar a traer muchas complicaciones a la hora de representar la coinducción. En general, la coinducción tiene conflictos con todos los lenguajes que no permitan la no terminación de programas puesto que, como se mencionó anteriormente, las pruebas por coinducción suelen ser infinitas. Esto hace que sea difícil convencer a los asistentes de pruebas de que las demostraciones son productivas y están bien definidas. 

\item El soporte para coinducción con tipos de tamaño definido (\textit{sized types}) ayuda al chequeo de terminación de programas de Agda, permitiendo realizar un número elevado de demostraciones que con el soporte de notación musical no eran posibles. Sin embargo, puede llegar a presentar problemas para reconocer y unificar valores que son iguales. 

\item Los números conaturales forman un monoide concurrente con la suma y el máximo como operaciones y el cero como elemento neutro. Esto quedó demostrado al generar la instancia de \AgdaRecord{ConcurrentMonoid} para el tipo \AgdaDatatype{Conat $\infty$}.

\item Si se tiene un conjunto $S$ que es un monoide concurrente, luego el funtor $\mathit{T}_S \ X = S \times X$ puede dotarse de una estructura de mónada concurrente. Esta implicancia se demostró en la prueba \AgdaFunction{cmonoid$\Rightarrow$cmonad}. 

\item La mónada \textit{writer} para los números conaturales constituye una mónada concurrente. La prueba \AgdaFunction{writerConatConcurrent} lo demuestra utilizando las pruebas mencionadas en los dos items anteriores. 
\end{enumerate}

Por cuestiones de tiempo y extensión de la tesina, quedaron algunas tareas pendientes para realizar más adelante. A continuación se detallan las principales:
\begin{enumerate}
\item Debido a la forma en la que se realizó la formalización de las mónadas concurrentes, al generar instancias de dicha estructura uno se ve obligado a utilizar la igualdad proposicional para los tipos de retorno. A futuro podía pensarse en modificar la formalización de manera que incluya como parámetro una noción de igualdad para el tipo de retorno. Así podrían darse instancias de mónadas concurrentes donde los valores de retorno se comparen mediante otros tipos de igualdad.

\item El soporte para coinducción utilizando \textit{sized types} parece más prometedor que el primero. Sería interesante intentar definir la mónada \textit{delay} utilizando una representación con dicho soporte y analizar si con esa representación se puede demostrar la ley de intercambio y, por lo tanto, probar que el tipo \textit{delay} constituye una mónada concurrente. 
\end{enumerate}

\begin{appendices}
\chapter{Referencia al código fuente}

El código fuente puede descargarse en el siguiente link: \href{https://github.com/ValeBini/Tesina/releases/tag/v1.0}{Código Tesina}. Allí se encuentra un archivo llamado \texttt{README.agda} que contiene un módulo \AgdaModule{README} donde se indica qué archivos se corresponden con cada parte del trabajo. La versión de Agda utilizada en el desarrollo del trabajo es la 2.6.1. 
%https://github.com/ValeBini/Tesina/tree/final-code

\end{appendices}

\backmatter

\chapter{Bibliograf\'{i}a}

%\bibliographystyleIntro{babalpha}
%\bibliographyIntro{bib/bibintro}
%\addcontentsline{toc}{section}{Referencias históricas}

%\newcommand{\nombredelabiblio}{Bibliograf\'{i}a general}
%\renewcommand\bibname{\nombredelabiblio}

\bibliographystyle{babalpha}
\bibliography{bib/biblio}
%\addcontentsline{toc}{section}{\nombredelabiblio}


\end{document}




